\documentclass[a4paper,oneside,14pt]{scrbook}

\usepackage[utf8]{inputenc}
\usepackage[russian]{babel}
\usepackage{indentfirst}
\usepackage[colorlinks=true]{hyperref}
\usepackage{graphicx}
\usepackage{amsmath}
\usepackage{guitar}
\usepackage{musixtex}

\usepackage{etex} %эта магическая херь избавляет от переполнения регистров TeX'а

%вставка изображений из metapost (post script)
\DeclareGraphicsRule{*}{mps}{*}{}

%пометка черновика (закомментировать три строки ниже для релиза)
\usepackage{draftwatermark}
\SetWatermarkScale{1.5}
\SetWatermarkText{$\beta$-версия от \today}

\title{О музыке и шестиструнной гитаре для аналитического ума}
\author{М.~М.~Шихов}
\date{\today}

\newtheorem{Example}{Пример}[chapter]
\newtheorem{Rule}{Правило}[chapter]
\newtheorem{Definition}{Определение}[chapter]
\newtheorem{Note}{Заметка}[chapter]

%определённые мной команды логической разметки
\newcommand{\myNemph}[1]{{\small{\emph{#1}}}}


\begin{document}
    \maketitle
    \tableofcontents

%    \chapter*{Введение}
\addcontentsline{toc}{chapter}{Введение}

\begin{music}
\startextract
\notes\qu{abcdefghi}\ql{jk}\enotes
\endextract
\end{music}
    %что-зачем
    \chapter{������? ���, �� �������\ldots}
\label{ch:music}

������� ����������� � ����������� ���� ����� � ���� �������� ����� 35. � ���� �������� � ������ ����, ��� �� ���� ��������� ���������. � ���� ��������� ����������. �����������. ����� � ������ � ����, ������������� ���������������� ������������� ��� ��� �������� ��������, ����� �������� ������������ ��������� �� ������������ �������: <<������, ������������ ��������� �� �������� ���������� ������ ������, � ����������� ������� --- �������>>. ��, �� ���� ������� �������� ��-����� ����� ������� � ��������� �� ����������� ������, � ��� ������ ������ ������� ����� (��������, � ���� ������ ���������, ������� �� ��������, � �.�.) �� �������� ����� ������ �� ����� �����. 

� ���� ��� �? ���������� �������� ���������� ������������ ��������� �������� �������� \emph{����������} �� �������� �������������. ���� ��� ������������� ��������, ��� ���������� ��������. 

� ���, ��������, �������� ���������������. � ����� ��������� ����� ���� � ����� �����������: ����� --- �������������, ������������, ����������� � ������������, � ������ --- �������������, ����������, ������������� � �������� � �������. � ��� � ����� ������� ���� ��������� ���������, ��� �� � ������: ������ ����� --- �����, � ������ --- �����. ����� ��������, � ����� �������������. ����� � �����, ����� � ���������. � ������ � ����� ������������ �� �������: � ���������� ��� �� �����, � �������� ������ ������!

� ��� �� ������ � ������ � �����������! ������ ���� �����������! ����������!!!

������, ������ ������� ���� ������� ������ ������, ��� ���������� ������ ���. ������ � ����� ������ �� ������ �����.


\section{��� ������ ������?}

������� ������.

\section{����������� �����}
\label{sec:music:sounds}

TODO: ����������-��������������� �����.

\section{���������� ������}
    %что есть музыка
%    \chapter{А на бумаге записать можно? Ноты, табы}
\label{ch:notes}

Нотную запись нельзя назвать эталоном простоты. Она, несомненно, сложнее, чем могла бы быть. Так уж сложилось.

Уважая гениальность предков, мы изучим её такой, какая она есть, и постораемся понять почему она такая.

% TODO:
%\begin{figure}[!ht]
%    \centering
%    \includegraphics{fig/intervals/octave-spiral} 
%    \caption{Цикличность музыкальных звуков}\label{fig:music:tone:octave}
%\end{figure} 
%
%ЛЯ-диез первой октавы, который на полутон выше эталонного ЛЯ, имеет частоту $440\cdot\sqrt[12]{2}\approx 466,16$ герц. Эти различия ощущаются на слух. Звук СИ первой октавы (два полутона от эталонного ЛЯ) имеет частоту $440\cdot(\sqrt[12]{2})^2\approx 493,88$. И так далее, например, ЛЯ второй октавы (на 12 полутонов, то есть не октаву выше эталонного) имеет, как и положено, в два раза большую частоту: $440\cdot(\sqrt[12]{2})^{12}=440\cdot 2=880$ Гц. Ряд музыкальных звуков по спирали <<наматывается>> на октаву (см. рисунок\footnote{Забегая немного вперед (иначе рисунок трудно понять), скажем, что гитаристы чаще обозначают ноты латинскими буквами: ДО(С), РЕ(D), МИ(E), ФА(F), СОЛЬ(G), ЛЯ(A), СИ(B). ЛЯ-диез, соответственно: A\#} \ref{fig:music:tone:octave}).


\section{А словами как назвать? Названия нот}
\label{ch:notes:names}

\emph{Нота} --- это \emph{буква} (знак) музыкального письма, позволяющая определить \emph{высоту} (частоту колебаний) и \emph{длительность} музыкального звука. 

В Русской традиции принято использовать следующие \emph{семь} названий нот, которые приведены в порядке возрастания частоты колебаний (высоты звука): 
\begin{center}
    ДО, РЕ, МИ, ФА, СОЛЬ, ЛЯ, СИ. 
\end{center}

Многие знают эту последовательность нот как детский стишок\footnote{Такие названия нот удобны тем, что их можно \emph{петь}. Тянуть последнюю гласную пока воздуха хватает! Человеку на самом деле не нужна гитара! Музыкальный инструмент всегда с собой --- голос. Друзья, поверьте, есть такие люди, которые могут петь по нотам. Их мало на нашей эстраде, но они есть! Только не надо думать, что если вы споете, например, ДО-О-О-О, то это прозвучит в унисон с пятой струной на третьем ладу. Фигушки. Не льстите себе. Такое может сделать только мастер. Он любую гласную вытянет в унисон с любым сыгранным не гитаре звуком!}. Эти ноты составляют октаву (и правда, чуть позже об октавах!).

Американцы и англичане называют ноты не как мы, а буквами латинского алфавита: 
\begin{center}
    A(ля), B(си), C(до), D(ре), E(ми), F(фа), G(соль).
\end{center}

И жили они, не тужили, с октавой, начинающейся с ЛЯ\ldots Ага, вроде всё логично и просто, как азбука\footnote{Латинские алфавит читается: A-эй, B-би, C-си, D-ди, E-и, F-эф, G-джи, H-эйч}. Как вдруг, в результате спонтанного шизоидного сдвига, всем вдруг стало ясно, что октава должна начинаться с ДО\footnote{Причина сего интересна не только психиатрам! Если вы разберетесь с \emph{музыкальным ладом}, то вы сами дадите этому заскоку рациональное объяснение}! И стало: CDEFGAB. А потом кому-то покорзилось, что для ноты СИ буквы не хватило! И назначили для ноты СИ очередную свободную латинскую букву H! Гитаристам с латиницей дело иметь придется часто, поэтому запомните приведенное соответствие и будьте готовы к тому, что для обозначения ноты СИ может быть использована латинская буква H\footnote{Все даже чуть сложнее. Если вы увидели книжку, где используется H, то знайте, что это СИ. Но не падайте в обморок, если вдруг увидите, что используется и латинская B. В данном клиническом случае B будет обозначать ноту CИ-бемоль!}.

Нота звучит \emph{выше}, если частота колебаний струны \emph{больше}. Ну а низкая нота --- это колебания с малой частотой. Например, в соответствии с международным стандартом, струна, звучащая на ноте Ля первой октавы (об октавах скажу позже) колеблется с частотой 440 герц, то есть совершает 440 полных колебаний в секунду\footnote{Допускается вольность принять за Ля первой октавы любую частоту из интервала от 430 до 450 герц. Многие фанатики утверждают, что Ля в 432 Гц одобрена Богом (и оздоравливающе действует на организм), а стандартизированная 440 Гц --- от Сатаны (и разрушает психику)}.

Допустим, с названиями нот все понятно\ldots Так что получается, музыкальных звуков всего семь? Конечно нет! Пусть мы дошли до последней ноты СИ и... И текущая \emph{октава} кончилась. Но началась следующая! В следующей октаве все повторится: ДО, РЕ, МИ, ФА, СОЛЬ, ЛЯ, СИ. Только вот частота звука для соотвествующей ноты будет в \emph{два} раза больше! То есть, например, ноте Ля второй октавы соответствует вдвое большая частота (880 Гц), чем ноте Ля первой октавы (стандартные 440 Гц).

Октавы, использующиеся в музыке, имеют следующие названия.
\begin{center}
    \begin{tabular}{ll}
        \hline\hline
        Название октавы         & Частота ноты Ля, Гц \\
        \hline\hline
        
        Субконтроктава          & 27.5 \\
        Контроктава             & 55   \\
        \emph{Большая октава}   & 110  \\
        \emph{Малая октава}     & 220  \\
        \emph{Первая октава}    & \fbox{440}  \\
        \emph{Вторая октава}    & 880  \\
        Третья октава           & 1760 \\
        Четвертая октава        & 3520 \\
        Пятая октава            & 7040 \\
        \hline
    \end{tabular}
\end{center}

В таблице \emph{выделены} те октавы, которые входят в диапазон шестиструнной гитары. Справедливости ради следует сказать, диапазон звучания шестиструнной гитары полностью включает лишь малую и первую октавы.

А вот теперь пришло время для секретов: традиционно в октаве принято выделять \emph{двенадцать} звуков! 

Погодите, погодите, скажете вы: <<До, Ре, Ми, Фа, Соль, Ля, Си>> --- семь нот! И октава названа, вероятно, не просто так: <<octo>> --- это восемь, восьмая нота\footnote{Почему октава названа октавой см. в разделе \ref{sec:harmony:interval}. На самом деле октава --- это определенное расстояние между звуками}! Все логично!

Люди в шоке, но молчат! <<До, Ре, Ми,\ldots>> --- это \emph{названия} лишь семи звуков из 12, составляющих октаву. (12-7)=5 звуков не удостоились отдельных имен. Итак, оставшиеся 5 звуков находятся между:
\begin{enumerate}
    \item ДО и РЕ (этот звук может быть назван либо ДО-диез, либо РЕ-бемоль)
    \item РЕ и МИ (РЕ-диез или МИ-бемоль)
    \item ФА и СОЛЬ (ФА-диез или СОЛЬ-бемоль)
    \item СОЛЬ и ЛЯ (СОЛЬ-диез или ЛЯ-бемоль)
    \item ЛЯ и СИ (ЛЯ-диез или СИ-бемоль)
\end{enumerate}    

Как видно, эти <<промежуточные>> ноты могут называться двояко. Какое из имен выбрать? Постараюсь ответить кратко: если вы читаете этот текст и узнаёте для себя что-то новое, то вам позволительно использовать любое\footnote{Это как грамотность в Русском языке: надеть или одеть? Одеть Надежду, надеть одежду! Чужак не осилит. Так что называйте как хотите, знающие друзья поправят, если что. Только помалкивайте на какой-нибудь музыкальной конференции в кругу маститых классических исполнителей}! Все 12 звуков октавы в порядке увеличения высоты:
\begin{center}
    ДО, \emph{ДО-диез}, РЕ, \emph{РЕ-диез}, МИ, ФА, \emph{ФА-диез}, СОЛЬ, \emph{СОЛЬ-диез}, ЛЯ, \emph{ЛЯ-диез}, СИ
\end{center}

Отметим, что между МИ и ФА, а также между СИ и ДО, промежуточных звуков нет\footnote{Вам уже давно хочется сказать: <<Какого черта?>>. Почему? Зачем эти сложности? Друзья мои, болею за вас! Творится полный беспредел с точки зрения кодирования. Можно проще, но все привыкли. Традиция}.

В учебниках говорится, что добавление суффикса <<диез>> означает повышение частоты исходной ноты на <<полутон>>, что математически соответствует увеличению частоты в $\sqrt[12]{2}$ раз. И это в целом правильно, только обычно ученики начинают думать, что между соседними нотами из списка ДО, РЕ, МИ, ФА, СОЛЬ, ЛЯ, СИ --- расстояние в два <<полутона>> (то есть в <<тон>>)! Не забыайте, что между нотами МИ и ФА, а также между СИ и ДО <<промежуточных>> нот нет и расстояние между ними --- один <<полутон>>. 

Как видно из реальной последовательности, следуя такой логике, например, МИ-диез --- это ФА, или СИ-диез --- это ДО следующей октавы. Обычно ноту ФА, конечно никто не называет МИ-диез --- это оскорбительно, но никто и не запрещает так делать. 

То же самое можно сказать и о суффиксе <<бемоль>>, он понижает ноту на полутон. И, например, До-диез, это тот же звук, что и РЕ-бемоль. А если вы хотите оскорбить ноту МИ, то назовите её ФА-бемоль. Если вы хотите назвать все 12 звуков октавы в обратном порядке, то грамотно будет использовать суффикс <<бемоль>>, а не <<диез>>:

\begin{center}
СИ, СИ-бемоль, ЛЯ, ЛЯ-бемоль, СОЛЬ, СОЛЬ-бемоль, ФА, МИ, МИ-бемоль, РЕ, РЕ-бемоль, ДО.
\end{center}


\section{Что тут зашифровано? Запись нот}

Чтобы записать ноты, купите нотную тетрадь или на обычном листе начертите нотоносец --- пять параллельных, расположенных друг под другом через равные интервалы (около двух миллиметров) линий:
 

Традиционно ноты для шестиструнной гитаре записываются в скрипичном ключе, который своим хвостиком огибает вторую снизу линию нотоносца, на которой располагется СОЛЬ \emph{малой} октавы\footnote{Для других музыкальных инструментов, в первую очередь, для фортепиано, скрипичный ключ показывает положение СОЛЬ \emph{первой} октавы, но для гитары, чтобы использовать один нотоносец, ноты пишут на <<фортепианный>> скрипичный нотоносец на октаву выше.}.

\section{Как это сыграть на гитаре? Гитарная табулатура}



%TODO ритм и переборы    %о нотной грамоте пару ласковых
    \chapter{Устройство гитары}
\label{ch:guitar}

Частота от звука к звуку повышается в \emph{геометрической} прогрессии. То есть частота каждого следующего музыкального звука в \[\sqrt[12]{2}\approx 1,059463\] больше частоты звука предыдущего.

Так, следующий за ЛЯ первой октавы, звук ЛЯ-диез, имеет частоту $440\cdot\sqrt[12]{2}\approx 466,16$ герц. Звук СИ имеет частоту $440\cdot(\sqrt[12]{2})^2\approx 493,88$. И так далее, например, ЛЯ второй октавы имеет, как и положено, в два раза большую частоту, чем ЛЯ первой октавы: $440\cdot(\sqrt[12]{2})^{12}=440\cdot 2=880$ Гц.

Задав эталонную частоту любого музыкального звука, частоты для всех остальных звуков можно \emph{вычислить}.


TODO: ноты на грифе

\section{Устройство гитары}
\label{ch:guitar:construction}

Мы разобрались с теорией нот в предыдушем разделе, теперь коснемся особенностей устройства гитары, которые позволяют извлекать ноты именно такими, какими они и должны быть. 

Для начала стоит взглянуть на рисунок \ref{fig:guitar:construction} и запомнить, что значат незнакомые вам обозначения.

\begin{figure}[!ht]
    \centering
    \includegraphics{fig/guitar-construction} 
    \caption{Устройство гитары}\label{fig:guitar:construction}
\end{figure} 

Любая открытая\footnote{То есть не зажатая ни на каком ладу} струна гитары звучит строго определенной нотой (о настройке гитары поговорим позже). Лады на грифе (промежутки между порожками), равно как и \emph{ладовые порожки} считаются от \emph{верхнего} порожка: 1,2,3,\ldots и т.д. То есть <<зажать струну на первом ладу>> значит, что вы ставите палец на струну, на первый лад, то есть между верхним порожком и первым ладовым\footnote{Чем ближе к первому ладовому, тем лучше. Таким образом и звук будет чище, и рука уставать будет меньше. Ставить палец сверху на порожек не стоит --- звук будет <<глохнуть>>. Правда иногда именно это и требуется. Но в начале обучения стоит ставить палец на ладу ближе к тому порожку, от которого идет <<звучащая>> часть струны. Добивайтесь чистого звука.} и нажимаете до тех пор, пока струна не прижмётся к первому ладовому порожку. Но нам важно сейчас не то, как правильно зажимать струну. 

Важно понять, что каждый следующий лад повышает звук на струне на <<полутон>>. В октаве 12 нот и каждая звучит на струне на своём ладу. На дветадцатом ладу звучит нота открытой струны, только выше на октаву.

Из физики известно, что частота колебаний струны обратно пропорциональна её длине\footnote{Надо честно заметить, что частота колебаний струны зависит также и от силы её натяжения, которая меняется, когда струну <<зажимают>> на ладу. Но это влияние столь незначительно, что им можно пренебречь.}. Стало быть, чтобы частота издаваемого струной звука \emph{увеличилась} вдвое (а языком музыки --- чтобы нота зазвучала октавой выше), надо вдвое \emph{укоротить} струну. 

Зажимая струну на 12 ладу (языком музыки --- повышая ноту открытой струны на октаву), вы укарачиваете звучащую часть струны вдвое. Линейка в помощь, если не верите\footnote{Конечно нужно мерять только звучащую (колеблющуюся часть) струны от опоры на подставке до 12-го ладового порожка.}.

Конечно, частота колебаний струны зависит также и от силы её натяжения. Сила натяжения струны регулируется колками на грифе, когда гитару настраивают. Играя, гитарист только меняет длину звучащего участка струны, зажимая струны на ладах. Редкие психи\footnote{Конечно, имелось в виду: \emph{мастера}! Прим. ред.} крутят колок во время исполнения, добиваясь сомнительных\footnote{Конечно, имелось в виду: \emph{удивительных}! Прим. ред.} эффектов.

Исходя из того, что частота каждой следующей ноты в $\sqrt[12]{2}$ больше предыдущей, запишем формулу длины струны ($L$) от места крепления струны к подставке до $n$-го ладового порожка:

\[L(n)=\frac{L}{(\sqrt[12]{2})^n},\]
где $n$ - номер лада ($0$-й лад соответствует открытой струне), а $L$ --- общая длина струны от подставки до верхнего порожка.


\includegraphics{fig/string-length.png}


Из картинки, надеюсь, ясно, почему ладовые порожки на гитаре расположены не на равном расстоянии друг от друга.

Кстати, некоторые ушастые выпендрёжники говорят, что различают своим сверхмузыкальным слухом больше 12 нот в октаве! И им мало 12 ладов! Есть спрос --- есть предложение: на некоторых гитарах можно заметить дополнительные ладовые порожки между <<каноническими>>, которые позволяют <<всунуть>> дополнительную ноту.


\section{Запись гитарных нот на бумаге}

Чтобы записать ноты, купите нотную тетрадь или на обычном листе начертите нотоносец --- пять параллельных, расположенных друг под другом через равные интервалы (около двух миллиметров) линий:
 

Традиционно ноты для шестиструнной гитаре записываются в скрипичном ключе, который своим хвостиком огибает вторую снизу линию нотоносца, на которой располагется Соль \emph{малой} октавы\footnote{Для других музыкальных инструментов, в первую очередь, для фортепиано, скрипичный ключ показывает положение Соль \emph{первой} октавы, но для гитары, чтобы использовать один нотоносец, ноты пишут на <<фортепианный>> скрипичный нотоносец на октаву выше.}.


\section{Поиск нот на грифе}

\begin{figure}[!ht]
    \centering
    \includegraphics[width=\textwidth]{fig/lad-by-notes} 
    \caption{Ноты на грифе (гриф поперек нотоносца)}\label{fig:ladByNotes}
\end{figure} 

\begin{figure}[!ht]
    \centering
    \includegraphics[width=\textwidth]{fig/lad-by-griph} 
    \caption{Ноты на грифе (гриф вдоль нотоносца)}\label{fig:ladByGriph}
\end{figure} 

\begin{figure}[!ht]
    \centering
    \includegraphics[width=\textwidth]{fig/notes-on-griph} 
    \caption{Ноты на грифе (относительное расположение)}\label{fig:notesOnGriph}
\end{figure} 



   %об устройтсве гитары
%    \chapter{А так звучит красиво? Гармоничность}
\label{ch:harmony}


Подчиняется ли наше восприятие музыки математическим законам? Что определяет наши вкусы:
\begin{itemize}
    \item законы физики;
    \item индивидуальные особенности личности;
    \item сформировавшиеся за многие годы общие культурные привычки;
    \item что-то еще?
\end{itemize}

В общем и целом --- да, подчиняется. Как уже было сказано --- математика внутри нас. Тому, что нам нравится, а тем более тому, что не нравится, есть естественное объяснение. Наши предки нашли и оставили нам многие законы гармонии, в соответствии с которыми музыкальные гении создавали и создают музыку. Нельзя сбрасывать со счетов особенности психики отдельного человека. Да, к определенной музыке мы просто привыкли.

И конечно же есть что-то еще, пока непознанное, что всегда удобно назвать словом \emph{чудо}!


\section{Сколько вешать в полутонах? Интервалы}
\label{ch:harmony:interval}

\begin{Definition}[Интервал]
    \emph{Интервал} --- это расстояние между \emph{двумя} музыкальными звуками, выраженное в полутонах. 
\end{Definition}

Это определение для математиков. Для музыкантов интервалы \emph{звучат}! Если два звука прозвучали одновременно, то интервал называется \emph{гармоническим}, а если друг за другом --- \emph{мелодическим}.

\begin{Example}[Послушаем гармонические интервалы]
    \label{ex:harmony:interval:string5and6}
    Возьмите настроенную гитару. Поиграем на 5 и 6-й струне. Если гитара настроена страндартно, то 6-я струна на 5-м ладу прозвучит в унисон с открытой 5-й струной. Такое расстояние в 0 полутонов музыканты называют \emph{примой}. 
    
    Инструмент обязательно должен быть настроен! Иначе эксперимент не получится.
    
    Теперь расслабьтесь, успокойтесь, забудьте обо всех горестях и радостях. Сосредоточьтесь на своем дыхании. Существует только ваше дыхание. Абсолютный покой. 
    
    Не получается? Чёрт с ним!

    Вам нужно будет оценить свои ощущения от сыгранных интервалов. Поставьте каждому интервалу оценку, например по 5-и балльной шкале: 1(ужос), 2(срам), 3(терпимо), 4(хорошо), 5(прекрасно).
    
    Начнем. Зажимите 6-ю струну на 5-м ладу и одновременно сыграйте две струны: 6-ю и 5-ю. Звучит гармонический интервал \emph{прима}! Интервал в ноль полутонов. Прислушиваемся к ощущениям, ставим приме оценку.
    
    Продолжаем оценивать интервалы. Играем одновременно 5-ю открытую струну и 6-ю струну на 6-м ладу. Расстояние в 1 полутон. Оценивайте результат.
    
    И так далее, играем интервал в 2 полутона (6-я струна на 7-м ладу) и так далее до интервала в 12 полутонов (6-я струна на 17 ладу). 
    
    Читая этот раздел, периодически поглядывайте на свои записи. Будет полезно.
\end{Example}

Приятный на слух интервал (неважно, гармонический или мелодический) музыканты называют \emph{консонансом}, а неприятный --- \emph{диссонансом}. 

Чтобы глубже разобраться в том, почему звучание одного интервала нам нравится, а другого --- нет, визуализируем результат наложения двух звуков. Пусть функция $\sin(x)$ изображает основной тон исходного звука. Функция $\sin(x\cdot(\sqrt[12]{2})^n)$ будет изображать звук, который \emph{выше} исходного на $n$ \emph{полутонов}. Результату совместного звучания (то есть гармоническому интервалу) будет соответствовать их сумма\footnote{В Интернете масса сайтов, позволяющих построить график функции. Более того, просто вбейте в гугл \texttt{sin(x)+sin(x*(2\^{}(1/12)))} и дивитесь чудесам Технологии!}:

\begin{equation}
    \label{eq:harmony:interval:sin}
    \sin(x) + \sin(x\cdot(\sqrt[12]{2})^n).
\end{equation}

Результаты построения графиков совместного звучания (построен красным цветом) на фоне исходного звука (синий цвет) для интервалов от $n=1$ до $n=11$ полутонов (т.е. в рамках октавы) приведены в таблицах \ref{t:harmony:interval:disso-1-2-10-11}, \ref{t:harmony:interval:conso-3-4-8-9}, \ref{t:harmony:interval:conso-5-7}, \ref{t:harmony:interval:disso-6}.

Диссонансы приведены в таблицах \ref{t:harmony:interval:disso-1-2-10-11} и \ref{t:harmony:interval:disso-6}. Диссонансы получаются на интервалах в 1,2,6,10 или 11 полутонов. Значит вам не должны были понравиться интервалы на 6,7,11,15,16 ладах 6-й струны, если вы обратили внимание на пример \ref{ex:harmony:interval:string5and6}. Только не говорите, что понравились! Пора сходить к доктору!!! Не затягивайте. 

\begin{table}[!ht]
    \caption{Диссонансы в графиках функции \eqref{eq:harmony:interval:sin}}
    \label{t:harmony:interval:disso-1-2-10-11}
    \centering
    \begin{tabular}{c|c}
        \hline\hline
        1 полутон, $n=1$        & 2 полутона, $n=2$ \\
        малая секунда           & большая секунда \\
        \includegraphics[width=0.45\textwidth]{fig/intervals/i01}
            & \includegraphics[width=0.45\textwidth]{fig/intervals/i02} \\
        \hline\hline
        10 полутонов, $n=10$    & 11 полутонов, $n=11$ \\
        малая септима           & большая септима \\
        \includegraphics[width=0.45\textwidth]{fig/intervals/i10}
            & \includegraphics[width=0.45\textwidth]{fig/intervals/i11} \\
        \hline\hline
    \end{tabular}
\end{table}

Приглядитесь к приведенным графикам и попробуйте самостоятельно сделать выводы о причинах благозвучия консонансов и некоей хаотичности диссонансов. Консонансы, кстати, принято разделять на:
\begin{itemize}
    \item \emph{абсолютные}. Это полностью сливающиеся на слух интервалы в 0 полутонов (\emph{прима}, если помните) или кратные 12-ти полутонам. О причинах слияния этих звуков мы поговорили в самом начале, см. раздел \ref{ch:music:tone}. Интервал в 12 полутонов называется \emph{октавой}.
    
    \item \emph{совершенные}. Расстояние между звуками составляет 5 или 7 полутонов. См. таблицу \ref{t:harmony:interval:conso-5-7}. Смело можете понизить или повысить любой из звуков такого интервала на одну или несколько октав (12 полутонов) и совершенный консонанс останется\footnote{Уже заметили, что 5+7=12?}. Эти интервалы не сливаются на слух, но звучат благозвучно. Отличник среди консонансов.
    
    \item \emph{несовершенные}. Звуки не сливаются, точно не диссонанс, но и не совершенный консонанс. Короче, консонанс с помарочкой. Когда мы слышим несовершенный консонанс, то хочется, чтобы он побыстрее перешел консонанс совершенный, стал отличником. Разница между звуками составляет 3, 4, 8 или 9 полутонов. Точно так же, любой звук такого интервала можно понизить или повысить на октаву и несовершенство останется.
\end{itemize}


\begin{table}[!ht]
    \caption{Несовершенные консонансы в графиках функции \eqref{eq:harmony:interval:sin}}
    \label{t:harmony:interval:conso-3-4-8-9}
    \centering
    \begin{tabular}{c|c}
        \hline\hline
        3 полутона, $n=3$   & 4 полутона, $n=4$ \\
        малая терция        & большая терция \\
        \includegraphics[width=0.45\textwidth]{fig/intervals/i03}
            & \includegraphics[width=0.45\textwidth]{fig/intervals/i04} \\
        \hline\hline
        8 полутонов, $n=8$  & 9 полутонов, $n=9$ \\
        малая секста        & большая секста \\
        \includegraphics[width=0.45\textwidth]{fig/intervals/i08}
            & \includegraphics[width=0.45\textwidth]{fig/intervals/i09} \\
        \hline\hline
    \end{tabular}
\end{table}

\begin{table}[!ht]
    \caption{Совершенные консонансы в графиках функции \eqref{eq:harmony:interval:sin}}
    \label{t:harmony:interval:conso-5-7}
    \centering
    \begin{tabular}{c|c}
        \hline\hline
        5 полутонов, $n=5$  & 7 полутонов, $n=7$ \\
        чистая кварта       & чистая квинта \\
        \includegraphics[width=0.45\textwidth]{fig/intervals/i05} 
            & \includegraphics[width=0.45\textwidth]{fig/intervals/i07} \\
        \hline\hline
    \end{tabular}
\end{table}

\begin{table}[!ht]
    \caption{Диссонанс в графике функции \eqref{eq:harmony:interval:sin}}
    \label{t:harmony:interval:disso-6}
    \centering
    \begin{tabular}{c}
        \hline\hline
        6 полутонов, $n=6$ \\
        \includegraphics[width=0.45\textwidth]{fig/intervals/i06} \\
        увеличенная кварта,\\
        она же --- уменьшенная квинта,\\
        он же --- тритон\\
        \hline\hline
    \end{tabular}
\end{table}

Звуки октавы удобно зациклить и изобразить на окружности. На рисунке \ref{fig:harmony:interval:oct-round} изображены ноты октавы и отмечены консонансы и диссонансы от ноты ДО (т.е. C --- ноты обозначены в латинской нотации). Кстати, можно сделать удобный приборчик для определения консонансов и диссонансов, если сделать кружок с нотами подвижным. Но погодите. Если уж делать, то квинто-квартовый круг (см. раздел \ref{ch:harmony:kvinto-kvarto-round}).

\begin{figure}[!ht]
    \centering
    \includegraphics[width=\textwidth]{fig/intervals/octave-round} 
    \caption{Интервалы от ноты ДО}\label{fig:harmony:interval:oct-round}
\end{figure} 

Осталось разобраться какого лешего интервалы называются так странно? Нет никакой видимой связи между названиями интервала и количеством полутонов, его составляющих! Справочник\footnote{Для тех, кому зубрежка покажется проще понимания} по интервалам см. в таблице \ref{t:harmony:interval:names}. Собственно, ответ прост: по историческим причинам название интервала отражало не количество полутонов, а номер ступени мажорного музыкального лада (о ладах см. раздел \ref{ch:harmony:lad}). Так как каждая ступенька мажорного лада состояла из одного или двух полутонов, то определить количество полутонов по названию интервала без достаточного опыта затруднительно, если не представить в уме рисунок \ref{fig:harmony:interval:names}.

\begin{table}[!ht]
    \caption{Интервалы}
    \label{t:harmony:interval:names}
    \centering
    \begin{tabular}{l|l|l|c|l}
        \hline\hline
        Название интервала & Перевод            &               & Количество  & Кратко  \\
                           & на русский         &               & полутонов   &         \\
        \hline\hline
        Прима(prima)       & Первая (ступень)   & Чистая        & 0                 & ч.1 \\
        Секунда(secunda)   & Вторая             & Малая         & 1                 & м.2 \\
                           &                    & Большая       & 2                 & б.2 \\
        Терция(tertia)     & Третья             & Малая         & 3                 & м.3 \\
                           &                    & Большая       & 4                 & б.3 \\
        Кварта(quarta)     & Четвертая          & Чистая        & 5                 & ч.4 \\
                           &                    & Увеличенная   & 6                 & ув.4\\
        Квинта(quinta)     & Пятая              & Уменьшенная   & 6                 & ум.5\\
                           &                    & Чистая        & 7                 & ч.5 \\
        Секста(sexta)      & Шестая             & Малая         & 8                 & м.6 \\
                           &                    & Большая       & 9                 & б.6 \\
        Септима(septima)   & Седьмая            & Малая         & 10                & м.7 \\
                           &                    & Большая       & 11                & б.7 \\
        Октава(octava)     & Восьмая            & Чистая        & 12                & ч.8 \\
        \hline\hline
        Нона(nona)         & Девятая            & Малая         & 13                & м.9  \\
                           &                    & Большая       & 14                & б.9  \\
        Децима(decima)     & Десятая            & Малая         & 15                & м.10 \\
                           &                    & Большая       & 16                & б.10 \\
        Ундецима           & Одиннадцатая       & Чистая        & 17                & ч.11 \\
                           &                    & Увеличенная   & 18                & ув.11\\
        Дуодецима          & Двенадцатая        & Уменьшенная   & 18                & ум.12\\
                           &                    & Чистая        & 19                & ч.12 \\
        Терцдецима         & Тринадцатая        & Малая         & 20                & м.13 \\
                           &                    & Большая       & 21                & б.13 \\
        Квартдецима        & Четырнадцатая      & Малая         & 22                & м.14 \\
                           &                    & Большая       & 23                & б.14 \\
        Квинтдецима        & Пятнадцатая        & Чистая        & 24                & ч.15 \\
        \hline\hline
    \end{tabular}
\end{table}

\begin{figure}[!ht]
    \centering
    \includegraphics{fig/intervals/interval-names} 
    \caption{Исторически имена интервалов --- это имена ступеней мажорного лада}\label{fig:harmony:interval:names}
\end{figure} 

Тогда все становится относительно просто. Например, терция, это расстояние от ноты ДО (первая ступень мажорного лада) до ноты МИ (третья ступень). Считаем ДО-РЕ --- 2 полутона, РЕ-МИ --- 2-а полутона. Получилось 4-е полутона. Только вот вспоминается, что терция бывает <<большая>> и <<малая>>. При таком подходе мы всегда будем получать значение для <<большого>> и <<чистого>> интервалов. Для <<малого>> или <<уменьшенного>> интервалов нужно уменьшить получившееся число на 1, а для <<увеличенного>> --- увеличить на 1. Значит: большая терция --- 4 полутона, малая --- 3.

Закрепим. Например, квинта: расстояние ДО-СОЛЬ --- 7 полутонов. Уменьшенная квинта --- 6 полутонов, чистая --- 7.


\section{Лады? Лады}
\label{ch:harmony:lad}

\begin{Definition}[Лад]
    \emph{Лад}\footnote{На английском \emph{лад} --- \emph{mode}. Режим работы, способ, вид, метод} --- это интервальный шаблон, позволяющий из 12-и последовательных музыкальных звуков октавы выбрать \emph{условно} <<правильные>>. 
\end{Definition}

Если это определение показалось вам тяжеловатым, почитайте учебники или Википедию. От некоторых определений веет такой суровой философией, что хочется курить в глубокий затяг. 

Задача лада: из 12 музыкальных звуков, составляющих октаву, выбрать лишь несколько таких, которые можно играть в любом порядке и все равно будет МУЗЫКА! Задача не из тривиальных и кажется весьма субъективной, ведь всегда найдется кто-то, кто скажет: <<А мне не нравится!>>. 

Однако эта задача была решена\footnote{Не исключено, что кем-то она решается и в данный момент} предками неоднократно, и в культурном наследнии мы имеем немало ладов, самыми известными из которых являются \emph{мажорный} и \emph{минорный}.

Мажорный и минорный лады --- лады \emph{семиступенные}. То есть такой лад выбирает из 12 звуков октавы только 7.

\paragraph{Мажорный лад.} Начнем с мажорного лада, интервальная структура которого приведена на рисунке \ref{fig:harmony:lad:mode:maj}. Тёмными кружками обозначены <<выбранные>> ладом звуки --- \emph{ступени} лада. Например, вторая ступень мажорного лада находится на расстоянии 2-х полутонов от первой. Интервалы (в полутонах) между ступенями \emph{мажорного} лада расположены так:

\[
    \texttt{2-2-1-2-2-2-1}
\]

Всем с детства знакомое ДО, РЕ, МИ, ФА, СОЛЬ, ЛЯ, СИ есть не что иное, как 7 идеальных ноток, отобранных мажорным ладом, начиная от ноты ДО. Проверьте: (ДО-РЕ)=2 полутона, (РЕ-МИ)=2, (МИ-ФА)=1, (ФА-СОЛЬ)=2 и т.д. 

Интересно то, что уникальные имена получили только 7 нот, а остальные 5 нот октавы имеют производные имена (с суффиксом <<бемоль>> или <<диез>>) --- есть следствие использования ладов.

\begin{figure}[!ht]
    \centering
    \includegraphics{fig/intervals/mode-maj} 
    \caption{Интервальная структура мажорного лада}\label{fig:harmony:lad:mode:maj}
\end{figure} 

Таким образом, шаблон лада может накладываться на любую ноту, любой музыкальный звук. Сплошная теория относительности!

Когда первая ступень лада накладывается на определенную ноту, то набор нот, попавших на ступени лада, образует \emph{тональность}\footnote{На английском \emph{тональность} --- tonality}. Допустим, мы совместили первую ступень мажорного лада с нотой ДО, тогда мы получим тональность <<ДО-мажор>>: 
\[
    \text{ДО}\xrightarrow{2}
    \text{РЕ}\xrightarrow{2}
    \text{МИ}\xrightarrow{1}
    \text{ФА}\xrightarrow{2}
    \text{СОЛЬ}\xrightarrow{2}
    \text{ЛЯ}\xrightarrow{2}
    \text{СИ}\xrightarrow{1}
\]

Название тональности складывается из названия ноты, попавшей на первую ступень и названия лада. Обычно мелодия составляется только из семи нот, входящих в тональность. Так и говорят, например, мелодия в тональности <<ЛЯ-минор>>.

\begin{Example}[Тональность <<РЕ-мажор>>]
    \label{ex:harmony:lad:d:maj}
    
    Чтобы получить ноты в тональности РЕ-мажор, нам нужно совместить ноту РЕ и первую ступень мажорного лада. Отступаем два полутона, и на вторую ступень попадет нота МИ. На третью --- ФА-диез.
    
    Целиком:
    \[
        \text{РЕ}\xrightarrow{2} 
        \text{МИ}\xrightarrow{2} 
        \text{ФА-диез}\xrightarrow{1} 
        \text{СОЛЬ}\xrightarrow{2} 
        \text{ЛЯ}\xrightarrow{2} 
        \text{СИ}\xrightarrow{2} 
        \text{ДО-диез}\xrightarrow{1}
    \]
    
    В эту тональность попали нотки, имеющие производные названия: ФА-диез, ДО-диез.
\end{Example}

Задача определить ноты, входящие в ту или иную тональность, а также количество диезов и бемолей, является любимой пыткой среди музыкальных инквизиторов. Сдвинуть шаблончик --- дело плёвое. А вот ноты после этого назвать --- уже подвиг! Совершенно искусственная проблема, растущая только от принятого способа обозначать ноты.

Например, для певца, поющего по нотам\footnote{Да, есть люди которые могут делать такие штуки со своим голосом: тянуть гласные с нужной частотой основного тона} чтобы перейти из тональности <<ДО-мажор>> в <<РЕ-мажор>> достаточно каждую исходную нотку спеть двумя полутонами выше (сдвинуть шаблон) и не думать о том, какая нота получается в итоге (закодировать название ноты).

Короче, с практической вещи все проще, чем с теоретической!

Чтобы сыграть \emph{гамму}\footnote{Слово \emph{гамма} в русском очень похоже на \emph{Game} (игра) в английском. И вроде бы логично: гамма --- это то, что \emph{играется}! Но \emph{гамма} на английском --- \emph{scale}. Шкала, звукоряд} в заданной тональности нужно:
\begin{itemize}
    \item начать с ноты первой ступени;
    \item продолжить играть ноты тональности в порядке возрастания (или убывания) высоты;
    \item сыграв таким образом одну или несколько октав, закончить на ноте первой ступени (естественно уже в другой октаве); 
    \item (необязательно) проиграть только что сыгранную последовательность в обратном порядке.
\end{itemize}

Например, гамма в тональности <<ДО-мажор>> или просто <<гамма ДО-мажор>> это известное: 
\begin{center}
    ДО, РЕ, МИ, ФА, СОЛЬ, ЛЯ, СИ, ДО, СИ, ЛЯ, СОЛЬ, ФА, МИ, РЕ, ДО.
\end{center}


\paragraph{Минорный лад.} Интервальная структура \emph{минорного} лада приведена на рисунке \ref{fig:harmony:lad:mode:min}. Интервалы (в полутонах) между ступенями \emph{минорного} лада расположены так:
\[
    \texttt{2-1-2-2-1-2-2}
\]

\begin{figure}[!ht]
    \centering
    \includegraphics{fig/intervals/mode-min} 
    \caption{Интервальная структура минорного лада}\label{fig:harmony:lad:mode:min}
\end{figure} 

Заметьте, что если замкнуть минорную интервальную структуру в кольцо и немного повращать (а это можно сделать, так как в следующей октаве все повторится), то получится мажорный лад. Совместите первую ступень мажорного лада и третью минорного и убедитесь, что в принципе структура этих ладов одна и та же. 

Например, давайте положим в первую ступень минора ноту ЛЯ. Получим тональность, состоящую из нот:
\begin{center}
    ЛЯ, СИ, ДО, РЕ, МИ, ФА, СОЛЬ.
\end{center}

Ноты в тональности <<ЛЯ-минор>> те же, что и в <<ДО-мажор>> (как видно ни одной нотки с бемолем или диезом). Поэтому тональности <<ДО-мажор>> и <<ЛЯ-минор>> называются \emph{параллельными}. Как нетрудно догадаться, параллельных тональностей столько же, сколько фактических нот в октаве: 12. А вот гамму <<ЛЯ-минор>>:
\begin{center}
    ЛЯ, СИ, ДО, РЕ, МИ, ФА, СОЛЬ, ЛЯ, СОЛЬ, ФА, МИ, РЕ, ДО, СИ, ЛЯ
\end{center}
с гаммой <<ДО-мажор>> точно на слух не спутаешь!

\paragraph{Современные 7-ступенные лады.} Эти лады имеют сходную интервальную структуру и называются диатоническими\footnote{Диатонические лады или просто <<диатоника>> --- это система семиступенных ладов, постоенных из пяти интервалов величиной в два полутона, и двух полутоновых интервалов. $5\cdot2 + 2\cdot 1 = 12$ --- октава}. Мажор и минор --- также диатонические лады. На рисунке \ref{fig:harmony:lad:modes} иображена октава, разделенная на 12 полутонов. Лады отличаются друг от друга только тем, откуда начинается первая ступень. Каждый из семи возможных вариантов имеет собственное название. Например, первая ступень <<лидийского>> лада начинается с 4-й отметки на октаве, и, обойдя от 4-й отметки всю окраву, легко получить его интервальную структуру:
\[
    \texttt{2-2-2-1-2-2-1}
\]

\begin{figure}[!ht]
    \centering
    \includegraphics[width=\textwidth]{fig/intervals/modes} 
    \caption{Интервальная структура диатонических ладов}\label{fig:harmony:lad:modes}
\end{figure} 

<<Лидийский>> лад активно используется в джазовой музыке. Если вы хотите сыграть <<ФА-лидийскую>> гамму, то достаточно проиграть октаву:
\begin{center}
    ФА, СОЛЬ, ЛЯ, СИ, ДО, РЕ, МИ, ФА, МИ, РЕ, ДО, СИ, ЛЯ, СОЛЬ, ФА
\end{center}
 
построенная от ноты фа, <<ФА-лидийская>> тональность содержит (как видно из рисунка \ref{fig:harmony:lad:modes}) ноты без диезов и бемолей. Кстати, этот лад для мелодий, дарящих ощущение счастья.

\paragraph{Пентатоника.} Пентатоника --- это тоже лад, но имеющий только 5-ступеней. То есть пентатоника из 12 нот выделяет только 5 правильных, из которых можно составлять мелодию. Аналогично 7-ступенным ладам, получают 5 вариантов ладов с различными названиями, представленные на рисунке \ref{fig:harmony:lad:pentatonic}.

\begin{figure}[!ht]
    \centering
    \includegraphics[width=\textwidth]{fig/intervals/pentatonic} 
    \caption{Интервальная структура пентатоники}\label{fig:harmony:lad:pentatonic}
\end{figure} 

Соответственно, например, интервальная структура мажорной пентатоники в полутонах:
\[
    \texttt{2-2-3-2-3}
\]


\section{Чем больше, тем лучше? Аккорды}
\label{ch:harmony:chords}

TODO

\section{Так квинтовый или квартовый? Квинто-квартовый круг}
\label{ch:harmony:kvinto-kvarto-round}

Замечателен тот факт (см. рисунок \ref{fig:harmony:interval:oct-round}), что для отдельно взятого звука в пределах октавы можно подобрать лишь \emph{два} звука, образующий с исходным звуком совершенный консонанс. Можно ли переупорядочить ноты на рисунке \ref{fig:harmony:interval:oct-round} так, чтобы звуки, образующие с исходным звуком совершенный консонанс, стали его (исходного звука) соседями: один справа, другой слева?

Можно, иначе быть не может! Давайте выпишем последовательность идущих друг за другом совершенных консонансов, отступая всякий раз кварту (5 полутонов). Начнем с ноты C. Кварта вверх от C --- это G. Кварта вверх от G --- D. И так далее. В результате получим:
\begin{center}
    C, G, D, A, E, B, 
\end{center}  %об ладах, интервалах и аккордах
%    \chapter{Тройное сальто назад можешь? Трюки и фишечки}
\label{ch:tricks}

Далее пойдет речь о красивых приёмах игры, которые лучше осваивать не с листа бумаги, а под присмотром и руководством опытного гитариста. Ощутимо могут помочь видеоролики, поэтому не поленитесь и зайдите, например, на Youtube-канал\footnote{От себя: очень рекомендую Youtube-канал <<Гитара с нуля --- уроки игры на гитаре>> \cite{url:guitarFromZero}, как полноценный обучающий курс, построенный по принципу от простого к сложному. Канал <<Нескучный саунд>> \cite{url:funnySound} хорош для тех, кто подкован в музыке стальными подковами. Электрогитаистам стоит обратить внимание на канал <<fredguitarist>> \cite{url:fredguitarist}, предварительно отсеяв замечательные уроки от отвратительных поливаний грязью всех остальных гитаристов} <<Pima Live>> \cite{url:pimalive}.


\section{А чтоб как капелька упала? Флажолет}
\label{ch:tricks:flageolet}

Флажолетом\index{флажолет}\footnote{Флажолет (старофр. flageolet --- маленькая флейта). На английском флажолет называют string harmonic, а чаще просто harmonic} называется приём, позволяющий <<изъять>> из обычного звука основной тон и часть обертонов. В результате получается весьма необычный звук.

Для начала вспомним структуру звука, издаваемого струной, обратившись к рисунку \ref{fig:tricks:flageolet:nodes}. Заметим, что в помеченных на рисунке серыми кружками точках струны колебания отсутствуют --- это узлы колебаний. Первый обертон имеет один узел на струне, второй --- два, и тд.

\begin{figure}[!ht]
    \centering
    \includegraphics{fig/string-nodes} 
    \caption{Структура звука}\label{fig:tricks:flageolet:nodes}
\end{figure}

\begin{Example}[Сыграем первый флажолет]
    Давайте сыграем флажолет на первой, самой тонкой струне. Там он прозвучит лучше всего. Найдем середину струны --- место, где находится узел первого обертона. Как известно, это прямо над 12-м ладовым порожком. Далее нужно легко поставить палец\footnote{Обычно указательный или средний, какой лучше слушается} левой руки на середину струны, не нужно сильно давить, а тем более прижимать струну к 12-му порожку --- нужно легкое касание. Далее щипните правой рукой струну как обычно, но чуть порезче. Палец левой руки должен уйти с узла вверх на мгновение позже щипка, почти одновременно с ним.
    
    Попробуйте несколько раз, вы поймете, когда у вас получится. Поищите нужное движение, при котором звук получается наиболее ярким.
    
    Причина постоянных неудач: палец левой руки стоит не на узле. 
    
    Если всё получилось, попробуйте сыграть флажолет над 11 или 13-м ладами. Не получается? И не должно. Флажолет --- штука капризная.
    
    Что же получилось в результате извлечения звука таким способом? Палец левой руки заглушил основной тон, а также все обертоны, не имеющие узла в середине струны. Громче всех (вместо основного тона) прозвучит при этом 1-й обертон.
    
    Разница между сыгранным нами флажолетом и обычным щипком на 12 ладу изображена на рисунке \ref{fig:tricks:flageolet:first}.
\end{Example}
 
\begin{figure}[!ht]
    \centering
    \includegraphics{fig/string-flageolet} 
    \caption{Флажолет первого порядка}\label{fig:tricks:flageolet:first}
\end{figure} 

Этим же приёмом можно сыграть флажолет в узле, делящем струну на три части (см. второй обертон на рисунке \ref{fig:tricks:flageolet:nodes}). В этом случае один из узлов будет находится примерно на 7-м ладу. Давайте это проверим. По формуле \ref{fig:guitar:construction:length} рассчитаем длину звучащего участка струны от нижнего порожка до 7-го лада:
\[
    L(7)=\frac{L}{(\sqrt[12]{2})^7}\approx L\cdot 0.66742
\]

А необходимые нам две трети струны ($\frac{2}{3}\cdot L$) составляют:
\[
    \frac{2}{3}\cdot L \approx L\cdot 0.66667
\]

Абсолютная погрешность будет равна $\Delta \approx 0,00075 \cdot L$. Так как длина струны $L$ на полноразмерной классической гитаре составляет $66$ см., то погрешность составит меньше половины миллиметра. Поглядите на свой пухленький пальчик и смело пренебрегайте погрешностью --- ставьте палец левой руки прямо над 7-м ладовым порожком.

Играя флажолет на трети струны, вы столкнетесь с еще одним фактором, влияющим на качество звука: флажолет не прозвучит, если правая рука будет щипать струну вблизи второго узла (треть струны от подставки). Поэкспериментируйте. Капризов у флажолета добавилось.

Четверть струны от верхнего порожка (а от нижнего, соответственно $\frac{3}{4}$) находится примерно над 5-м ладовым порожком. Не забывайте о наличии уже трех узлов колебаний, вблизи которых нельзя щипать струну правой рукой.

На этом пожалуй можно остановиться, потому что флажолеты более высоких порядков играть всё сложнее: они звучат всё тише и тише, а вероятность ошибки всё больше и больше.

Мы разобрались с флажолетами на открытых струнах. Музыканты называют их \emph{натуральными}\index{флажолет!натуральный} или \emph{естественными}\index{флажолет!естественный}. Так как на практике играют флажолеты в основном на половине струны, и гораздо реже на трети или четверти, то вариантов не слишком-то много.

Представим, что вы зажали струну на первом ладу, и она стала короче. На каком ладу теперь находится половина струны? Правильно, на 13-м! Сомневаетесь --- поиграйте с формулой \ref{fig:guitar:construction:length}. На каком бы ладу вы не зажали струну, её половина будет находится на 12 ладов выше, треть --- на 7, четверть --- на 5. Количество мест, где можно сыграть флажолет, резко возросло.

Это, конечно, здорово, но если мы зажимаем лад левой рукой, то где взять еще одну руку, чтобы <<придерживать>> узел? На открытой струне это делалось левой рукой. Увы, если у вас нет лишней руки, то справляться придется одной правой. Обычно правая рука делает это так: указательным пальцем касается нужного узла, а большим (безымянным или мизинцем, кому как удобнее) играет щипок, практически одновременно с этим снимая с узла указательный палец (прием проще исполнить, если отодвигать от грифа всю кисть, а не один указательный палец). Знакомьтесь: \emph{искуственный}\index{флажолет!искуственный} флажолет.

Итак, самые громкие натуральные флажолеты звучат с узлами на 12, 7 и 5 ладу\ldots Знакомые интервалы, не правда ли? Пожалуй, самое время ещё раз задуматься о природе консонансов! Ну и, конечно, самому поискать ответ на давно, я надеюсь, мучивший вас вопрос: <<Почему октава делится именно на 12 частей?!>> Успехов в самообразовании!


\section{А нужен молоток? Hammer-on, Pull-off}

Это прозвучит напыщенно, но извлекать звуки из гитары можно <<одной левой>>.

Суть приёма Hammer-on\index{легато!Hammer on} в том, что по уже звучащей струне палец левой наносит точный и резкий <<молоточковый>>\footnote{Hammer --- англ. молоток. Hammer-on --- буквально <<молотком --- на!>>} удар, резко прижимая её к ладу. Главное гриф не пробить, не переусердствуйте\footnote{Конечно, главное тут --- не сила, а скорость и точность}! Струна очень резко прижимается к ладу и нерастраченная струной энергия преобразуется в новый, более высокий звук. Чем резче и точнее удар пальцем, тем меньше энергопотери струны, тем громче и чище получится новый звук\footnote{На электрогитаре можно нанести удар и по покоящейся струне --- эффект будет. Кстати, игра <<молоточковыми>>, пробивающими струну до лада, ударами пальцев только уже \emph{правой} руки (левая, при этом либо как обычно, зажимает лады, либо исполняет приемы Hammer-on и Pull-off) называется \emph{тэппинг}. Tapping --- англ. tapping out --- постукивание, выстукивание}.

Pull-off\index{легато!Pull off} исполняется обычно так: два пальца левой руки заранее ставятся на разные лады на одной и той же струне; правая рука обычным образом извлекает звук; палец левой руки, стоящий на более <<высоком>> ладу (то есть прижимающий конец звучащего отрезка струны) резко сдергивает струну\footnote{Pull off --- буквально переводится с английского как <<срыв>>}, освобождая её. При этом струна, которой <<сдергивание>> добавило энергии, удлиняется до лада, на котором стоит второй палец левой руки, и начинает издавать более низкий звук.

Hammer-on и Pull-off --- приёмы \emph{связного} извлечения звуков. На слух, звуки плавно (а чаще быстро и плавно) как бы <<переходят>> друг в друга\footnote{На самом деле просто отсутвтует резкое увеличение громкости следующего звука, которое бывает при игре щипком правой рукой, ну и струна, перестраиваясь на новую частоту, испытывает длящийся какое-то время <<переходный процесс>>, также создающий ощущение <<плавности>>}. Классическая щкола игры на гитаре обычно использует единый термин для <<свзяного>> извлечения звуков: \emph{легато}\index{легато}. Приём Hammer-on, повышающий звук, классики назовут \emph{восходящим} легато, а понижающий звук Pull-off --- \emph{нисходящим} легато\footnote{Legato --- итал. связанно, плавно}. 

Как говорится: <<К волкам попал --- по волчьи вой>>. В общем, каждый называет эти приемы, в зависимости от того, как требует окружение. Названия <<Hammer-on>> и <<Pull-off>> появились позже <<легато>> и являются более модными в настящее время, особенно в среде эстрадных гитаристов, использующих гитары с электрическим усилением звука.


\section{Как выжать слезу? Вибрато\index{вибрато}}
\label{ch:tricks:vibrato}

Когда звук сыгран правой рукой, а палец левой руки, прижимающий струну примерно посередине между ладовыми порожками, за счет движения запястья и предплечья как бы <<прокатывается>> подушечкой туда-обратно вдоль струны, возникает очень выразительный, рвущий душу на части звук.  

\begin{figure}[!ht]
    \centering
    \includegraphics{fig/vibrato} 
    \caption{Приём вибрато}\label{fig:tricks:vibrato}
\end{figure}

Едва уловимые периодические биения высоты основного тона привлекают к себе внимание. Это происходит из-за незначительных изменений натяжения струны, когда по ней прокатывается подушечка пальца. На рисунке \ref{fig:tricks:vibrato} видно, что участок струны между ладами в разных положениях имеет разную длину, а значит из-за этого изменилось и натяжение струны в целом.


\section{Подтяжки? Подтяжка}

О подтяжке мы успели поговорить аж в самом начале. Палец левой ркуи, продолая прижимать звучащую струну к ладу, сдвигает её поперек грифа, как тетиву лука. Натяжение струны при этом сильно меняется, чем достигается куда более сильное повышение высоты звука, чем при вибрато: на полутон а то и на все два. Вариаций подтяжек --- множество, например, можно сначала подтянуть струну, щипнуть, а потом, прижимая к ладу, снять натяжение. 

На английском языке подтяжка называется bend\footnote{Bend --- англ. существительное --- изгиб; глагол --- гнуть, искривлять}. И Русский язык впитал в себя ещё одно заимствование, потому что всё чаще произносят и пишут <<бенд>> вместо <<подтяжка струны>>.
   %нестандартные способы звукоизвлечения 
    \chapter*{В заключение}
\addcontentsline{toc}{chapter}{В заключение}

Данный текст подготовлен в издательской системе {\LaTeXe} (автор использовал MiC\TeX.). Эта издательская система является стандартом де-факто в научных и технических кругах.

Заинтересовавшимся версткой в {\LaTeX} можно рекомендовать следующие книги: \cite{bib:cotelnikov,bib:baldin}.

Про предшественника {\LaTeX} --- программу {\TeX} следует читать бестселлер от автора\footnote{{\TeX} на самом деле является ядром \LaTeX} \cite{bib:knuth:AllAbout}.

Ноты в прекрасном оформлении можно получить с помощью программы LilyPond, которая использует {\TeX} в качестве <<рисующего>> нотоносец ядра, а ноты задаются в особом текстовом формате.
    %заключение
    
    \bibliographystyle{plain}
    \bibliography{./bibliobase}
\end{document} %конец документа
