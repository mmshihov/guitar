\chapter*{В заключение}
\addcontentsline{toc}{chapter}{В заключение}


Данный текст подготовлен в издательской системе {\LaTeXe} (автор использовал MiC\TeX.). Эта издательская система является стандартом де-факто в научных и технических кругах.

Заинтересовавшимся версткой в {\LaTeX} можно рекомендовать следующие книги: \cite{bib:cotelnikov,bib:baldin}.

Про предшественника {\LaTeX} --- программу {\TeX} следует читать бестселлер от автора\footnote{{\TeX} на самом деле является ядром \LaTeX} \cite{bib:knuth:AllAbout}.

Ноты в прекрасном оформлении можно получить с помощью программы LilyPond, которая использует {\TeX} в качестве <<рисующего>> нотоносец ядра, а ноты задаются в особом текстовом формате.

Странно бы не помянуть хоть что-нибудь из самоучителей. Тут я скажу только за себя, так как лучшую книгу я, возможно, еще не нашел и, может быть, не найду. Да и лучшей она будет только для меня и еще нескольких человек на этой земле. Итак, лично мне (возможно, что только мне\ldots) нравится как раскрывает тему Алексей Кофанов. У него есть замечательная книжка о гитаре \cite{bib:kofanov:AboutGuitar} и даже книжка о сочинении музыки, написанная просто и без зауми \cite{bib:kofanov:MusicGeneration}.
