\chapter*{Введение}
\addcontentsline{toc}{chapter}{Введение}

Иногда человек, сталкивавшийся с музыкой только как слушатель, вдруг приходит к мысли: <<А почему бы мне не попробовать научиться \emph{играть} на инструменте X?>>. И тут он открывает для себя параллельную вселенную.

Через некоторое время, если этот человек не прибегает к помощи знающих людей, и ему не везет с учебниками (самоучителями, обучающими видео в Интернет и т.д.), он понимает, что в этой вселенной ему нет места.

Особенно остро это понимают люди с инженерным (математическим, техническим и т.д.) образованием. Им хочется найти в музыке целостную систему, знания, обобщения, красивые законы, а не обилие разрозненных фактов.

Погружаясь в музыкальню тему, человек с аналитическим мышлением быстро офигевает от необходимости зубрить и принимать на веру практически всё. Не находя в учебниках ответа на фундаментальные вопросы <<зачем и почему?>>, он закономерно посылает музыку подальше.

Прекрасные музыканты, которые в свое время прошли пытку музыкальной школой (может быть училищем или даже консерваторией), имели достаточно времени, чтобы подсознательно объяснить себе некоторые вещи и смириться с ними. Не каждый человек задумывается над тем, с чем просто сжился. Привычные вещи люди чаще перестают замечать, а не стараются в них разобраться.

И если новичку с аналитическим типом мышления вдруг не везет с учителем, который в силу изложенных выше причин гораздо больше \emph{чует}, чем \emph{знает}, то новичок очень быстро убеждается в инопланетности, а то и вовсе в избранности музыкантов.

Нет никакой избранности, большинство музыкантов --- Земляне, а в музыкальной вселенной можно не только выжить, но и жить счастливо, ибо там есть законы и гармония! 

В чем и попробуем убедиться:

\begin{music}
    \startextract
    \notes\qu{cdefghi}\enotes
    \endextract
\end{music}
