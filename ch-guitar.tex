\chapter{А это что за штука? Устройство гитары}
\label{ch:guitar}

Гитара\index{гитара} --- простой и изящный музыкальный инструмент. Благодаря этим качествам ей и удалось завоевать сердца многих исполнителей.

Но не путайте простоту с безграмотностью! Гитара --- инструмент для извлечения звуков \emph{равномерно темперированного строя}. Вспомним (см. раздел \ref{ch:music:tone}), что строй --- это правила отбора частоты колебаний основного тона для особых звуков, которые называются \emph{музыкальными}. Именно такие звуки извлекают из \emph{музыкального} инструмента.

Так увидим же математику строя в конструкции гитары, а затем разберемся с тем, как правильно её на\emph{строить}.


\section{Может разберем? Конструкция}
\label{ch:guitar:construction}

Взглянем на рисунок \ref{fig:guitar:construction}, чтобы узнать, как называются отдельные части гитары.

\begin{figure}[!ht]
    \centering
    \includegraphics{fig/guitar-construction} 
    \caption{Устройство гитары\index{гитара!устройство}}\label{fig:guitar:construction}
\end{figure} 

Отдельная \emph{струна} одним концом крепится к неподвижной \emph{подставке}, а вторым концом наматывается на барабанчик \emph{колка}. Червячный механизм колка позволяет очень тонко регулировать степень натяжения струны. Струна натягивается вдоль \emph{грифа}, поперек которого врезаны на определенном расстоянии друг от друга металлические выступы --- \emph{ладовые порожки}. Струна натянута так, что её звучащая (колеблющаяся) часть опирается своими концами на верхний и нижний порожки, а остальных ладовых порожков не касается, хотя и проходит достаточно близко к ним\footnote{
    Приведу все же немного справочных данных о расстояниях\index{струна!высота над грифом} от струн до ладов. С одной стороны, чем оно меньше, тем легче и быстрее укорачивается струна, а с другой стороны, слишком <<зaниженная>> струна может во время игры начать звенеть, касаясь ладовых порожков. Для классической аккустической гитары (с нейлоновыми струнами) средние показатели такие: 
    \begin{itemize}
        \item расстояние от 1-й струны до 1-го ладового порожка составляет 0.61 мм, а до 12-го --- 3.18 мм;
        \item от 6-й струны до 1-го порожка --- 0.76 мм, а до 12-го --- 3.96 мм;
    \end{itemize}
    
    Для эстрадной аккустической гитары (со стальными струнами и анкерным стержнем внутри грифа) расстояния следующие:
    \begin{itemize}
        \item от 1-й струны до 1-го порожка --- 0.33 мм, а до 12-го --- 1.78 мм;
        \item от 6-й струны до 1-го порожка --- 0.58 мм, а до 12-го --- 2.29 мм;
    \end{itemize}
    
    Пусть вас не пугают приведенные с точностью до сотых долей миллиметра числа --- они являются расчетными. На практике регулировка осуществляется подтачиванием верхнего порожка и порожка на подставке. Замеры расстояния осуществляются (в порядке увеличения популярности): штангенциркулем, стальной слесарной линейкой, просто <<на глазок>> --- но никак не микрометром! У людей профессионально занимающихся доводкой гитар, обычно заготовлены пластинки нужной толщины, и высота контролируется простым просовыванием пластинки между струной и порожком.
    
    Единого стандарта на высоту струн нет. Профессионалы, естественно, занижают высоту струн до предела, чего новичкам делать не следует. С завода бюджетные гитары обычно поступают в магазины с <<завышенной>> высотой струн и требуют доводки. В хорошем музыкальном магазине либо <<доведут>> гитару при покупке, либо посоветуют гитарного мастера.
}

Известно, что частота колебаний струны зависит от степени её натяжения и её длины. Перед тем как играть, гитару \emph{настраивают}, то есть с помощью колков каждую струну натягивают так, чтобы она издавала строго определенный \emph{музыкальный} звук. Чем сильнее натянута струна, тем выше звук (т.е. тем больше частота колебаний струны). А уже играя, гитарист прижимает пальцем левой рукой струну к ладовому порожку, чем практически мгновенно укорачивает её звучащую часть. Что в свою очередь приводит к повышению звука, так как частота колебаний струны обратно пропорциональна её длине\footnote{Надо честно заметить, что частота колебаний струны зависит также и от силы её натяжения, которая меняется, когда струну <<зажимают>> на ладу. Но это влияние столь незначительно, что им можно пренебречь}.

Теперь давайте вспомним о самых важных музыкальных единицах измерения расстояния между двумя \emph{музыкальными} звуками.
\begin{itemize}
    \item \emph{Октава} --- звук $x$ \emph{выше} звука $y$ на \emph{октаву}, если частота колебаний основного тона звука $x$ больше частоты основного тона звука $y$ вдвое. Чтобы повысить звук открытой струны на октаву, нужно укоротить её вдвое, сохранив натяжение.
    
    \item \emph{Полутон} --- двенадцатая часть октавы. Теоретически неделимое расстояние между двумя музыкальными звуками. Звук $x$ \emph{выше} звука $y$ на \emph{полутон}, если частота основного тона звука $x$ больше частоты основного тона $y$ в $\sqrt[12]{2}$. Откуда квадратные корни? Проверьте: двенадцать раз повысив высоту звука $y$ на полутон, получим звук $x$ с частотой в $(\sqrt[12]{2})^{12}$ больше частоты исходного $y$. Так как $(\sqrt[12]{2})^{12} = 2$, то мы получили расстояние в октаву, что и требовалось! 
\end{itemize}


\begin{Definition}[Суть гитарной простоты]
    Номер лада на грифе гитары соответствует количеству \emph{полутонов}, на которое \emph{повышается} звук открытой струны, если её зажать на этом ладу.
\end{Definition}

Сдвинулись по грифу на лад --- повысили (или понизили) звук на \emph{полутон}. Сдвинулись на 12 ладов --- повысили/понизили на \emph{октаву}. И вообще: над 12 ладовым порожком находится середина струны!

В соотвествии с правилами равномерно темперированного строя (см. раздел \eqref{ch:music:tone}) частота основного тона струны зажатой на $n$-м ладу $f(n)$ будет определяться как 
\[
    f(n) = f_\text{откр}\cdot(\sqrt[12]{2})^n,
\] 
где $f_\text{откр}$ --- частота основного тона открытой струны. А так как частота колебаний струны обратно-пропорциональна её длине, то длина зажатой на $n$-м ладу струны $L(n)$, дающей звук нужной частоты, будет определяться так:

\begin{equation}
    \label{eq:guitar:construction:length}
    L(n) = \frac{L}{(\sqrt[12]{2})^n},
\end{equation}
где $n$ --- номер лада ($0$-й лад соответствует открытой струне, см. рисунок \ref{fig:guitar:construction}), а $L$ --- общая длина <<открытой>> струны: от подставки до верхнего порожка.

\begin{figure}[!ht]
    \centering
    \includegraphics{fig/string-length.png}
    \caption{Расстояние между ладами}\label{fig:guitar:construction:length}
\end{figure} 

Вот и готова конструкция гитары! На рисунке \ref{fig:guitar:construction:length} соотнесён график функции \ref{eq:guitar:construction:length} с расположением ладов на гитарном грифе. Теперь ясно, например, почему по мере приближения к розетке расстояние между ладами становится всё меньше.

Кстати, некоторые ушастые выпендрёжники говорят, что различают своим сверхмузыкальным слухом больше 12 нот в октаве! Давайте делить полутон! Есть спрос --- есть предложение: на некоторых гитарах можно заметить дополнительные ладовые порожки между <<каноническими>>, которые позволяют <<всунуть>> дополнительную ноту.

Играя, гитарист лишь меняет длину звучащего участка струны, зажимая струны на ладах. Однако редкие психи/мастера крутят колок во время исполнения, добиваясь сомнительных/удивительных музыкальных эффектов.

Так, ядом поплевались, хорошо, хватит. Давайте закончим тему конструкции гитары чем-нибудь полезным. Например, выбирая гитару, стоит проверить, что она изготовлена по всем правилам\footnote{Сейчас нарваться в музыкальном магазине на нестроящую гитару --- случай редкий. Но всё же нельзя полностью исключать брак серийного производства. 3Б подход: Береженого --- Бог Бережет\ldots}. Музыканты скажут: проверить строй. Середина струны должна находиться точно над 12-м ладом, четверть (приблизительно) над 5-м ладом, треть (приблизительно) --- над 7-м. Проще всего это проверить без линейки, сыграв флажолеты, см. раздел \ref{ch:tricks:flageolet}. На <<нестр\'{о}ящей>> гитаре флажолеты не прозвучат.


\section{Правильно устроена? --- не значит, что настроена!}
\label{ch:guitar:tuning}

Начнём с того, что никогда, никогда, никогда\ldots \emph{НИКОГДА} не играйте на ненастроенном инструменте!

Продолжим тем, что настроить\index{гитара!настройка} гитару можно по-разному\index{строй!гитары}. Все зависит от того, как натянуть струны. И, строго говоря, этих вариантов много. Очень много. Но есть один, который любят все! И называется он: <<классический гитарный строй>>\index{строй!гитары!испанский}, он же <<испанский>>\index{строй!гитары!классический}, он же <<МИ>> (по-русски), он же <<E>>\footnote{E --- буква латинского алфавита. Используется для обозначения ноты МИ и произносится на русском как <<и>>. Учтите, что <<и>>-кать нужно только для иностранцев --- не злите Русских гитаристов}. 

Мы научимся настраивать гитару классическим строем. Именно он используется по умолчанию. Да он практически всегда используется! Увидели аппликатуры и табулатуры для шестиструнной гитары? --- в 99.9\% случаев они для классического строя. Для новичка уж точно выбора нет: сначала научись как все, а потом экспериментируй с настройкой гитары.

Итак, классический строй\footnote{Если не знаете, что такое ноты --- обратитьесь к разделу \ref{ch:notes}}.
\begin{itemize}
    \item МИ(E): первая струна настраивается колком на звук МИ \emph{первой} октавы.
    \item СИ(B): вторая струна настраивается на СИ \emph{малой} октавы.
    \item СОЛЬ(G): треться струна --- СОЛЬ \emph{малой} октавы.
    \item РЕ(D): четвертая --- МИ \emph{малой} октавы.
    \item ЛЯ(A): пятая --- ЛЯ \emph{большой} октавы.
    \item МИ(E): шестая --- МИ \emph{большой} октавы.
\end{itemize}

Неплохо бы запомнить последовательность: МИ, СИ, СОЛЬ, РЕ, ЛЯ, МИ, но ещё лучше понять относительную настройку струн:
\begin{itemize}
    \item МИ: первая струна настраивается на звук МИ первой октавы. Это база, никуда не деться\ldots
    \item СИ: вторая струна настраивается на 5 полутонов ниже первой. Проверьте расстояние от СИ до МИ: СИ,ДО,РЕ,МИ --- 1+2+2 = 5.
    \item СОЛЬ: третья струна --- на 4 полутона ниже второй (единственное исключение!).
    \item РЕ: четвертая --- на 5 полутонов ниже третьей.
    \item ЛЯ: пятая --- на 5 ниже четвертой.
    \item МИ: шестая --- на 5 ниже пятой.
\end{itemize}

Довольно компактное правило получилось:
\begin{Rule}[Классический строй]
    Каждая следующая (по номеру) струна настраивается на 5 полутонов выше предыдущей, за исключением третьей струны, которая на 4 полутона ниже второй. То есть музыкальные интервалы между струнами такие:
    \[
        \text{1}\xrightarrow{-5}
        \text{2}\xrightarrow{-4}
        \text{3}\xrightarrow{-5}
        \text{4}\xrightarrow{-5}
        \text{5}\xrightarrow{-5}
        \text{6}
    \]
\end{Rule}

С теоретическими основами разобрались, перейдем к практике. Начнем с наиболее комфортных способов и закончим хардкором.

Для начала о технике безопасности: струна может порваться, поэтому нужно следить, чтобы важные органы не находились на <<линии огня>>. Учтите, премию Дарвина за выбитый глаз не дают!

Также стоит сразу отметить, что независимо от способа, эту самую настройку следует делать в несколько подходов. Дело в том, что гитара в ходе настройки немного деформируется и натяжение уже настроенныех струн чуть слабнет. Тем более, если вы поставили новые струны: они вообще некоторое время\footnote{Например, нейлоновые струны на классической гитаре могут <<садиться>> неделю. Стальные на эстрадной аккустике обычно <<садятся>> за день} будут <<растягиваться>> и <<усаживаться>>. Новую струну обычно сразу после настройки довольно аграссивно натягивают пальцем --- помогают быстрее <<сесть>>, после чего она расстраивается, и её тут же можно настроить заново.

В настоящее время удобнее всего воспользоваться электронным тюнером\index{тюнер}. Это может быть прицепляющийся на головку грифа тюнер-прищепка или смартфон с установленным на него тюнер-приложением\footnote{Так как на рынке много бесплатных тюнер-приложений для смартфона, а смартфон сейчас есть почти у каждого, то это простое и разумное решение. Об удобстве специализированных прищепок можно долго спорить, а вот на смартфон можно еще и метроном с редактором музыки закачать}. Смысл в том, что электронный тюнер покажет вам ноту, которой звучит струна, а также подскажет: натянуть струну или ослабить, чтобы нота звучала правильно. Особенностей тут немного: щиплете струну, смотрите на тюнер и подкручиваете колок. В помещении должна быть тишина, иначе тюнер может начать ошибаться\footnote{Можете попеть в розетку гитары и посмотреть, что покажет тюнер. Попробуйте пропеть: <<ДО-о-о-о>>, чтобы было действительно ДО!}. Некоторые тюнеры-прищепки показывают ноту, но не показывают ни октаву, ни частоту, так что на первых порах новички трясущимися руками натягивают струну очень слабо --- ниже на октаву. И если у вас возникло ощущение, что вы играете на соплях, а не на струнах --- стоит докрутить колок, последовательно пройдя по нотам до нужной из следующей октавы.

Теперь способ пожёстче: тюнера у вас нет, смартфона тоже. Зато есть уши и камертон\index{камертон}! Или какой-нибудь настроенный музыкальный инструмент. Камертон --- устройство, которое может долгое время издавать звук строго определенной высоты. Ну, например, чаще всего попадается в руки <<эталонный>> камертон, издающий ту самую A4=440 Гц, то есть ЛЯ первой октавы. Обычно камертон --- это металлическая вилка или свисток. Камертон может быть и электронным, при этом вы можете найти его в самых неожиданных местах: в сети Интернет (в виде онлайн-приложения), в тюнер-приложении для смартфона, в электронном метрономе, в синтезаторе, электронном пианино и даже в электронных часах. Камертон-вилка имеет особенности: чтобы его завести, нужно им стукнуть по чему-нибудь твердому\footnote{Лучше стукнуть камертоном по голове, если вы подумали, что им можно стукнуть по гитаре!} и приложить ножкой к верхней деке гитары --- издаваемый тон станет громче.

Грубый способ настройки\index{гитара!настройка} такой:
\begin{itemize}
    \item МИ: первая струна настраивается на звук МИ первой октавы. Если у вас камертон на ноту МИ --- добейтесь звучания открытой струны в \emph{унисон} с камертоном. Если у вас камертон A4 (то есть на ноту ЛЯ), то зажмите первую струну на 5-м ладу (ЛЯ) и добейтесь звучания в унисон с камертоном. Принцип: послушали --- подтянули --- послушали. Настроили? --- камертон можно убрать.
    \item СИ: вторая струна, зажатая на 5-м ладу настраивается в \emph{унисон} с открытой первой струной. Мы уже знаем, что открытая вторая струна звучит на 5 полутонов ниже открытой первой. Значит на 5-м ладу она звучит с открытой первой в унисон.
    \item СОЛЬ: третья струна, зажатая на 4 четвертом ладу настраивается в унисон с открытой второй.
    \item РЕ: четвертая на 5-м ладу --- в унисон с третьей.
    \item ЛЯ: пятая на 5-м ладу --- с четвертой.
    \item МИ: шестая на 5-м ладу --- с пятой.
\end{itemize}

Если со слухом пока неважно (слух, кстати, дело наживное!) и \emph{выслушать} унисон не получается, то можно использовать явление \emph{резонанса}\index{резонанс}. Дело в том, что когда две струны настроены на одну частоту колебаний (т.е. <<в резонанс>>, а на слух --- <<в унисон>>), колебания одной струны, передаваясь через остальные части гитары, будут легко возбуждать колебания в другой. При этом другие струны, не настроенные в резонанс, возбуждаться не будут. Допустим, вы настраиваете вторую струну, чтобы на 5-м ладу она звучала в унисон первой. Если вы сыграли вторую струну на 5-м ладу, затем легко коснулись плоскостью ногтя большого пальца первой струны (покоившейся до сих пор) и услышали при этом характерный <<чик>>, то это значит, что первая струна <<вошла в резонанс>> (завелась) и дело сделано\footnote{На легких нейлоновых струнах резонанс поймать довольно сложно. Но можно. В этом случае лучше сначала потренироваться на тяжелых басовых струнах. На них явление резонанса заметно даже визуально: невооруженным глазом видно, как вдруг начинает вибрировать до сих пор покоившаяся струна}!

Теперь хардкор: нет ничего, кроме немузыкального слуха\index{слух немузыкальный}. Знаете, это все чушь, чтобы первая струна звучала нотой МИ первой окравы! Если душа просит музыки, а других способов точно настроить эталонную частоту нет --- расслабьтесь и получайте удовольствие. Натяните первую струну так, что по вашему субъективному мнению, она звучит <<как надо>> и настройте остальные струны относительно первой, как уже умеете.

Расположение нот на грифе придется запомнить. Поняв на каком расстоянии в полутонах находятся ноты друг относительно друга и запомнив положение некоторых нот, вы начнёте относительно быстро ориентироваться на грифе. Например, вы знаете, что настроили открытую 5-ю струну на ноту ЛЯ, тогда не втором ладу будет нота СИ (от ЛЯ до СИ --- два полутона), а на третьем --- ДО (ЛЯ--СИ--ДО --- три полутона), и т.д.

Если же вы проявите упорство и будете уделять гитаре достаточно времени, то ваш организм поймет, что от него просто так не отстанут и быстренько сформирует необходимые рефлексы. О поиске нот думать уже не придется: руки сами всё найдут. 

Однако, но на первое время не помешает несколько шпаргалок. Привожу их ниже. Используйте, пока возникают проблемы. И не забывайте пожалуйста, что эти шпаргалки лишь для классического (испанского) гитарного строя, с которого, несомненно, стоит начать новичку!

На рисунке \ref{fig:guitar:notes-on-griph-ru} приведен рисунок участка грифа (12 ладов) с подписанными обозначениями нот на русском языке. Нижний индекс у названия ноты определяет октаву: <<Б>> --- большая октава, <<М>> --- малая, <<1>> --- первая, <<2>> --- вторая.

\begin{figure}[!ht]
    \centering
    \includegraphics[width=\textwidth]{fig/notes-on-griph-ru} 
    \caption{Ноты на грифе RU\index{нота!расположение на грифе}}\label{fig:guitar:notes-on-griph-ru}
\end{figure} 

Гитаристы очень часто сталкиваются с латинскими обозначениями нот, поэтому на рисунке \ref{fig:guitar:notes-on-griph-lat} приведено все то же самое, что и на рисунке \ref{fig:guitar:notes-on-griph-ru}, только в латинских обозначениях.

\begin{figure}[!ht]
    \centering
    \includegraphics[width=\textwidth]{fig/notes-on-griph-lat} 
    \caption{Ноты на грифе EN\index{нота!расположение на грифе}}\label{fig:guitar:notes-on-griph-lat}
\end{figure} 

Чтобы было проще работать с \emph{нотной записью} гитарной музыки, приведу также несколько рисунков (см. \ref{fig:guitar:lad-by-notes}, \ref{fig:guitar:lad-by-griph}, \ref{fig:guitar:lad-by-diagonal}), соотносящих нотоносец и гриф. Используйте тот, что больше понравится или придумайте что-нибудь свое.

На рисунке \ref{fig:guitar:lad-by-notes} горизонтально изображен нотоносец, а вертикально, поперек линий нотоносца --- струны. Допустим, нам нужна нота СОЛЬ малой октавы. Она пишется на второй линии нотоносца, поэтому смотрим на каких ладах эта линия пересекает линии струн: десятый лад пятой струны, пятый лад четвертой и нулевой лад третьей струны. Лады промежуточных нот не подписаны в целях экономии места, но несложно вычислить, что, например, СОЛЬ-диез малой октавы будет находится на одиннадцатом ладу пятой струны, шестом ладу четвертой и первом ладу третьей струны.
 
\begin{figure}[!ht]
    \centering
    \includegraphics[width=\textwidth]{fig/lad-by-notes} 
    \caption{Ноты на грифе (гриф поперек нотоносца)\index{нота!расположение на грифе}}\label{fig:guitar:lad-by-notes}
\end{figure} 

На рисунке \ref{fig:guitar:lad-by-griph} горизонтально изображен не только нотоносец, но и струны. Например, спустившись по пунктирной линии от ноты СОЛЬ малой октавы, сделаем те же выводы, что и ранее о положении этой ноты на грифе гитары. 

\begin{figure}[!ht]
    \centering
    \includegraphics[width=\textwidth]{fig/lad-by-griph} 
    \caption{Ноты на грифе (гриф вдоль нотоносца)\index{нота!расположение на грифе}}\label{fig:guitar:lad-by-griph}
\end{figure} 

На рисунке \ref{fig:guitar:lad-by-diagonal} струны пересекают стан нотоносца под наклоном. Над нотой подписан лад, на котором нужно зажать соответствующую струну, а под нотой --- её название.

\begin{figure}[!ht]
    \centering
    \includegraphics[width=\textwidth]{fig/lad-by-diagonal} 
    \caption{Ноты на грифе (струны пересекают нотоносец под наклоном)\index{нота!расположение на грифе}}\label{fig:guitar:lad-by-diagonal}
\end{figure} 


Стоит отметить, что на самом деле гриф 12-м ладом не заканчивается, и на всех приведенных справочных рисунках счет ладов можно было бы продолжить. Но, как уже было сказано, в процессе обучения ценность этих рисунков упадет до нуля, а на начальном этапе 12 ладов --- более, чем достаточно.

Надеюсь, вы поняли, чем гитара отличается от рояля? Верно: на гитаре \emph{одну и ту же} ноту можно найти в нескольких местах!
