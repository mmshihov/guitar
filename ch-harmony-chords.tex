\section{Чем больше звуков, тем лучше? Аккорды}
\label{ch:harmony:chords}

Из раздела \ref{ch:harmony:interval} вы узнали, что \emph{интервал} для физиков --- это \emph{мера расстояния} между двумя звуками, а для лириков --- это \emph{два звука}, сыгранных одновременно или друг за другом.

Аккорд же --- понятие целиком лирическое.

\begin{Definition}[Аккорд]
    \emph{Аккорд} --- это три или более музыкальных звука, извлеченных (звучащих) одновременно.
\end{Definition}

Видели ведь, наверное, как играют <<боем>>? Правая рука ходит вверх-вниз, и ритмично бьет по всем струнам одним или несколькими пальцами. Говорить о том, что звуки извлекаются одновременно не приходится --- струны задеваются одна за другой, но так как это происходит достаточно быстро, то \emph{звучат} они некоторое время все вместе и мы слышим \emph{аккорд}.

Какие же звуки должны входить в аккорд, чтобы он звучал гармонично? Точно не любые. Есть ли правила построения аккордов?

Да, система аккордов сложилась и давно устоялась. Она определяет интервальную структуру аккордов и правила их именования, а в её основе лежат мажорный и минорный лады. Во всем множестве аккордов ярко выделяются две большие группы: мажорные (веселые) и минорные (грустные).

\begin{figure}[!ht]
    \centering
    \includegraphics[width=\textwidth]{fig/chords/structure} 
    \caption{Интервальная структура некоторых аккордов}\label{fig:harmony:chords:structure}
\end{figure} 

Например, давайте рассмотрим интервальню структуру простого мажорного аккорда. Взгляните на рисунок \ref{fig:harmony:chords:structure}. В мажорный аккорд входят три музыкальных звука, на которые попадают соответственно первая, третья и пятая ступени мажорного лада. Первый, самый низкий, звук аккорда называется <<основным тоном>>\footnote{В западной терминологии основной тон аккорда называют root. Корень, источник. Все чаще и Русские люди называют основной тон <<корнем>> аккорда. А некоторые уважаемые люди (\cite{url:pimalive}) даже \emph{тоникой} аккорда!} аккорда. Второй звук будет отстоять от певого на 4 полутона, а третий от второго --- на 3 полутона. Интервальная структура мажорного аккорда:
\[
    \texttt{0-4-3}.
\]

Основной тон аккорда --- нота, от которой строится аккорд, определяет его название. Допустим, мы откладываем шаблон мажорого аккорда от ноты ДО (С). Получившееся трезвучие будет называться <<ДО-мажор>>, и в его состав войдут ноты: 
\[
    \text{ДО}\xrightarrow{4}
    \text{МИ}\xrightarrow{3}
    \text{СОЛЬ}
\]

Домисолька! Это не только название известного деткого музыкального театра города Москвы, но и веселый (мажорный) аккордик от ноты ДО! 

Аккорды принято обзначать в более короткой нотации. Так например, чтобы обозначить мажорный аккорд, просто используют латинское обозначение ноты основного тона. Домисолька будет обозначена кратко: <<C>>. А <<ЛЯ-мажор>>, в который входят ноты (см. рисунок \ref{fig:harmony:chords:structure}) ЛЯ(A), ДО-диез (C\#), МИ(E) будет обозначен как <<A>>.

Приведем краткий справочник обозначений и интервальных структур наиболее популярных\footnote{Описание полной системы аккордов выходит за скромные рамки этой книги. Нам важно общий принцип понять, а справочной информации об аккордах в Интернете --- завались} типов аккордов с жалкими попытками найти логику в обозначениях (не забывайте поглядывать на рисунок \ref{fig:harmony:chords:structure}). Вместо $X$ смело подставляйте латинское обозначение любой из 12-ти нот октавы:

\begin{itemize}
    \item $X$ --- мажорный аккорд. Состоит из трех нот, попадающих на 1, 3 и 5-ю ступени мажорного лада. Основа основ, нужно знать. Структура: \texttt{0-4-3}
    \item $X_m$ --- минорный аккорд. Те же три ступени, но минорного лада. Чтобы испольнить большинство <<гитарных>> песен, достаточно научиться играть мажорный и минорный аккорды. \texttt{0-3-4}

    \item $X^{\sharp 5}$ --- увеличенный мажорный аккорд. Смотрим на обозначение: это мажорный аккорд, у которого повышена на полутон пятая ступень мажорного лада (знак диез перед цифрой $\sharp 5$). \texttt{0-4-4}.
    \item ${X_m}^{\flat 5}$ --- уменьшенный минорный аккорд. Минорный аккорд с пониженной на полтона 5-й ступенью минорного лада. \texttt{0-3-3}.

    \item $X^6$ --- мажорный секстаккорд. Взяли мажорный аккорд и добавили нотку: 6-ю ступень мажорного лада. \texttt{0-4-3-2}. А вот называют <<секст>>-аккордом потому, что добавленный звук с основным тоном аккорда образует интервал <<секста>> (см. раздел \ref{ch:harmony:interval}).
    \item ${X_m}^6$ --- минорный секстаккорд. То же самое с минором. \texttt{0-3-4-2}.

    \item $X^7$ --- малый мажорный септаккорд (доминантсептаккорд). Взяли мажорный аккорд и добавили звук на расстоянии <<малой септимы>> от основного тона аккорда (напомню, малая септима --- пониженная на полтона 7-я стумень мажорного лада). \texttt{0-4-3-3}
    \item ${X_m}^7$ --- малый минорный септаккорд. No comments. \texttt{0-3-4-3}

    \item $X_{maj^7}$ --- большой мажорный септаккорд. Добавили к мажорному трезвучию большую септиму. \texttt{0-4-3-4}.
    \item $X_{m(maj^7)}$ --- большой минорный септаккорд. \texttt{0-3-4-4}.
\end{itemize}

Стоит отметить, что в оформлении обозначений аккордов особого единства нет. Например, <<ФА-малый мажорный септаккорд>> могут обозначить: $F_7$, $F^7$, а то и просто $F7$. Но разобраться можно. Помните только, что числа в обозначениях --- это ступени соответствующего лада.

Теперь подойдем поближе к гитаре. Кроме вопроса: <<А как зажать нужные струны?>>, возникает много вопросов. В простейших аккордах (например, в мажорном или минорном) три звука, а на гитаре 6 струн. Как же играют песни аккордами на гитаре <<боем>>? Бьют вроде по всем струнам, а звучит только три звука? Непонятно.

Дело в том, что гитара не только \emph{устроена} с умом, но ещё и с умом \emph{настроена}! Настройте гитару стандартным строем\footnote{Если возникают сложности с настройкой, обратитесь к разделу \ref{ch:guitar:tuning}}: МИ, СИ, СОЛЬ, РЕ, ЛЯ, МИ и давайте разбираться как <<ставить>> аккорды в первой позициии.

Что еще за позиции такие? Эх, оставьте на время эротические фантазии, тут все прозаично: позиция --- это номер лада на грифе, на который ставится (или может быть поставлен) указательный палец левой руки гитариста. Если вы, дожив до прочтения этих строк, сохранили большую часть пальцев левой руки, то в первой позиции вы сможете зажимать струны по крайней мере на четырех ладах, с 1-го по 4-й. 

Давайте самостоятельно построим аппликатуры нескольких аккордов. 

Построим аппликатуру аккорда $G$ --- <<СОЛЬ-мажор>>. Накладываем шаблон аккорда на ноту СОЛЬ (G), и определяем входящие в его состав ноты: СОЛЬ --- основной тон, СИ (В) и РЕ (D). Основной тон должен быть самым <<басистым>> поэтому важно найти ноту основного тона на басовых струнах (лучше на 6-й, но в крайнем случае можно и на 4-й). Нам везет: СОЛЬ находится на 3-м ладу 6-й струны --- в пределах первой позиции. На пятой струне в пределах первой позиции выбор невелик: только нота СИ. На 4-й струне мы вообще экономим палец --- её зажимать не надо, на открытой 4-й струне звучит нужная нам нота РЕ. Еще одна экономия: на открытой 3-й струне также звучит нота СОЛЬ (пусть окравой выше --- это ничего, она сольется с обертонами СОЛЬ на 6-й струне). Вторую струну можно оставить открытой (СИ), а можно зажать на 3-м ладу (РЕ). И наконец первую струну придется зажимать на 3-м ладу, чтобы получить еще одну СОЛЬ.

С местами прижатия определились. Рисуем аппликатурный бокс, отмечаем точки прижатия, подбираем удобные варианты распальцовки.



% TODO: струн шесть - звука три
% пример аппликатурного бокса
%   построение аккорда в первой позиции
%   кошерные и некошерные аккорды не задеваем шестую!

% аппликатуры простейших аккордов, которые можно унести с баррэ в любое место грифа.


