\section{Чем больше звуков, тем лучше? Аккорды}
\label{ch:harmony:chords}

Из раздела \ref{ch:harmony:interval} вы узнали, что \emph{интервал} для физиков --- это \emph{мера расстояния} между двумя звуками, а для лириков --- это \emph{два звука}, сыгранных одновременно или друг за другом.

Аккорд же --- понятие целиком лирическое.

\begin{Definition}[Аккорд]
    \emph{Аккорд} --- это три или более музыкальных звука, извлеченных (звучащих) одновременно.
\end{Definition}

Видели ведь, наверное, как играют <<боем>>? Правая рука ходит вверх-вниз, и ритмично бьет по всем струнам одним или несколькими пальцами. Говорить о том, что звуки извлекаются одновременно не приходится --- струны задеваются одна за другой, но так как это происходит достаточно быстро, то \emph{звучат} они некоторое время все вместе и мы слышим \emph{аккорд}.

Какие же звуки должны входить в аккорд, чтобы он звучал гармонично? Точно не любые. Есть ли правила построения аккордов?

Да, система аккордов сложилась и давно устоялась. В основе этой системы лежат мажорный и минорный лады, и она определяет интервальную структуру аккордов и правила их именования. Кстати, поэтому аккорды делятся на две большие группы: мажорные (веселые) и минорные (грустные).

TODO

% интервальная структура аккорда и названия
% как декодировать этот шифр --- обозначения аккордов?

% TODO: струн шесть - звука три
% пример аппликатурного бокса

% ходовые аккорды в первой позиции

% аппликатуры простейших аккордов, которые можно унести с баррэ в любое место грифа.


