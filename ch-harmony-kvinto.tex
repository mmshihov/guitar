\section{Так квинтовый или квартовый? Квинто-квартовый круг}
\label{ch:harmony:kvinto-kvarto-round}

Замечателен тот факт (см. рисунок \ref{fig:harmony:interval:oct-round}), что для отдельно взятого звука в пределах октавы можно подобрать лишь \emph{два} звука, образующий с исходным звуком совершенный консонанс. Можно ли переупорядочить ноты на рисунке \ref{fig:harmony:interval:oct-round} так, чтобы звуки, образующие с исходным звуком совершенный консонанс, стали его (исходного звука) соседями: один справа, другой слева?

Можно, иначе быть не может! Давайте выпишем последовательность идущих друг за другом совершенных консонансов, отступая всякий раз кварту (5 полутонов). Начнем с ноты C. Кварта вверх от C --- это G. Кварта вверх от G --- D. И так далее. В результате получим:
\begin{center}
    C, G, D, A, E, B, 
\end{center}
