\section{Так квинтовый или квартовый? Квинто-квартовый круг}
\label{ch:harmony:kvinto-kvarto-round}

На самом деле полное название этого полезного помошника, которго легко сделать из бумаги: <<квинто-квартовый круг мажорных и минорных тональностей>>. Выглядит он странно: см. рисунок \ref{TODO}. И хотелось бы не только научиться им пользоваться, но и понять почему он именно такой.

TODO рисунок

Зачем он нужен:
\begin{itemize}
    \item Чаще всего круг используется как справочник <<сочетающихся>> аккордов для аккомпанемента песен.

    \item При исполнении песен под музыку также часто возникает задача сменить <<тональность>>. Человеческим языком: \emph{каждую} ноту нужно повысить или понизить на заданное количество полутонов. Зачем? Чтобы было удобнее петь. При этом интервальная структура (т.е. характер, например, веселый или грустный) музыки не изменится, а произведение <<в целом>> будет звучать ниже или выше, <<подстраиваясь>> таким образом под голос исполнителя.
    
    \item Обычно музыкальное произведение исполняется в одной тональности, но кто запрещает выйти в другую тональность прямо в процессе исполнения? Это называется модуляцией.
    
    \item Обсуждая тональности в разделе \ref{ch:harmony:lad}, мы узнали, что святая музыкальная инквизиция испльзует следующие пытки в музыкальных школах:
    \begin{itemize}
        \item А поглядите как, батенька, в ноты и по бемолям и диезам после знака ключа определите тональность произведения! 
        \item А скажите, милочка, в тональности РЕ-мажор где и какие знаки альтерации (диез,бемоль) мы поставим в нотах после ключа?!
    \end{itemize}
    Квинто-квартовый круг является компактной шпаргалкой, позволяющей быстро ответить на эти бессмысленные вопросы.
\end{itemize}

Замечателен тот факт (см. рисунок \ref{fig:harmony:interval:oct-round}), что для отдельно взятого звука в пределах октавы можно подобрать лишь \emph{два} звука, образующий с исходным звуком совершенный консонанс. Можно ли переупорядочить ноты на рисунке \ref{fig:harmony:interval:oct-round} так, чтобы звуки, образующие с исходным звуком совершенный консонанс, стали его (исходного звука) соседями: один справа, другой слева?

Можно, иначе быть не может! Давайте выпишем последовательность идущих друг за другом совершенных консонансов, отступая всякий раз кварту (5 полутонов). Начнем с ноты C. Кварта вверх от C --- это G. Кварта вверх от G --- D. И так далее. В результате получим:
\begin{center}
    C, G, D, A, E, B, 
\end{center}
