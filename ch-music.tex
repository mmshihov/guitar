\chapter{Музыка? Нет, не слышали\ldots}
\label{ch:music}

Когда автор учился в ВУЗе, преподаватель интеллектуальной собственности\footnote{Это о всяких там авторских правах и изобретениях} дал студентам красивый критерий, чтобы отличать произведение искусства от произведения техники: <<ребята, произведение искусства со временем становится только дороже, а техническое решение --- дешевле>>. Ну, то есть картины Леонардо Да-Винчи будут уходить с аукционов за космические деньги, а вот модель вашего крутого компа (телефона, и даже версия программы, которую вы напишете, и т.д.) со временем даром никому не будет нужна. 

К чему это я? Существует огромное количество сомнительных критериев отделить почётное \emph{искусство} от стрёмной науки. Выше был экономический критерий, мол искусство дорожает. 

А вот, например, критерий неврологический. В вашей черепушке сидит мозг с двумя полушариями: левым --- аналитическим, сравнивающим, разделяющим и рассуждающим, и правым --- синтезирующим, творческим, эмоциональном и витающем в облаках. И вот в какую сторону ваша черепушка перевесит, тем вы и будете: клонит влево --- физик, а вправо --- лирик. Физик разумный, а лирик эмоциональный. Физик в науке, лирик в искусстве. И дойдем в своих рассуждениях до абсурда: в искусстве уму не место, а наука лишена эмоций!

И вот у многих уже есть повод сказать: не лезьте в музыку с математикой, не путайте искусство с наукой! 

Ой, да ладно! Друзья, музыка волнует наши чувства только потому, что математика внутри вас\footnote{Да и наукой (настоящей) люди занимаются чисто по эмоциональным причинам: им любопытно!}. Эмоции и разум растут от одного корня.

\begin{Definition}[Музыка]
    Музыка начинается тогда, когда звуки \emph{правильной} высоты\footnote{\emph{Нота} обозначает звук нужной высоты} с \emph{правильной} громкостью\footnote{В музыке это называется \emph{акцентом}} звучат в \emph{правильное} время\footnote{В музыке есть определенный \emph{ритм}}.
\end{Definition}

Итак, разберемся с \emph{правилами} для высоты, громкости и времени звучания звуков.


\section{Немного <<правильной>> физики}

Если задеть гитарную струну, вы услышите звук. Струна начнет колебаться и создавать вокруг себя периодические сжатия и разряжения воздуха, которые затем будут распространяться во все стороны от струны. Этот эффект называется звуковой \emph{волной}\footnote{В отличие от волн на поверхности воды, которые распространяются <<кругами>> от упавшего в воду предмета, звуковые волны имеют эффект 3D и распространяются <<шарами>> --- во все стороны от источника звука.}. Волна распространяется от источника звука со скоростью примерно 340 метров в секунду.

Количество периодических сжатий (или разряжений) в секунду физики называют частотой, а лирики --- высотой звука. То есть чем чаще колеблется струна, тем \emph{выше} звук, который она издаёт. Ну и наоборот: чем реже колеблется струна, тем \emph{ниже} звук\footnote{И если частоту можно измерить точно, то оценки выше-ниже --- относительны и субъективны. Басом (низким голосом) разговаривает огромный спокойный мужик, а истеричка визжит на высоких тонах}.

Если вы дернете струну сильнее, она зазвучит громче. Струна от этого не станет колебаться быстрее, нет, она издаст звук той же высоты, что и обычно. Она будет совершать колебания с большим размахом --- физики скажут: <<с большей амплитудой>>. Из-за этого сжатия и разряжения воздуха усилятся. И, когда звук дойдет до уха слушателя, эти сжатия и разряжения начнут сильнее шатать барабанную перепонку в ухе и\ldots И физика кончится, а начнутся биология и информатика.

Человеческое ухо может различать такие характеристики звуковой волны, как её частота и амплитуда. Незатухающие колебания одной неизменной частоты, человек услышит как тон\footnote{Например, заходящий на посадку на ваше ухо комар, машет крыльями примерно 659.26 раз в секунду и вы слышите незабываемое Ми второй октавы!}. А амплитуду волны (размах колебаний) --- как громкость. Человеческий слуховой аппарат воспринимает ограниченный диапазон частот (примерно от 16 до 20000 Гц), а восприятие громкости звука, если честно, зависит не только от амплитуды звуковых колебаний, но и от частоты.


\section{Как звучит струна? Или правильная высота}

TODO: Равномерно-темперированный строй.

TODO:
% Частота от звука к звуку повышается в \emph{геометрической} прогрессии. То есть частота каждого следующего музыкального звука в \[\sqrt[12]{2}\approx 1,059463\] больше частоты звука предыдущего.
% 
% Так, следующий за ЛЯ первой октавы, звук ЛЯ-диез, имеет частоту $440\cdot\sqrt[12]{2}\approx 466,16$ герц. Звук СИ имеет частоту $440\cdot(\sqrt[12]{2})^2\approx 493,88$. И так далее, например, ЛЯ второй октавы имеет, как и положено, в два раза большую частоту, чем ЛЯ первой октавы: $440\cdot(\sqrt[12]{2})^{12}=440\cdot 2=880$ Гц.
% 
% Задав эталонную частоту любого музыкального звука, частоты для всех остальных звуков можно \emph{вычислить}.


Слияние звуков кратной частоты.

Октава.

\section{Маршируем? Вальсируем? Правильная громкость}
\label{sec:music:sounds}


\section{Еще и глушить надо? Правильное время}



