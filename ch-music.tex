\chapter{Музыка? Нет, не слышали\ldots}
\label{ch:music}

Когда автор учился в ВУЗе, преподаватель интеллектуальной собственности\footnote{Это о всяких там авторских правах и изобретениях} дал студентам красивый критерий, чтобы отличать произведение искусства от произведения техники: <<ребята, произведение искусства со временем становится только дороже, а техническое решение --- дешевле>>. Ну, то есть картины Леонардо Да-Винчи будут уходить с аукционов за космические деньги, а вот модель вашего крутого компа (телефона, и даже версия программы, которую вы напишете, и т.д.) со временем даром никому не будет нужна. 

К чему это я? Существует огромное количество сомнительных критериев отделить почётное \emph{искусство} от стрёмной науки. Выше был экономический критерий, мол искусство дорожает. 

А вот, например, критерий неврологический. В вашей черепушке сидит мозг с двумя полушариями: левым --- аналитическим, сравнивающим, разделяющим и рассуждающим, и правым --- синтезирующим, творческим, эмоциональном и витающем в облаках. И вот в какую сторону ваша черепушка перевесит, тем вы и будете: клонит влево --- физик, а вправо --- лирик. Физик разумный, а лирик эмоциональный. Физик в науке, лирик в искусстве. И дойдем в своих рассуждениях до абсурда: в искусстве уму не место, а наука лишена эмоций!

И вот у многих уже есть повод сказать: не лезьте в музыку с математикой, не путайте искусство с наукой! 

Ой, да ладно! Друзья, музыка волнует наши чувства только потому, что математика внутри вас\footnote{Да и наукой (настоящей) люди занимаются чисто по эмоциональным причинам: им любопытно!}. Эмоции и разум растут от одного корня.

\begin{Definition}[Музыка]
    Музыка начинается тогда, когда звуки \emph{правильной} высоты\footnote{\emph{Нота} обозначает звук нужной высоты} с \emph{правильной} громкостью\footnote{В музыке это называется \emph{акцентом}} звучат в \emph{правильное} время\footnote{В музыке есть определенный \emph{ритм}}.
\end{Definition}

Итак, разберемся с \emph{правилами} для высоты, громкости и времени звучания звуков.


\section{Немного правильной физики}


\section{Как звучит струна? Или правильная высота}

TODO: Равномерно-темперированный строй.

Слияние звуков кратной частоты.

Октава.

\section{Маршируем? Вальсируем? Правильная громкость}
\label{sec:music:sounds}


\section{Еще и глушить надо? Правильное время}



