\chapter{Музыка? Нет, не слышали\ldots}
\label{ch:music}

Сейчас мы зайдем на территорию искусства, вооружившись инструментами математики и физики. Читатель спросит: неужели мы сейчас начнем войну разума и чувств, науки и искусства? Нет и нет! Наоборот, мы будем наводить мосты дружбы между ними! 

Музыка отражает красоту математики, существующей \emph{внутри} нас, с помощью физики, существующей \emph{снаружи}. Музыка --- это один из способов передать математическую гармонию \emph{внутреннего} мира человека в несколько хаотичный \emph{окружающий} его мир.

А для тех, кто действительно не слышал музыку, дадим простое определение:

\begin{Definition}[Музыка]
    Музыка начинается тогда, когда звуки \emph{правильной} высоты с \emph{правильной} громкостью появляются в \emph{правильное} время
\end{Definition}

Давайте разбираться с музыкальными \emph{правилами}!


\section{Немного физики}
\label{ch:music:physics}

Если задеть гитарную струну, вы услышите звук. Струна начнет колебаться и создавать вокруг себя периодические сжатия и разряжения воздуха, которые затем будут распространяться во все стороны от струны. Этот эффект называется звуковой \emph{волной}. По воздуху звуковая волна распространяется от источника со скоростью примерно\footnote{Скорость звука зависит от температуры, давления и влажности воздуха. В воде же звук распространяется гораздо быстрее} 340 метров в секунду.

Количество периодических сжатий (или разряжений) в секунду физики называют \emph{частотой}\footnote{Единица измерения частоты --- Герц. Сокращенно --- Гц. 1Гц --- одно колебание в секунду. 400Гц --- четыреста колебаний в секунду}, а лирики --- \emph{высотой} звука. Чем чаще колеблется струна, тем \emph{выше} звук, который она издаёт. Ну и наоборот: чем реже колеблется струна, тем \emph{ниже} звук. Верхние струны на гитаре (шестая, пятая и четвертая), издающие относительно низкие звуки называются \emph{басовыми}.

Если вы дёрнете ту же струну сильнее, то она зазвучит \emph{громче}. При этом она издаст звук той же высоты, что и раньше. Но она будет совершать колебания с большим размахом --- физики скажут: <<с большей амплитудой>>. Сжатия и разряжения воздуха усилятся. И, когда звук дойдет до уха слушателя, эти сжатия и разряжения начнут сильнее шатать барабанную перепонку в его ухе и\ldots Тут физика кончится, и начнутся биология и биоинформатика.

Человеческое ухо может различать такие характеристики звуковой волны, как её частота и амплитуда. Незатухающие колебания одной неизменной частоты, человек услышит как тон\footnote{Например, заходящий на посадку на ваше ухо комар, машет крыльями примерно 659 раз в секунду и вы слышите незабываемое Ми второй октавы!}. А амплитуду волны (размах колебаний) --- как громкость. 

Человеческий слуховой аппарат воспринимает ограниченный диапазон частот --- примерно от 16 до 20000 Гц, а восприятие громкости звука, если честно, зависит не только от амплитуды звуковых колебаний, но и от частоты. 


\section{Как звучит струна? Правильная высота}
\label{ch:music:tone}

Если не менять натяжение и длину струны, то при игре она будет издавать звуки одной высоты (одного тона). Можно увеличить высту звука (увеличить частоту колебаний струны) двумя способами:
\begin{itemize}
    \item натянуть струну сильнее;
    \item укоротить струну, оставив натяжение прежним.
\end{itemize}

Перед игрой гитару настраивают, то есть регулируют определенным образом натяжение струн. В процессе игры натяжение отдельной струны не меняется. Гитарист извлекает звуки разной высоты, прижимая струну к разным металлическим порожкам лада, тем самым укорачивая её. Быстро и просто\footnote{Подробнее об устройстве гитары см. раздел \ref{ch:guitar}}. 

На самом деле струна порождает не одну звуковую волну определенной частоты, а сразу бесконечное множество волн, потому что независимо друг от друга колеблются половинки струны, её трети, четверти и так далее! 

\begin{figure}[!ht]
    \centering
    \includegraphics{fig/string-moving} 
    \caption{Колебания струны}\label{fig:music:tone:stringmoving}
\end{figure} 

На рисунке \ref{fig:music:tone:stringmoving} колебания струны разложены на составляющие (справа от знака равенства). Максимальную амплитуду колебаний имеет так называемый \emph{основной тон}, когда струна колеблется по всей длине. Основной тон слышен громче всех. Остальные составляющие принято называть \emph{обертонами}. Они звучат тише основного тона, но придают звуку особую окраску. Каждая гитара имеет свой особенный <<голос>> именно благодаря обертонам.

Когда говорят о высоте звука струны, то по умолчанию имеют в виду именно частоту колебаний \emph{основного тона}.

Набор звуков различной высоты, которые можно извлечь из музыкального инструмента называется \emph{строем}. Чтобы создать гармонию, высота (частота) музыкальных звуков должна укладываться в строгую математическую систему. Таких систем (строев) сложилось достаточно много\footnote{TODO: обозначить: Натуральная система, и пр}, но мы ограничимся только тем строем, который положен в основу конструкции гитары. Он называется <<равномерно-темперированным>>.

Прежде чем начать разбираться с равномерно-темперированным строем, важно определить очень важное понятие: \emph{расстояние} между двумя звуками. То есть ввести меру отличия одного звука от другого по \emph{высоте}. 

Если укоротить струну вдвое, то частота её колебаний вдвое увеличится. Если взять такие звуки на открытой струне и на её половинке по отдельности, то вы легко уловите разницу между ними <<на слух>>. Если сыграть такие звуки одновременно, то они почти сольются: <<на слух>> один звук просто потеряется в другом. Это объяснимо: как следует из рисунка \ref{fig:music:tone:stringmoving} самый громкий обертон имеет частоту вдвое меньшую, чем основной тон. То есть звук, издаваемый половинкой струны, полностью <<содержится>> в обертонах открытой струны.

\begin{Definition}[Октава]
    Два звука находятся друг от друга на расстоянии \emph{октавы}, если их частоты отличаются в два раза. Звук выше на октаву, если его частота в два раза больше частоты исходного звука. 
\end{Definition}

Ответ на вопрос: <<почему эта мера расстояния названа <<октавой>>?>> будет позже\footnote{Читайте раздел \ref{ch:harmony:interval}}. Октава --- это большое расстояние, а человеческое ухо воспринимает куда меньшие изменения частоты. Поэтому, внимание: \emph{исторически} октаву делят на 12 кусочков. Такой кусочек называется <<полутоном>>.

\begin{Example}[Октава на грифе гитары]
    Давайте немного разбавим теорию практикой: возьмите гитару и линейку (можно даже просто суровую нитку). Измерьте длину любой струны (от верхнего порожка на грифе, до порожка на подставке, см. рисунок \ref{fig:guitar:construction}, если возникли трудности). Поделите длину струны на два (сложите нитку вдвое) и убедитесь, что середина струны находится над 12-м ладовым порожком. Звук, извлеченный на 12-м ладу, на \emph{октаву выше} звука открытой струны. Звук, извлеченный на 1-м ладу, \emph{выше} звука открытой струны на \emph{полутон}.
\end{Example}

В равномерно-темперированном строе частота от звука к звуку повышается в \emph{геометрической} прогрессии. То есть частота звука, который на полутон выше исходного, в \[\sqrt[12]{2}\approx 1,059463\] больше частоты исходного звука. В такой же пропорции находятся и расстояния между ладовыми порожками гитары.

Помимо способа задать расстояния между звуками, нужно также определиться с точкой отсчета. Международным стандартом зафиксировано, что эталонная частота, от которой ведется отсчет составляет 440 Гц. Эта частота соотвествует ноте (подробно о нотах см. раздел \ref{ch:notes:names}) ЛЯ первой октавы. 

ЛЯ-диез первой октавы, который на полутон выше эталонного ЛЯ, имеет частоту $440\cdot\sqrt[12]{2}\approx 466,16$ герц. Эти различия ощущаются на слух. Звук СИ первой октавы (два полутона от эталонного ЛЯ) имеет частоту $440\cdot(\sqrt[12]{2})^2\approx 493,88$. И так далее, например, ЛЯ второй октавы (на 12 полутонов, то есть не октаву выше эталонного) имеет, как и положено, в два раза большую частоту: $440\cdot(\sqrt[12]{2})^{12}=440\cdot 2=880$ Гц.


\section{Маршируем или Вальсируем? Правильная громкость}
\label{ch:music:volume}


\section{Еще и глушить надо? Правильное время}
\label{ch:music:rythm}



