\chapter{Музыка? Нет, не слышали\ldots}
\label{ch:music}

Сейчас мы зайдем на территорию искусства, вооружившись инструментами математики и физики. Читатель спросит: неужели мы сейчас начнем войну разума и чувств, науки и искусства? Нет и нет! Наоборот, мы будем наводить мосты дружбы между ними! 

Музыка отражает красоту математики, существующей \emph{внутри} нас, с помощью физики, существующей \emph{снаружи}. Музыка --- это один из способов передать математическую гармонию \emph{внутреннего} мира человека в несколько хаотичный \emph{окружающий} его мир.

А для тех, кто действительно не слышал музыку, дадим простое определение:

\begin{Definition}[Музыка]
    Музыка начинается тогда, когда звуки \emph{правильной} высоты с \emph{правильной} громкостью появляются в \emph{правильное} время
\end{Definition}

Давайте разбираться с музыкальными \emph{правилами}!


\section{Немного физики}
\label{ch:music:physics}

Если задеть гитарную струну, вы услышите звук. Струна начнет колебаться и создавать вокруг себя периодические сжатия и разряжения воздуха, которые затем будут распространяться во все стороны от струны. Этот эффект называется звуковой \emph{волной}\footnote{В отличие от волн на поверхности воды, которые распространяются <<кругами>> от упавшего в воду предмета, звуковые волны имеют эффект 3D и распространяются <<шарами>> --- во все стороны от источника звука.}. По воздуху звуковая волна распространяется от источника со скоростью примерно\footnote{Скорость звука зависит от температуры, давления и влажности воздуха. В воде же звук распространяется гораздо быстрее} 340 метров в секунду.

Количество периодических сжатий (или разряжений) в секунду физики называют частотой, а лирики --- высотой звука. Чем чаще колеблется струна, тем \emph{выше} звук, который она издаёт. Ну и наоборот: чем реже колеблется струна, тем \emph{ниже} звук.

Если вы дернете ту же струну сильнее, то она зазвучит \emph{громче}. При этом она издаст звук той же высоты, что и раньше. Но она будет совершать колебания с большим размахом --- физики скажут: <<с большей амплитудой>>. Сжатия и разряжения воздуха усилятся. И, когда звук дойдет до уха слушателя, эти сжатия и разряжения начнут сильнее шатать барабанную перепонку в его ухе и\ldots Тут физика кончится, и начнутся биология и биоинформатика.

Человеческое ухо может различать такие характеристики звуковой волны, как её частота и амплитуда. Незатухающие колебания одной неизменной частоты, человек услышит как тон\footnote{Например, заходящий на посадку на ваше ухо комар, машет крыльями примерно 659 раз в секунду и вы слышите незабываемое Ми второй октавы!}. А амплитуду волны (размах колебаний) --- как громкость. 

Человеческий слуховой аппарат воспринимает ограниченный диапазон частот (примерно от 16 до 20000 Гц), а восприятие громкости звука, если честно, зависит не только от амплитуды звуковых колебаний, но и от частоты. 
%TODO: восприятие индивидуально простыми словами


\section{Как звучит струна? Правильная высота}
\label{ch:music:tone}

TODO: Равномерно-темперированный строй.

TODO:
% Частота от звука к звуку повышается в \emph{геометрической} прогрессии. То есть частота каждого следующего музыкального звука в \[\sqrt[12]{2}\approx 1,059463\] больше частоты звука предыдущего.
% 
% Так, следующий за ЛЯ первой октавы, звук ЛЯ-диез, имеет частоту $440\cdot\sqrt[12]{2}\approx 466,16$ герц. Звук СИ имеет частоту $440\cdot(\sqrt[12]{2})^2\approx 493,88$. И так далее, например, ЛЯ второй октавы имеет, как и положено, в два раза большую частоту, чем ЛЯ первой октавы: $440\cdot(\sqrt[12]{2})^{12}=440\cdot 2=880$ Гц.
% 
% Задав эталонную частоту любого музыкального звука, частоты для всех остальных звуков можно \emph{вычислить}.


Слияние звуков кратной частоты.

Октава.

\section{Маршируем? Вальсируем? Правильная громкость}
\label{ch:music:volume}


\section{Еще и глушить надо? Правильное время}
\label{ch:music:rythm}



