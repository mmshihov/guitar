\chapter{Элементы нотной грамоты}
\label{ch:note}

\emph{Нота} --- колебания воздуха с некоторой постоянной частотой. Такие колебания некоторое время производит гитарная струна после щипка --- особого удара пальцем по ней. Нота выше, если частота колебаний струны больше. Ну и низкая нота --- это колебания с малой частотой. Например, в соответствии с международным стандартом, струна, звучащая на ноте Ля первой октавы (о октавах позже) колеблется с частотой 440 герц, то есть совершает 440 полных колебаний в секунду\footnote{Допускается вольность принять за Ля первой октавы любую частоту из интервала от 430 до 450 герц. Многие фанатики утверждают, что Ля в 432 Гц от Бога (и оздоравливающе действует на организм), а стандартизованная 440 Гц --- от Сатаны, соответственно (и разрушает психику).}

Человеческое ухо способно различать такие характеристики звуковой волны, как её частота и амплитуда. Колебания одной неизменной частоты, человек услышит как тон. А амплитуду волны (размах колебаний) --- как громкость. Человеческий слуховой аппарат воспринимает ограниченный диапазон частот (примерно от 16 до 20000 Гц), а восприятие громкости звука, если честно, зависит не только от амплитуды звуковых колебаний, но и от его частоты.

Традиционно принято использовать семь обозначений для кодирования нот (в порядке возрастания частоты/высоты звука): До, Ре, Ми, Фа, Соль, Ля, Си. Пусть мы дошли до последней ноты Си и... Да, текущая \emph{октава} кончилась. Но началась следующая! В следующей октаве все повторится: До, Ре, Ми, Фа, Соль, Ля, Си. Только вот частота соотвествующей ноты будет в \emph{два} раза больше! То есть, например, ноте Ля второй октавы соответствует вдвое большая частота (880 Гц), чем ноте Ля первой октавы (стандартные 440 Гц).

Настало время поговорить об исторически сложившейся сложности кодирования музыки.

Традиционно в октаве принято выделять \emph{двенадцать} нот! Погодите, скажете вы: До, Ре, Ми, Фа, Соль, Ля, Си --- семь! Увы это названия лишь семи нот из 12. А вот (12-7)=5 нот не удостоились отдельных обозначений. Итак, оставшиеся 5 нот находятся между упорядоченными по частоте нотами:
1) До и Ре (и может быть названа двояко: либо До-диез, либо Ре-бемоль)
2) Ре и Ми (Ре-диез или Ми-бемоль)
3) Фа и Соль (Фа-диез или Соль-бемоль)
4) Соль и Ля (Соль-диез или Ля-бемоль)
5) Ля и Си (Ля-диез или Си-бемоль)
Эти ноты имеют, как видно сразу два имени.

Все 12 нот октавы в порядке увеличения высоты:

До, До-диез, Ре, Ре-диез, Ми, Фа, Фа-диез, Соль, Соль-диез, Ля, Ля-диез, Си

Если взять клавиатуру фортепиано, то можно увидеть однозначное соответствие между нотами и клавишами.

Частота от ноты к ноте повышается равномерно, в геометрической прогрессии. Частота следующей ноты в \[\sqrt[12]{2}\approx 1,059463.\] больше частоты предыдущей. Так, нота Ля-диез имеет частоту $440\cdot\sqrt[12]{2}\approx 466,16$ герц. Нота Си имеет частоту $440\cdot\sqrt[12]{2}\cdot\sqrt[12]{2}\approx 493,88$. Имея эталонную частоту Ля первой октавы (440 Гц), можно вычислить частоты для всех остальных нот во всех октавах.

Многие источники вам скажут, что добавление к основному обозначению суффикса <<диез>> означает повышение частоты на <<полтона>>, что математически должно соответствовать увеличению частоты в $\sqrt[12]{2}$ раз. И это в целом правильно, только обычно ученики начинают думать, что между соседними нотами из списка До, Ре, Ми, Фа, Соль, Ля, Си --- расстояние в тон! Нет, потому что между нотами Ми и Фа, а также между Си и До "промежуточных" нот нет. Как видно из реальной последовательности, следуя такой логике, например, Ми-диез --- это Фа, или Си-диез --- это До следующей октавы. Обычно ноту Фа, конечно никто не называет Ми-диез --- это оскорбительно, но никто и не запрещает так делать. 

То же самое можно сказать и о суффиксе <<бемоль>>, он понижает ноту на полтона. И, например, До-диез, это та же нота, что и Ре-бемоль. А если вы хотите оскорбить ноту Ми, то назовите её Фа-бемоль. Если вы хотите прочитать все 12 нот в обратном порядке, то грамотно будет использовать бемоль, а не диез:

Си, Си-бемоль, Ля, Ля-бемоль, Соль, Соль-бемоль, Фа, Ми, Ми-бемоль, Ре, Ре-бемоль, До.


Пройдемся по всем октавам, покрывающим возможности человеческого слухового аппарата:
субконтроктава
контроктава
большая октава
малая октава
первая октава
вторая октава
третья октава
четвертая октава
пятая октава


