\chapter{Элементы нотной грамоты}
\label{ch:note}

Нотную запись нельзя назвать эталоном простоты. Она, несомненно, сложнее, чем могла бы быть. Так уж сложилось, и, уважая гениальность предков, мы изучим её такой, какая она есть.


\section{Немного теории}

\emph{Нота} --- это \emph{обозначение} (способ записи, если хотите) колебаний воздуха с некоторой постоянной частотой. Такие колебания некоторое время производит гитарная струна после щипка --- особого удара пальцем по ней. Нота звучит \emph{выше}, если частота колебаний струны \emph{больше}. Ну а низкая нота --- это колебания с малой частотой. Например, в соответствии с международным стандартом, струна, звучащая на ноте Ля первой октавы (о октавах позже) колеблется с частотой 440 герц, то есть совершает 440 полных колебаний в секунду\footnote{Допускается вольность принять за Ля первой октавы любую частоту из интервала от 430 до 450 герц. Многие фанатики утверждают, что Ля в 432 Гц от Бога (и оздоравливающе действует на организм), а стандартизованная 440 Гц --- от Сатаны, соответственно (и разрушает психику).}.

Человеческое ухо способно различать такие характеристики звуковой волны, как её частота и амплитуда. Незатухающие колебания одной неизменной частоты, человек услышит как тон\footnote{Например, заходящий на посадку на ваше ухо комар, машет крыльями примерно 659.26 раз в секунду и вы слышите незабываемое Ми второй октавы!}. А амплитуду волны (размах колебаний) --- как громкость. Человеческий слуховой аппарат воспринимает ограниченный диапазон частот (примерно от 16 до 20000 Гц), а восприятие громкости звука, если честно, зависит не только от амплитуды звуковых колебаний, но и от частоты.

В Русской традиции принято использовать \emph{семь} обозначений для кодирования нот (которые приведены ниже в порядке возрастания частоты колебаний (высоты звука)): 
\begin{center}
    До, Ре, Ми, Фа, Соль, Ля, Си. 
\end{center}

Так что получается, мы имеем дело всего с семью звуками? Конечно нет!

Пусть мы дошли до последней ноты Си и... И текущая \emph{октава} кончилась. Но началась следующая! В следующей октаве все повторится: До, Ре, Ми, Фа, Соль, Ля, Си. Только вот частота соотвествующей ноты будет в \emph{два} раза больше! То есть, например, ноте Ля второй октавы соответствует вдвое большая частота (880 Гц), чем ноте Ля первой октавы (стандартные 440 Гц).

Октавы, использующиеся в музыке, имеют следующие названия.
\begin{center}
    \begin{tabular}{ll}
        \hline\hline
        Название октавы         & Частота ноты Ля, Гц \\
        \hline\hline
        
        Субконтроктава          & 27.5 \\
        Контроктава             & 55   \\
        \emph{Большая октава}   & 110  \\
        \emph{Малая октава}     & 220  \\
        \emph{Первая октава}    & \fbox{440}  \\
        \emph{Вторая октава}    & 880  \\
        Третья октава           & 1760 \\
        Четвертая октава        & 3520 \\
        Пятая октава            & 7040 \\
        \hline
    \end{tabular}
\end{center}

В таблице \emph{выделены} те октавы, которые входят в диапазон шестиструнной гитары. Справедливости ради следует сказать, диапазон звучания гитары полностью включает лишь малую и первую октавы.

А вот теперь пришло время для секретов: традиционно в октаве принято выделять \emph{двенадцать} нот! 

Погодите, погодите, скажете вы: <<До, Ре, Ми, Фа, Соль, Ля, Си>> --- семь нот! И октава названа, вероятно, не просто так: <<octo>> --- это восемь, восьмая нота! Все логично!

Сам в шоке! Но увы, <<До, Ре, Ми,\ldots>> --- это названия лишь семи нот из 12. А вот (12-7)=5 нот не удостоились отдельных имен. Итак, оставшиеся 5 нот находятся между упорядоченными по частоте нотами:
\begin{enumerate}
    \item До и Ре (эта нота может быть названа либо До-диез, либо Ре-бемоль)
    \item Ре и Ми (Ре-диез или Ми-бемоль)
    \item Фа и Соль (Фа-диез или Соль-бемоль)
    \item Соль и Ля (Соль-диез или Ля-бемоль)
    \item Ля и Си (Ля-диез или Си-бемоль)
\end{enumerate}    

Эти <<промежуточные>> ноты имеют сразу два названия. Все 12 нот октавы в порядке увеличения высоты:
\begin{center}
    До, \emph{До-диез}, Ре, \emph{Ре-диез}, Ми, Фа, \emph{Фа-диез}, Соль, \emph{Соль-диез}, Ля, \emph{Ля-диез}, Си
\end{center}

Частота от ноты к ноте повышается равномерно, в \emph{геометрической} прогрессии. Частота каждой следующей ноты в \[\sqrt[12]{2}\approx 1,059463\] больше частоты предыдущей. 

Так, следующая за нотой Ля первой октавы, нота Ля-диез, имеет частоту $440\cdot\sqrt[12]{2}\approx 466,16$ герц. Нота Си имеет частоту $440\cdot(\sqrt[12]{2})^2\approx 493,88$. И так далее, например, Ля второй октавы имеет в два раза большую частоту, чем Ля первой октавы: $440\cdot(\sqrt[12]{2})^{12}=440\cdot 2=880$ Гц.

Задав эталонную частоту любой ноты, частоты для всех остальных нот можно вычислить.

В учебниках говорится, что добавление суффикса <<диез>> означает повышение частоты исходной ноты на <<полутон>>, что математически соответствует увеличению частоты в $\sqrt[12]{2}$ раз. И это в целом правильно, только обычно ученики начинают думать, что между соседними нотами из списка До, Ре, Ми, Фа, Соль, Ля, Си --- расстояние в два <<полутона>> (то есть в <<тон>>)! Не забыайте, что между нотами Ми и Фа, а также между Си и До <<промежуточных>> нот нет и расстояние между ними --- один <<полутон>>. 

Как видно из реальной последовательности, следуя такой логике, например, Ми-диез --- это Фа, или Си-диез --- это До следующей октавы. Обычно ноту Фа, конечно никто не называет Ми-диез --- это оскорбительно, но никто и не запрещает так делать. 

То же самое можно сказать и о суффиксе <<бемоль>>, он понижает ноту на полтона. И, например, До-диез, это та же нота, что и Ре-бемоль. А если вы хотите оскорбить ноту Ми, то назовите её Фа-бемоль. Если вы хотите прочитать все 12 нот в обратном порядке, то грамотно будет использовать суффикс <<бемоль>>, а не <<диез>>:

\begin{center}
Си, Си-бемоль, Ля, Ля-бемоль, Соль, Соль-бемоль, Фа, Ми, Ми-бемоль, Ре, Ре-бемоль, До.
\end{center}


\section{Устройство гитары}

Мы разобрались с математикой нот в предыдушем разделе. Физика же такова, что если <<открытая>> струна звучит, например, нотой Ми большой октавы (так звучит шестая, самая толстая струна), то зажатая на 12 ладу она будет звучать октавой выше --- Ми малой октавы, то есть давать колебания вдвое большей частоты. 

Так как физика дает нам закон, что частота колебаний струны обратно пропорциональна её длине, то двенадцатый лад на грифе гитары делит струну пополам\footnote{Надо честно заметить, что частота колебаний струны зависит также и от силы её натяжения, которая меняется, когда струну <<зажимают>> на ладу. Но это влияние столь незначительно, что им можно пренебречь.}.

Исходя из того, что частота каждой следующей ноты в $\sqrt[12]{2}$ больше предыдущей, запишем формулу длины струны ($L$) от места крепления струны к подставке до $n$-го ладового порожка:

\[L(n)=\frac{L}{(\sqrt[12]{2})^n},\]
где $n$ - номер лада ($0$-й лад соответствует открытой струне), а $L$ --- общая длина струны от подставки до верхнего порожка.


\includegraphics{fig/string-length.png}

Ушастые выпендрежники говорят, что различают больше 12 нот в октаве! И им мало 12 ладов! На некоторых гитарах можно заметить дополнительные ладовые порожки между <<каноническими>>, которые позволяют <<всунуть>> дополнительную ноту.



\section{Запись гитарных нот на бумаге}

\section{Гитарная табулатура}

