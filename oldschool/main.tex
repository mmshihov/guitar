\documentclass[a4paper,oneside,14pt]{scrbook}


\usepackage[utf8]{inputenc}
\usepackage[russian]{babel}
\usepackage{indentfirst}
\usepackage{verse}
\usepackage[colorlinks=true]{hyperref}
\usepackage{graphicx}
\usepackage{arcs}
\usepackage{amsmath}
\usepackage{amssymb}

\title{Сборник песен, стихов и частушек}
\author{М.~М.~Шихов}
\date{\today}

%ненормативная лексика
\newcommand{\myPiz}[2]{$^{\text{#1}}${\large{$\boldsymbol{\pi}$}}$_{\text{#2}}$}
\newcommand{\myEbt}[2]{$^{\text{#1}}${\large{$\boldsymbol{\varepsilon}$}}$_{\text{#2}}$}
\newcommand{\myMnd}[2]{$^{\text{#1}}${\large{$\boldsymbol{\mu}$}}$_{\text{#2}}$}
\newcommand{\myXyi}[2]{$^{\text{#1}}${\large{$\boldsymbol{\chi}$}}$_{\text{#2}}$}
\newcommand{\myXer}[2]{$^{\text{#1}}${\large{$\boldsymbol{\chi r}$}}$_{\text{#2}}$}
\newcommand{\myBld}[2]{$^{\text{#1}}${\large{$\boldsymbol{\beta}$}}$_{\text{#2}}$}
\newcommand{\mySuk}[2]{$^{\text{#1}}${\large{$\boldsymbol{s}$}}$_{\text{#2}}$}
\newcommand{\myCel}[2]{$^{\text{#1}}${\large{$\boldsymbol{\lceil\varnothing\rceil}$}}$_{\text{#2}}$}
\newcommand{\myAss}[2]{$^{\text{#1}}${\large{$\boldsymbol{\omega}$}}$_{\text{#2}}$}

\begin{document}

    \maketitle
    \tableofcontents
    \renewcommand{\poemtoc}{subsection}
    \newcommand{\attrib}[1]{\nopagebreak{\raggedleft\footnotesize #1\par}}

    \chapter*{Предисловие от составителя}
    \addcontentsline{toc}{chapter}{Предисловие}
    
    Говорят: <<из песни слова не выкинешь>>. Увы, человеческая память коротка, что-то забудется само, что-то мы захотим забыть и забудем. 
    
    Сохранить хоть что-нибудь мучительно захотелось после кончины Вахрушевой Валентины Алексеевны, моей тётушки. Уверен, что многое из собранного было не раз рассказано и спето Валентиной Алексеевной. Мучительно жаль, что самую большую ценность предстваляет именно то, \emph{как} это было сделано. Увы, ни один носитель информации, кроме человека, не сохраняет того душевного тепла, которое было подарено в момент исполнения. Век человека недолог, но однажды полученное тепло захочется передать потомкам. Бумага, увы, тут не поможет: она лишь пронесет сквозь время то, \emph{что} может быть поможет кому-то выполнить эту передачу в течение жизни. А после текст на бумаге станет лишь сборником песен, стихов и частушек и ничем иным. Что тоже неплохо.
    
    Об авторском праве. Автором многого является сама Валентина Алексеевна, авторами оставшегося являются (чаще всего) неизвестные мне лица. Когда нет возможности воздать им должное\footnote{Хотя бы упоминанием в примечании}, буду считать их \emph{русским народом}. 
    
    \chapter{Песни}
    
    \poemtitle{Изгиб гитары жёлтой\ldots}
    \settowidth{\versewidth}{Как здорово, что все мы здесь сегодня собрались}
    \begin{verse}[\versewidth]
        Изгиб гитары жёлтой\footnote{звонкой} ты обнимаешь нежно,\\*
        Струна осколком эха пронзит тугую высь.\\*
        Качнется купол неба большой и звездно-снежный,\\*
        Как здорово, что все мы здесь сегодня собрались.\\*
        \vin Качнется купол неба большой и звездно-снежный,\\*
        \vin Как здорово, что все мы здесь сегодня собрались.

        Как отблеск от заката, костер меж сосен пляшет,\\*
        Ты что грустишь бродяга? А ну-ка улыбнись!\\*
        И кто-то очень близкий тебе тихонько скажет:\\*
        <<Как здорово, что все мы здесь сегодня собрались!>>\\*
        \vin И кто-то очень близкий тебе тихонько скажет:\\*
        \vin <<Как здорово, что все мы здесь сегодня собрались!>>

        И все же с болью в горле, мы тех сегодня вспомним,\\*
        Чьи имена, как раны, на сердце запеклись,---\\*
        Мечтами их и песнями, мы каждый вдох наполним:\\*
        Как здорово что все мы здесь сегодня собрались!\\*
        \vin Мечтами их и песнями, мы каждый вдох наполним:\\*
        \vin Как здорово что все мы здесь сегодня собрались!
    \end{verse}
    \attrib{Олег Митяев}

    \poemtitle{Однажды морем я плыла\ldots}
    \settowidth{\versewidth}{Ай-яй, в глазах туман, кружится голова,}
    \begin{verse}[\versewidth]
        Однажды морем я плыла на пароходе том,\\*
        Погода чудная была, но вдруг начался шторм.\\
        
        \vin Ай-яй, в глазах туман, кружится голова,\\*
        \vin Едва стою я на ногах, но я ведь не пьяна.\\
        \vin Ай-яй, в глазах туман, кружится голова,\\*
        \vin Едва стою я на ногах, но я ведь не пьяна. 

        А капитан приветлив был, в каюту пригласил,\\*
        Налил шампанского бокал и выпить предложил.\\

        \vin Ай-яй, в глазах туман, кружится голова,\\*
        \vin Едва стою я на ногах, но я ведь не пьяна.\\
        \vin Ай-яй, в глазах туман, кружится голова,\\*
        \vin Едва стою я на ногах, но я ведь не пьяна.

        Бокал я выпила до дна, в каюте прилегла,\\*
        И то, что с детства берегла, ему я отдала.\\

        \vin Ай-яй, в глазах туман, кружится голова,\\*
        \vin Едва стою я на ногах, но я ведь не пьяна.

        А через год родился сын, морской волны буян,\\*
        И кто же в этом виноват? Конечно, капитан.\\

        \vin Ай-яй, в глазах туман, кружится голова,\\*
        \vin Едва стою я на ногах, но я ведь не пьяна.

        С тех пор прошло немало лет, как морем я плыла,\\*
        А как увижу пароход, кружится голова.\\

        \vin Ай-яй, в глазах туман, кружится голова,\\*
        \vin Едва стою я на ногах, но я ведь не пьяна.

        Умейте жить, умейте пить и всё от жизни брать,\\*
        Ведь всё равно когда-нибудь придётся умирать.\\

        \vin Ай-яй, в глазах туман, кружится голова,\\*
        \vin Едва стою я на ногах, но я ведь не пьяна.\\
        \vin Ай-яй, в глазах туман, кружится голова,\\*
        \vin Едва стою я на ногах, и все же я пьяна!
    \end{verse}
    \attrib{Народная песня}
    
    \poemtitle{Ой, калина\ldots (Кадышева)}
    \settowidth{\versewidth}{Ой, калина, ой, малина в речке тихая вода,}
    \begin{verse}[\versewidth]
        Ой, калина, ой, малина в речке тихая вода,\\*
        Ты скажи, скажи калина как попала ты сюда?

        Как-то раннею весной парень бравый прискакал,\\*
        Долго мною любовался, а потом со собой забрал.

        Он хотел меня калину, посадить в своём саду\\*
        Не довез он, в землю бросил, думал, что я пропаду.

        Я за землю ухватилась, стала на ноги свои\\*
        Навсегда здесь поселилась, где шебечут соловьи

        Трактористы, комбайнёры, каждый день\footnote{??? И зимой? Наверное, все таки, год\ldots} бывают тут\\*
        Тонких веток не ломают, цвет мой белый берегут.

        Ты не дуй, голубчик ветер, не считай за сироту,\\*
        Я с землёю породнилась, вот по прежнему цвету.

        Ой, калина, ох, малина, не кручинься ты совсем\footnote{Отчаянное совсем},\\*
        Ты цвети, цвети калина, ты цвети на радость всем.\\*
        Ты цвети, цвети калина, ты цвети на радость всем.    
    \end{verse}
    \attrib{Народная песня в обработке Кадышевой}

    \poemtitle{Ой, калина\ldots (казачья)}
    \settowidth{\versewidth}{Ой калина, ой малина, в речке талая вода,}
    \begin{verse}[\versewidth]
        Ой, калина, ой, малина, в речке талая вода,\\*
        Расскажу я вам подружки, как попала я сюда.\\*
        \vin Расскажу я вам подружки, как попала я сюда.
        
        Как-то раннею весною парень бравый прискакал,\\*
        Долго мною любовался, а потом со собою взял.\\*
        \vin Долго мною любовался, а потом со собою взял.

        Он хотел меня калину, посадить в своём саду.\\*
        Не довез он, в землю бросил, думал, что я пропаду.\\*
        \vin Не довез он, в землю бросил, думал, что я пропаду.

        Я за землю ухватилась, стала на ноги свои\\*
        И навеки поселилась, там, где свищут соловьи.\\*
        \vin И навеки поселилась, там, где свищут соловьи.

        Трактористы, комбайнёры, каждый год бывают тут.\\*
        Тонких веток не ломают, цвет мой белый берегут.\\*
        \vin Тонких веток не ломают, цвет мой белый берегут.

        Ты не дуй, холодный ветер, не трепли мою красу,\\*
        Надо мною солнце светит --- я по-прежнему цвету.\\*
        \vin Надо мною солнце светит --- я по-прежнему цвету.

        Ой, калина, ой, малина в речке талая вода,\\*
        Рассказала, вам подружки, как попала я сюда.\\*
        \vin Рассказала, вам подружки, как попала я сюда.
    \end{verse}
    \attrib{Народная песня.\\Юрий Щербаков, ансамбли казачьей песни}

    \poemtitle{Сорвали розу}
    \settowidth{\versewidth}{Я встретил розу, она цвела}
    \begin{verse}[\versewidth]
        Я встретил розу, она цвела,\\*
        И ароматом была полна.\\
        Я эту розу сорвать готов,\\*
        Но побоялся её шипов.\\
        Я эту розу сорвать готов,\\*
        Но побоялся её шипов.

        \vin А утром рано я в сад вошёл,\\*
        \vin Но этой розы я не нашёл.\\
        \vin <<Ой, роза, роза, --- я закричал, ---\\*
        \vin Зачем тебя я да не сорвал?!>>\\
        \vin <<Ой, роза, роза, --- я закричал, ---\\*
        \vin Зачем тебя я да не сорвал?!>>

        Я побоялся шипов твоих,\\*
        Теперь ты, роза, в руках чужих.\\
        Сорвали розу, помяли цвет,\\*
        А этой розе семнадцать лет.\\
        Сорвали розу, помяли цвет,\\*
        А этой розе семнадцать лет.

        \vin Ой, парни, парни, мой вам совет ---\\*
        \vin Не рвите розу в семнадцать лет.\\
        \vin Розы прекрасны, розы нежны,\\*
        \vin И эти розы нам всем нужны.\\
        \vin Розы прекрасны, розы нежны,\\*
        \vin И эти розы нам всем нужны.

        Я встретил розу, она цвела,\\*
        И ароматом была полна.\\
        Я эту розу сорвать готов, \\*
        Но побоялся её шипов.\\
        Я эту розу сорвать готов,\\*
        Но побоялся её шипов.        
    \end{verse}
    \attrib{Народная}
        
        
    \poemtitle{Ах, судьба моя, судьба}
    \settowidth{\versewidth}{Никто нас в церкви не венчал}
    \begin{verse}[\versewidth]
        Никто нас в церкви не венчал,\\*
        А вся душа горит в огне.\\*
        Зачем, казак, ты в степь умчал\\*
        На вороном своем коне?\\
        Зачем ты встретился со мной,\\*
        Когда в Дону коня поил?\\*
        Зачем чубатой красотой\\*
        Казачке сердце покорил? 

        \vin Припев:\\*
        \vin Ах, судьба моя, судьба, ах, судьба,\\*
        \vin Ах, судьба моя, скажи, почему,\\*
        \vin Ах, судьба моя, разлука - судьба,\\*
        \vin Я ответ найти никак не могу?

        Зачем бросал сирень-цветы\\*
        В мое полночное окно?\\*
        Зачем всех в мире лучше ты,\\*
        Как солнца свет, когда темно?\\
        Зачем желанный ты такой,\\*
        Как синеглазый вешний Дон?\\*
        Зачем в станице за тобой\\*
        Казачки ходят табуном? 

        \vin Припев. 
        
        Как у мечты своей спросить,\\*
        Бедой-разлукой не грозя?\\*
        Ни в чем винить, ни позабыть,\\*
        Ни разлюбить тебя нельзя.\\
        Зачем взлетел ты на коня,\\*
        Умчал в лазоревый рассвет?\\*
        Ни у тебя, ни у меня,\\*
        Ни у судьбы ответа нет!        

        \vin Припев. 
    \end{verse}
    \attrib{Народная. Красиво исполняется Надеждой Кадышевой}

    
    \chapter{Стихи с элементами высшей \overarc{\emph{мат}}ематики}
    
    \poemtitle{Страдания}
    \settowidth{\versewidth}{Дома спать с своей женой, чтобы не чесаться!}
    \begin{verse}[\versewidth]
        Дайте в руки мне гармонь,\\*
        Золотые планки!\\*
        {\myMnd{}{вошек}} наловил\\*
        От одной засранки.
        
        \vin И как начали кусать\\*
        \vin Весело и дружно!\\*
        \vin Даже руки из штанов\\*
        \vin Вынимать не нужно!
        
        Вот пошел в аптеку я,\\*
        Чтоб купить втирание.\\*
        {\myMnd{}{вошкам}} хоть бы {\myXyi{}{}}!\\*
        А я пою страдания\ldots
        
        \vin Только мазь не помогла ---\\*
        \vin Яйца лишь облезли.\\*
        \vin Так и надо {\myBld{}{нам}},\\*
        \vin Чтоб к другим не лезли!
        
        Попросил прощенья я\\*
        У своей супруги:\\*
        <<Ты прости меня жена ---\\*
        Не пойду к {\myBld{}{юге}}!
        
        \vin Буду верен я тебе до сам\'{о}го гроба!\\*
        \vin Помоги их истребить только ради Бога!>>
        
        И ответила жена: <<На {\myXer{}{}} ты мне сдался!\\*
        Пусть пол-{\myXyi{}{}} отъедят, чтобы не шатался!>>
        
        \vin Все же сжалилась жена, сделала втиранье,\\*
        \vin А от боли я, друзья, пел всю ночь страдания!
        
        {\myMnd{}{вошек}} больше нет, но скажу вам, други:\\*
        Так натерла {\myXyi{}{}} жена --- не пойдешь к {\myBld{}{юге}}\ldots
        
        \vin Вам советую друзья: лучше не шататься ---\\*
        \vin Дома спать с своей женой, чтобы не чесаться!
    \end{verse}    
    
    \poemtitle{Ворона и сыр}
    \settowidth{\versewidth}{Ворона каркнула во все своё ебало}
    \begin{verse}[\versewidth]
        В лесу однажды был {\myPiz{}{ворот}}:\\*
        Ворона, {\myEbt{}{ная}} в рот,\\
        Головку сыра {\myPiz{c}{нула}},\\*
        Довольная, на сук вспорхнула:\\
        Сыр --- это вам не сало!\\*
        {\myEbt{Вы}{ться}} стала.
        
        \vin На ту беду лесной тропой\\*
        \vin Лису за {\myBld{}{во}} вёл конвой.\\
        \vin Она от них оторвалась,\\*
        \vin Да к дереву подобралась:
        
        <<Пой, {\mySuk{}{}}, не стыдись!\\
        Давно блатных я песен не слыхала!>>\\*
        Ворона каркнула во все своё {\myEbt{}{ало}},\\*
        И сыра у нее --- тотчас как не бывало\ldots
        
        \vin <<Отдай, лисица, сыр, а то слечу\\*
        \vin И все {\myEbt{}{ло}} тебе разворочу!\\
        \vin И волку напишу такую фразу,\\*
        \vin Чтоб {\myEbt{вы}{л}} тебя, заразу!>>
        
        <<А {\myXyi{}{ль}} мне волк!\\
        Я со слоном {\myEbt{}{ась}},\\
        От Мишки сделала аборт,\\*
        И с ним же еду на курорт!>>
        
        \vin Моралей в басне целых три.\\*
        \vin 1) Не городи пустые фразы.\\*
        \vin 2) Где {\myPiz{с}{ил}} --- там и жри.\\
        \vin 3) И {\myXyi{не}{}} по деревьям лазить!
    \end{verse}
    \attrib{По мотивам басни И.~А.~Крылова\footnote{Всего на данный момент имеется около 45 вариантов подобных переделок этой басни}}

    \poemtitle{Три сестрицы}
    \settowidth{\versewidth}{Три сестрицы под окном}
    \begin{verse}[\versewidth]
        Три сестрицы под окном\\*
        Пряли поздно вечерком.
        
        \vin <<Кабы я была царица, --- \\*
        \vin Молвит старшая сестрица. ---\\ 
        \vin Допьяна бы напилась\\*
        \vin И досыта {\myEbt{на}{ась}}!>>
        
        <<Кабы я была царица, ---\\*
        Молвит средняя сестрица. ---\\
        Я б {\myPiz{}{}} покрыла лаком\\*
        И давала только раком>>.
        
        \vin <<Кабы я была царица, ---\\*
        \vin Молвит младшая сестрица. ---\\
        \vin Ни кому бы не давала,\\*
        \vin На {\myPiz{}{}} печать наклала,\\
        \vin Умерла бы королевой\\*
        \vin И достойной старой девой!>>
        
        Только вымолвить успела,\\*
        Дверь тихонько заскрипела\\
        Во светлицу входит царь,\\*
        Стороны той государь.
        
        \vin <<Здравствуй, красная девица, ---\\*
        \vin Говорит он. --- будь царица.>>\\
        \vin {\myCel{}{у}} я давно искал,\\*
        \vin Ты и есть мой идеал!
        
        Царь не долго собирался,\\*
        Быстро в комнату забрался,\\
        Догола её раздел,\\*
        Сам как {\mySuk{}{}} озверел
        
        \vin Царь мужчина очень скромный,\\*
        \vin У него и {\myXyi{}{}} огромный,\\
        \vin Вот с такою то дубиной,\\*
        \vin Он и двинул на девчину.
        
        Тут она как заорет,\\*
        Да на помощь всех зовет:\\
        <<Ой, вы, сестры, помогите!\\*
        От царя меня спасите!\\
        Не хочу я быть царицей\\*
        А хочу я быть девицей!\\
        Если {\myXyi{}{}} ты свой воткнешь,\\*
        То насквозь меня проткнешь!>>
        
        \vin Отвечает царь ей: <<Врёшь!\\*
        \vin Ты от {\myXyi{}{}} не умрёшь!\\
        \vin Сколько девок я {\myEbt{}{}}\\*
        \vin И никто не умирал!\\
        \vin Захотела быть царицей,\\*
        \vin Так терпи, душа-девица!>>
        
        Царь наш очень удивился:\\*
        {\myXyi{}{}} как в бочку провалился!\\
        Доставать его не стал ---\\*
        Сам туда чуть не упал.
        
        \vin А она, {\myPiz{}{}}, хохочет,\\*
        \vin По {\myMnd{}{м}} его щекочет:\\
        \vin <<Я ведь только притворялась,\\*
        \vin Тыщу раз уже {\myEbt{}{сь}}!\\
        \vin И в {\myPiz{}{}} уже своей\\*
        \vin {\myXyi{}{}} видала здоровей!\\
        \vin Что? В диковинку тебе?\\*
        \vin Что инстр\'{у}мент не по мне?\\
        \vin Ты, мой милый, не шали!\\*
        \vin Влез на бабу, так {\myEbt{}{}}!>>
        
        Царь наш очень испугался,\\*
        На {\myPiz{}{}} у ней скончался.\\
        А она, душа-девица,\\*
        Стала в тот же день царица!
        
        \vin Говорит: <<Моя взяла!\\*
        \vin Всех сестер я {\myEbt{объ}{}}!>>
    \end{verse}
    \attrib{По мотивам сказки А.~С.~Пушкина.}
    
    \poemtitle{Раньше были времена\ldots}
    \settowidth{\versewidth}{И в лаптях всю жизнь ходили}
    \begin{verse}[\versewidth]
        В старину холсты носили\\*
        И в лаптях всю жизнь ходили\\
        И болезни рак не знали\\*
        Больше 100 лет проживали!\\
        \vin А теперь кримплен да шелк\\*
        \vin Как снимаешь --- треск да щёлк!\\
        \vin Стали модными ходить\\*
        \vin До пол ста лет стали жить.
        
        Раньше все здоровы были,\\*
        И в больниу не ходили.\\
        Порошков, лекарств не знали\\*
        И рентген не применяли.\\
        \vin А теперь в больницу ходят,\\*
        \vin Там себе и смерть находят.\\
        \vin Там диагноз ставая так:\\*
        \vin <<Свел его в могилу рак!>>
        
        Раньше суп да щи хлебали\\*
        По три пуда поднимали\\
        \vin А теперь рагу рубают\\*
        \vin Еле ноги поднимают
        
        Раньше ели редьку с квасом\\*
        И попёрдывали басом\\
        \vin А теперь рагу едят\\*
        \vin Не пердят, а еле бздят
        
        Раньше труд был весь вручную:\\*
        И косили и гребли.\\
        И, усталости не зная,\\*
        В ночь по 8 раз {\myEbt{}{}}!\\
        \vin А сейчас машиной только:\\*
        \vin Ей и косят, и гребут.\\
        \vin Бабы ночью воют волком:\\*
        \vin <<В 8 дней лишь раз {\myEbt{}{}}!>>
        
        Раньше были времена,\\*
        А теперь --- мгновения\ldots\\
        \vin Раньше поднимался {\myXyi{}{}},\\*
        \vin А теперь --- давление!
    \end{verse}    


    \poemtitle{Гром гремит, земля трясётся\ldots}
    \settowidth{\versewidth}{Гром гремит, земля трясётся}
    \begin{verse}[\versewidth]
        Гром гремит, земля трясётся,\\*
        Поп на курице несётся.\\
        Попадья идёт пешком,\\*
        {\myAss{}{у}} чешет гребешком.\\
        Вычесала вошку,\\*
        Позвала Тимошку.
        
        Тима вынул из порток\\*
        Трёхпудовый молоток.\\
        Как ударил по {\myPiz{}{}} ---\\*
        Раздалось по всей избе.
    \end{verse}    

    
    \poemtitle{Водовоз}
    \settowidth{\versewidth}{И загнал он мне такую сатану}
    \begin{verse}[\versewidth]
        Полубил меня Никита-водовоз.\\*
        Повалил меня Никита на навоз.\\
        \vin По навозу я каталася ---\\*
        \vin И Никите не давалася.\\
        А потом я согласилася ---\\*
        Под Никиту подкатилася:\\
        \vin И загнал он мне такую сатану,\\*
        \vin Что три года не давала никому!
    \end{verse}

    
    \chapter{Частушки}
    
    \section{Здровье}
    
    \settowidth{\versewidth}{Я пою и веселюсь}
    \begin{verse}[\versewidth]
        Я пою и веселюсь\\*
        Всем на удивление:\\*
        Очень пенсия мала---\\*
        меньше, чем давление.
        
        Не хожу теперь в аптеку,\\*
        Не хожу теперь к врачу!\\*
        Я теперь свои болезни\\*
        Только песнями лечу!
        
        Пошли плясать\\*
        Все порядочные ---\\*
        И косые, и хромые,\\*
        И припадочные!
    \end{verse}    

    
    \section{Клубничка}
    
    \settowidth{\versewidth}{Ты не жми меня к забору,}
    \begin{verse}[\versewidth]
        Ты не жми меня к забору,\\*
        Не таскай туды-сюды:\\*
        Я давно тебе сказала:\\*
        <<Когда женимся --- толды!>>
        
        Меня маменька род\'{и}ла,\\*
        Присмотрелась --- обмерла.\\*
        Я уже парням мигаю --- \\*
        И обратно загнал\'{а}!
        
        Я <<мотанечку>> плясать\\*
        Научилась просто:\\*
        Теперь замуж не пойду\\*
        Лет до 90!
        
        Я любила бригадира!\\*
        Ночь гуляла, день спал\'{а}\ldots\\*
        На работу не ходила\\*
        И стахановкой была!
        
        Мы с милёночком сидели,\\*
        Обнимались горячо:\\*
        Я ему сломала руку, \\*
        Он мне --- левое плечо!
        
        Я давала, всем  давала\ldots\\*
        И давать хотелося!\\*
        На такого нарвалась --- \\*
        Давалка разлетелася!
        
        Первобытный дроля мой,\\*
        Ты хоть рожу-то умой!\\*
        Кривы ноги, весь в шерсти --- \\*
        Тьфу ты, Господи прости!
        
        На веселое гулянье\\*
        Дроля не явился:\\*
        Шёл ко мне через болото ---\\*
        Клюквой подавился!
        
        Ноги долгие, худые\\*
        У миленка моего,\\*
        Сопли з\'{е}лены густые\ldots\\*
        Всё равно люблю его!
        
        На дворе собаки лают,\\*
        Во дворе петух поёт.\\*
        Бабка титьки потеряла,\\*
        Дед нашёл --- не отдает!
        
        Ты, соперница моя,\\*
        Какая ты худая.\\*
        У тя ноги колесом,\\*
        Да ещё глухая!
        
        Эх, девки, --- беда!\\*
        За кого я вышла?\\*
        Ни рубахи, ни трусов,\\*
        Хрен стоит, как дышло!
        
        Ваши девки --- сыроежки\\*
        Очень любят колбасу:\\*
        Я всегда их угощаю ---\\*
        Вместе с яйцами ношу.
        
        Юбка спала, юбка спала,\\*
        Кое-как напялила.\\*
        У меня никто не просит,\\*
        Кое-как навялила!
        
        Ты рябая, я рябой,\\*
        Поцелуемся с тобой!\\*
        Пусть народ любуется\\*
        Как р\'{я}бые целуются!
        
        Вот колечко золотое,\\*
        Золотое с пробою.\\*
        Если я не выйду замуж --- \\*
        Всё равно попробую!
        
        Мой милёночек дурак.\\*
        Не любит девушек никак.\\*
        Я советую ему\\*
        Любить бабушку мою.
        
        А я парень хоть-куда:\\*
        Денег---ничегошеньки,\\*
        Но зато 12 баб,\\*
        И все довольнёшеньки!
        
        Все ребята поженились,\\*
        А мне всё невесты нет\ldots\\*
        Оставался не женатым\\*
        До 70 лет!
        
        Наша улица большая:\\*
        Километра полтора.\\*
        Не любите, девки, Ваську ---\\*
        Он ломает трактора!
        
        На столе стоит стакан,\\*
        Во стакане --- татакан.\\*
        Я его --- за усики,\\*
        Он меня --- за трусики.
        
        Ты солома, ты солома,\\*
        Яровая, белая,\\*
        Ты не сказывай, солома,\\*
        Что я с милым делала!

        Шел я лесом, перелесом,\\*
        Видел чудо-чудеса:\\*
        Лежит баба на дороге,\\* 
        жопа шире колеса.

        Все папаши в поле пашут,\\*
        Моего папаши нет.\\*
        Мой папаша на мамаше\\* 
        Объезжает белый свет!

        Мы с миленочком сидели\\* 
        У большой магазины\ldots\\*
        Посидели, покурили,\\*
        Друг на друга слазили!

        Гармонист, гармонист,\\*
        Повали меня на низ.\\*
        А я сбоку погляжу,\\*
        Хорошо ли я лежу?!
        
        Жала, жала --- приустала\\*
        И легла не бережок.\\*
        Интересный сон приснился --- \\*
        Дроля смотрит пирожок!
        
        Бригадиру посулила,\\*
        Председателю дала\\*
        Счетоводу отказала --- \\*
        Больно шишечка мала!

        Зарубите петуха,\\*
        Натопите сала,\\*
        Напоите мужика\footnote{игрока},\\*
        Чтобы шишка встала!
        
        Огороды городили,\\*
        Я и кольев не втыкал\ldots\\*
        Аллименты присудили,\\*
        Я и девок не видал!
        
        Раньше были времена,\\*
        А теперь моменты\ldots\\*
        Даже кошка у кота\\*
        Просит аллименты!
        
        Ё-моё,\\*
        как я лазил на неё\ldots\\
        Как она вертелася,\\*
        Как ей не хотелося!
    \end{verse}    

    
    \section{О высоком и глубоком\ldots}
    
    \settowidth{\versewidth}{Что хотите говорите}
    \begin{verse}[\versewidth]
        Что хотите говорите,\\*
        Но по Вашему не быть!\\*
        До последнего дыхания,\\*
        Буду я его любить!
        
        Меня в армию забреют,\\*
        А жену мою куда?\\*
        Во дворе стоит колодец --- \\*
        Головой её туда!
        
        На суку сидит ворона,\\*
        Голова качается.\\*
        Полюбите кто-нибудь ---\\*
        Молодость кончается!
        
        Заиграла балалайка,\\*
        Я и встрепенулася!\\*
        <<Гондыбобером>> пошла\\*
        И не спотыкнулася!
        
        У миленка моего\\*
        Шляпа из велюра.\\*
        А под шляпой у него\\*
        Только шевелюра!
        
        У кого штаны худые?\\*
        Приходите, починю!\\*
        Я не дорого беру---\\*
        Шоколадку за дыру!
        
        Я гармошку люблю,\\*
        Гармониста пуще!\\*
        Я гармошку под кровать,\\*
        С гармонистом лягу спать!
        
        Мой милёнок х\'{и}тер, х\'{и}тер:\\*
        На козле уехал в Питер!\\
        Но я маху не дал\'{а} ---\\*
        На козлухе догнал\'{а}!
    \end{verse}    

    
    \section{Политика}
    
    \settowidth{\versewidth}{Наши Вятски парни хватски}
    \begin{verse}[\versewidth]
        Референдумы проводим,\\*
        Всё считаем голоса.\\*
        Цены выросли так быстро --- \\*
        Встали дыбом волоса.
        
        Наши Вятски парни хватски\\*
        Весь народ так говорит!\\*
        Вятский парень не уступит --- \\*
        Семерых один свал\'{и}т.
        
        Ой, деревня, ты деревня,\\*
        Захудалый уголок:\\*
        Кто-то старую кобылу\\*
        Из под носа уволок!
        
        По деревне мы проходим\\*
        99 раз.\\*
        Неужель и в сотый раз\\*
        Никто по морде нам не даст\footnote{Дублируются частушки. В некоторых: <<По башке никто не даст?!>>}?!
        
        Я иду, и заливается\\*
        В сандалии вода\ldots\\*
        Я на водку стары цены\\*
        Не забуду никогда!
        
        Что вы, граждане, глядите,\\*
        Что вы шеи тянете?\\*
        Мы Вахрушевские девчата!\\*
        Разве вы не знаете?!
        
        Ой ты тёща моя,\\*
        Налей рюмочку винца,\\*
        Каждый день я твою рожу\\*
        Провожаю до крыльца!
        
        Напоили пьяную,\\*
        Повалили в ямину,\\*
        Председатель наверху --- \\*
        Вот наделали см\'{е}ху!
        
        Повели меня на суд ---\\*
        Я иду трясуся\ldots\\*
        Присудили 100 яиц,\\*
        А я не несуся\ldots
        
        По деревенке пройдем\\*
        Шороху наделаем\\*
        Кому окна разобьём,\\*
        Кому ребенка сделаем!
        
        Папенька, купи коня,\\*
        Маменька, --- салазки.\\*
        Папенька, жени меня, ---\\*
        Больно девки баски!
        
        Мой милёнок бойкий очень,\\*
        Да ещё и деловой:\\*
        В одну ночку сделал дочку\\*
        С кудреватой головой!
        
        Я Гагарину дала,\\*
        И Титову хочется,\\*
        Потому, что у Титова\\*
        По земле волочится!
        
    \end{verse}    
    
    
    \section{Философия}
    \settowidth{\versewidth}{Парней в деревне не найдешь ---}
    \begin{verse}[\versewidth]
        Мой милёнок --- не теленок,\\*
        Ни корова, ни бычок\ldots\\*
        А надену серу шапку --- \\*
        Настоящий дурачок!
        
        Я отчаянна головушка:\\*
        Я ни чем не дорожу!\\*
        Если голову отрежут,\\*
        Я корчагу привяжу!
        
        Парней в деревне не найдешь ---\\*
        Ударли своевременно\ldots\\*
        А к телевизору прильнешь ---\\*
        Так вроде бы беременна\footnote{Поётся на мотив <<Парней так много холостых\ldots>>}\ldots
        
        Стоит домик на горе ---\\*
        Под горою мостик.\\*
        Едет милка на козе,\\*
        Держится за хвостик!
        
        У меня на сарафане\\*
        Петушок да курочка.\\*
        Меня замуж не берут---\\*
        Говорят, что дурочка!
        
        Эх, ботиночки мои,\\*
        Тупоносенькие,\\*
        Сколь милёнков не любила ---\\*
        Все курносенькие!
        
        Ты---цыган, и я---цыган:\\*
        Оба мы цыгане!\\*
        Ты---в карман, и я---в карман:\\*
        Оба за деньгами!

        Где бы милого увидеть,\\*
        Где бы с ним поговорить?\\*
        Столько горя\footnote{Счастья, денег, деток,\ldots Шучу} накопилось ---\\*
        Пополам бы разделить!
        
        В мире много есть болезней,\\*
        Но не надо унывать!\\*
        Ото всех одно лекарство:\\*
        Надо чаще выпивать\footnote{Хит}!

        %TODO
        %Ну какие ж мы старухи?\\*
        %Уцененна моложежь!
    
        До свиданья, дорогие,\\*
        Не болейте никогда!\\*
        И шампанского стаканчик\\*
        Можно выпить иногда.
        
        Вы пляшите мои туфли,\\*
        Вам не долго\footnote{трудно\ldots} поплясать!\\*
        Выйду замуж --- буду плакать!\\*
        Вам --- под лавочкой лежать!
        
        Научиться плясать ---\\*
        Не дрова в лесу рубить!\\*
        Надо корпусом работать\\*
        И ногами\footnote{мозгами\ldots} шевелить!
        
        Все бежала, все бежала, \\*
        Все бежала --- ничего\ldots\\*
        Вдруг, упала, запердела --- \\*
        Леший знает от чего!
        
        Не ругай меня, мамаша,\\*
        Не смотри так грозно.\\*
        Ты сама была такая ---\\*
        Приходила поздно!
        
        Там, там, за горам\\*
        Девка мается с родам\\*
        Окаянные ребята,\\*
        \vin больше разику не дам!
    \end{verse}    

    \section{Высшая \overarc{\emph{мат}}ематика}
    \settowidth{\versewidth}{Вся пизда обледенела, хуй стоял, как дед мороз}
    \begin{verse}[\versewidth]
        На осиновом мосту стоит баба раком:\\*
        Цементируют {\myPiz{}{}}, заливают шлаком!

        Мы {\myEbt{}{}} --- не боялись, в 40 градусов\footnote{$-40\,^{\circ}\mathrm{C}$} мороз:\\*
        Вся {\myPiz{}{}} обледенела, {\myXyi{}{}} стоял, как дед мороз!

        Шел я раз через солому, думал, мыши там пищат\ldots\\*
        Там {\myPiz{}{}} {\myPiz{}{у}} теребит --- только волосы трещат!

        Мне не надо <<жигули>> и машину <<волгу>>\\*
        Только был бы {\myXyi{}{}} большой и стоял подолгу!
        
        На окошке два цветочка: голубой да аленький:\\*
        Ни за что не променяю {\myXyi{}{}} большой на маленький!
        
        Бабка ночью\footnote{в оригинале <<бабка Нина>>} загрустила ---
            больно пенсия мала\ldots\\*
        {\myXyi{}{}} у деда отрубила --- на базаре продала!
        
        Раньше жили мы богато: много делали горшков.\\*
        Тятя умер, {\myXyi{}{}} оставил восемнадцати вершков!
        
        Я у ш\'{о}фера в кабиночке уснула на руке.\\*
        Я проснулась --- {\myCel{}{}} сломана, и трёшенка в руке!
        
        По реке плывёт топор из деревни Чуева\footnote{Из села Кукуева}\\*
        Ну и пусть себе плывет, железяка {\myXyi{}{ва}}!
        
        По дер\'{е}веньке идем,\\*
        Ни к кому не пристаём.\\
        Все мы председатели ---\\*
        Катись к {\myEbt{}{не}} матери!
        
        Мою милку ранили,\\*
        У страны Германии:\\*
        Вместо пули\\*
        \vin {\myXyi{}{}} воткнули\\*
        И в лазарет отправили!

        На Кавказе есть гора --- в небо упирается,\\*
        А в аптеке есть резина --- на {\myXyi{}{}} одевается!
        
        Ах, мать твою {\myEbt{}{и}} --- сделали ребёнка:\\*
        Три руки, четыре {\myXyi{}{я}} и одна {\myPiz{}{ёнка}}!
        
        Скоро-скоро лёд растает,\\*
        С гор покатится вода.\\*
        Скоро-скоро {\myXyi{}{}} состанет,\\*
        И расширится {\myPiz{}{}}.
        
        Деньги есть, так девки любят:\\*
        По середке спать кладут!\\*
        Денег нет, так {\myXyi{}{}} отрубят\\*
        И собакам отдадут!
        
        Шёл я по тропиночке,\\*
        Нашел {\myPiz{}{у}} в корзиночке,\\
        Она чирикает, поёт ---\\*
        Мне покою не даёт!
        
        Финский нож, советской стали\\*
        \vin Я на кладбище нашел:\\*
        {\mySuk{}{и}} режьте, {\mySuk{}{и}} бейте,\\*
        \vin Помирать сюда пришел!

        Трактор сеет, трактор пашет,\\*
        \vin Скоро трактор будет жать.\\*
        Скоро трактор приспособят ---\\*
        \vin Будет девушек {\myEbt{}{ть}}!
        
        Говорит старуха деду:\\*
        <<Я в Америку поеду!>>\\
        Что ты, старая {\myPiz{}{}},\\*
        Туда не ходят поезда!
        
        Валентине Терешковой\\* 
        За полёт космический\\*
        Подарил Хрущёв Никита\\*
        {\myXyi{}{}} автоматический!

        Мяса нет, муки не стало,\\*
        Рыбу скоро всю съедим\ldots\\*
        Баб своих {\myEbt{}{ть}} не станем ---\\*
        {\myXyi{}{}} Вьетнаму отдадим!
        
        Девки {\myBld{}{и}}, девки {\myBld{}{и}},\\*
        Оторвали {\myXyi{}{}} у дяди!!!\\
        Дядя плачет и ревёт\\*
        {\myXyi{}{}} под пазухой несёт!
        
        Старики, старики,\\*
        старые вы черти!\\*
        Нажились, {\myEbt{на}{сь}} -- \\*
        дожидайтесь смерти!
        
        Милый в горку --- я в догонку,\\*
        \vin Думаю, воротится\ldots\\*
        У него на {\myAss{}{е}} чирей ---\\*
        \vin Ко врачу торопится!
        
        Девки в озере купались\\*
        \vin Увидали рака\\*
        Долго думали-гадали:\\*
        \vin <<Где у рака {\myAss{}{}}?!>>
        
        Полюбила я пилота\ldots\\*
        Он, зараза, улетел,\\*
        {\myAss{}{у}} свесил с самолёта\\*
        Обо{\myAss{}{}}ть меня хотел!
        
        Девки в город ездили\\*
        {\myXyi{}{}} с витрины\footnote{Окошка\ldots} {\myPiz{с}{ли}}\\*
        Идут девушки на суд\\*
        {\myXyi{}{}} за пазухой несут!
        
        Эх, трактор идёт,\\*
        \vin Керосином пахнет.\\*
        Скоро милый придёт,\\*
        \vin Через {\myAss{}{}} {\myEbt{тр}{т}}!
    \end{verse}
    
    
    \chapter{Проза}

    \section{Цыганское гадание}
    
    Позолоти ручку, дорогая, всю сущую правду расскажу. Не скупись, дорогая, душа у тебя добрая, а сама скупая. Слушай, всю правду расскажу. Будет тебе скоро переметная жизнь. Ходишь ты по крутой горе, ищешь ты счастья, да никак найти не можешь. Будет скоро у тебя три досады. Первая досада: потеряешь штаны у сада. Вторая досада: или ты в бане запаришься, или же {\myXyi{}{м}} подавишься. Треться досада: или тебя бык забодет, или мужик {\myEbt{за}{т}}.
    
    \section{Мужские критерии}
    \begin{itemize}
        \item От 18--25 --- <<кларнет>>. Играет без разбора, кто понимает в музыке, тому не нравится.
        \item От 25--35 --- <<скрипка>>. Самый лучший музыкальный инструмент, но играет по желанию.
        \item От 35--45 --- <<рояль>>. Два часа настраивается, пять минут играет.
        \item От 45--55 --- <<трамбон>>. Пугает, но не играет.
    \end{itemize}
\end{document}


