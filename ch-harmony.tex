\chapter{А так звучит красиво? Гармоничность}
\label{ch:harmony}


Подчиняется ли наше восприятие музыки математическим законам? Что определяет наши вкусы:
\begin{itemize}
    \item законы физики;
    \item индивидуальные особенности личности;
    \item сформировавшиеся за многие годы общие культурные привычки;
    \item что-то еще?
\end{itemize}

В общем и целом --- да, подчиняется. Как уже было сказано --- математика внутри нас. Тому, что нам нравится, а тем более тому, что не нравится, есть естественное объяснение. Наши предки нашли и оставили нам многие законы гармонии, в соответствии с которыми музыкальные гении создавали и создают музыку. Нельзя сбрасывать со счетов особенности психики отдельного человека. Да, к определенной музыке мы просто привыкли.

И конечно же есть что-то еще, пока непознанное, что всегда удобно назвать словом \emph{чудо}!


%\section{Сколько вешать в полутонах? Интервалы}
\label{ch:harmony:interval}

\begin{Definition}[Интервал]
    \emph{Интервал} --- это расстояние между \emph{двумя} музыкальными звуками, выраженное в полутонах. 
\end{Definition}

Это определение для математиков. Для музыкантов интервалы \emph{звучат}! Если два звука прозвучали одновременно, то интервал называется \emph{гармоническим}, а если друг за другом --- \emph{мелодическим}.

\begin{Example}[Послушаем гармонические интервалы]
    \label{ex:harmony:interval:string5and6}
    Возьмите настроенную гитару. Поиграем на 5 и 6-й струне. Если гитара настроена страндартно, то 6-я струна на 5-м ладу прозвучит в унисон с открытой 5-й струной. Такое расстояние в 0 полутонов музыканты называют \emph{примой}. 
    
    Инструмент обязательно должен быть настроен! Иначе эксперимент не получится.
    
    Теперь расслабьтесь, успокойтесь, забудьте обо всех горестях и радостях. Сосредоточьтесь на своем дыхании. Существует только ваше дыхание. Абсолютный покой. 
    
    Не получается? Ну и чёрт с ним!

    Вам нужно будет оценить свои ощущения от сыгранных интервалов. Поставьте каждому интервалу оценку, например по 5-и балльной шкале: 1(ужос), 2(срам), 3(терпимо), 4(хорошо), 5(прекрасно).
    
    Начнем. Зажимите 6-ю струну на 5-м ладу и одновременно сыграйте две струны: 6-ю и 5-ю. Звучит гармонический интервал \emph{прима}! Интервал в ноль полутонов. Прислушиваемся к ощущениям, ставим приме оценку.
    
    Продолжаем оценивать интервалы. Играем одновременно 5-ю открытую струну и 6-ю струну на 6-м ладу. Расстояние в 1 полутон. Оценивайте результат.
    
    И так далее, играем интервал в 2 полутона (6-я струна на 7-м ладу) и так далее до интервала в 12 полутонов (6-я струна на 17 ладу). 
    
    Читая этот раздел, периодически поглядывайте на свои записи. Будет полезно.
\end{Example}

Приятный на слух интервал (неважно, гармонический или мелодический) музыканты называют \emph{консонансом}, а неприятный --- \emph{диссонансом}. 

Чтобы глубже разобраться в том, почему звучание одного интервала нам нравится, а другого --- нет, визуализируем результат наложения двух звуков. Пусть функция $\sin(x)$ изображает основной тон исходного звука. Функция $\sin(x\cdot(\sqrt[12]{2})^n)$ будет изображать звук, который \emph{выше} исходного на $n$ \emph{полутонов}. Результату совместного звучания (то есть гармоническому интервалу) будет соответствовать их сумма\footnote{В Интернете масса сайтов, позволяющих построить график функции. Более того, просто вбейте в гугл \texttt{sin(x)+sin(x*(2\^{}(1/12)))} и дивитесь чудесам Технологии!}:

\begin{equation}
    \label{eq:harmony:interval:sin}
    \sin(x) + \sin(x\cdot(\sqrt[12]{2})^n).
\end{equation}

Результаты построения графиков совместного звучания (построен красным цветом) на фоне исходного звука (синий цвет) для интервалов от $n=1$ до $n=11$ полутонов (т.е. в рамках октавы) приведены в таблицах \ref{t:harmony:interval:disso-1-2-10-11}, \ref{t:harmony:interval:conso-3-4-8-9}, \ref{t:harmony:interval:conso-5-7}, \ref{t:harmony:interval:disso-6}.

Диссонансы приведены в таблицах \ref{t:harmony:interval:disso-1-2-10-11} и \ref{t:harmony:interval:disso-6}. Диссонансы получаются на интервалах в 1,2,6,10 или 11 полутонов. Значит вам не должны были понравиться интервалы на 6,7,11,15,16 ладах 6-й струны, если вы обратили внимание на пример \ref{ex:harmony:interval:string5and6}. Понравились?! Пора сходить к доктору. Не затягивайте! 

\begin{table}[!ht]
    \caption{Интервалы-диссонансы в графиках \eqref{eq:harmony:interval:sin}}
    \label{t:harmony:interval:disso-1-2-10-11}
    \centering
    \begin{tabular}{c|c}
        \hline\hline
        1 полутон, $n=1$        & 2 полутона, $n=2$ \\
        малая секунда           & большая секунда \\
        \includegraphics[width=0.45\textwidth]{fig/intervals/i01}
            & \includegraphics[width=0.45\textwidth]{fig/intervals/i02} \\
        \hline\hline
        10 полутонов, $n=10$    & 11 полутонов, $n=11$ \\
        малая септима           & большая септима \\
        \includegraphics[width=0.45\textwidth]{fig/intervals/i10}
            & \includegraphics[width=0.45\textwidth]{fig/intervals/i11} \\
        \hline\hline
    \end{tabular}
\end{table}

Приглядитесь к приведенным графикам и попробуйте самостоятельно сделать выводы о причинах благозвучия консонансов и некоей хаотичности диссонансов. Консонансы, кстати, принято разделять на:
\begin{itemize}
    \item \emph{абсолютные}. Это полностью сливающиеся на слух интервалы в 0 полутонов (\emph{прима}, если помните) или кратные 12-ти полутонам. О причинах слияния этих звуков мы поговорили в самом начале, см. раздел \ref{ch:music:tone}. Интервал в 12 полутонов называется \emph{октавой}.
    
    \item \emph{совершенные}. Расстояние между звуками составляет 5 или 7 полутонов. См. таблицу \ref{t:harmony:interval:conso-5-7}. Смело можете понизить или повысить любой из звуков такого интервала на одну или несколько октав (12 полутонов) и совершенный консонанс останется\footnote{Уже заметили, что 5+7=12?}. Эти интервалы не сливаются на слух, но звучат благозвучно. Отличник среди консонансов.
    
    \item \emph{несовершенные}. Звуки не сливаются, точно не диссонанс, но и не совершенный консонанс. Короче, консонанс с помарочкой. Когда мы слышим несовершенный консонанс, то хочется, чтобы он побыстрее перешел консонанс совершенный, стал отличником. Разница между звуками составляет 3, 4, 8 или 9 полутонов. Точно так же, любой звук такого интервала можно понизить или повысить на октаву и несовершенство останется.
\end{itemize}


\begin{table}[!ht]
    \caption{Несовершенные консонансы в графиках \eqref{eq:harmony:interval:sin}}
    \label{t:harmony:interval:conso-3-4-8-9}
    \centering
    \begin{tabular}{c|c}
        \hline\hline
        3 полутона, $n=3$   & 4 полутона, $n=4$ \\
        малая терция        & большая терция \\
        \includegraphics[width=0.45\textwidth]{fig/intervals/i03}
            & \includegraphics[width=0.45\textwidth]{fig/intervals/i04} \\
        \hline\hline
        8 полутонов, $n=8$  & 9 полутонов, $n=9$ \\
        малая секста        & большая секста \\
        \includegraphics[width=0.45\textwidth]{fig/intervals/i08}
            & \includegraphics[width=0.45\textwidth]{fig/intervals/i09} \\
        \hline\hline
    \end{tabular}
\end{table}

\begin{table}[!ht]
    \caption{Совершенные консонансы в графиках \eqref{eq:harmony:interval:sin}}
    \label{t:harmony:interval:conso-5-7}
    \centering
    \begin{tabular}{c|c}
        \hline\hline
        5 полутонов, $n=5$  & 7 полутонов, $n=7$ \\
        чистая кварта       & чистая квинта \\
        \includegraphics[width=0.45\textwidth]{fig/intervals/i05} 
            & \includegraphics[width=0.45\textwidth]{fig/intervals/i07} \\
        \hline\hline
    \end{tabular}
\end{table}

\begin{table}[!ht]
    \caption{Диссонанс в графике функции \eqref{eq:harmony:interval:sin}}
    \label{t:harmony:interval:disso-6}
    \centering
    \begin{tabular}{c}
        \hline\hline
        6 полутонов, $n=6$ \\
        \includegraphics[width=0.45\textwidth]{fig/intervals/i06} \\
        увеличенная кварта,\\
        она же --- уменьшенная квинта,\\
        он же --- тритон\\
        \hline\hline
    \end{tabular}
\end{table}

Звуки октавы удобно зациклить и изобразить на окружности. На рисунке \ref{fig:harmony:interval:oct-round} изображены ноты и отмечены консонансы и диссонансы от ДО (C --- ноты обозначены в латинской нотации). 

\begin{figure}[!ht]
    \centering
    \includegraphics[width=\textwidth]{fig/intervals/octave-round} 
    \caption{Интервалы от ноты ДО}\label{fig:harmony:interval:oct-round}
\end{figure} 

Нужно иметь в виду, что консонансы и диссонансы определяются для любой ноты, поэтому можно не полениться и сделать из картона приборчик, изображенный на рисунке \ref{fig:harmony:interval:octave-kon-dis}: два кружка белый и серый, свободно вращаются на общей оси, проходящей через их центры. Достаточно совместить название ноты на белом кружке со стрелкой на сером и вы узнаете, в каких отношениях указанная нота находится со всеми остальными.

\begin{figure}[!ht]
    \centering
    \includegraphics[scale=0.7]{fig/intervals/octave-kon-dis} 
    \caption{Гармонометр}\label{fig:harmony:interval:octave-kon-dis}
\end{figure} 

На практических занятиях с гитарой очень полезной штукой может оказаться <<квинто-квартовый круг мажорных и минорных последовательностей>> (см. рисунок \ref{fig:harmony:kvinto-kvarto:kvinto-kvarto-final}). Название ни о чем не говорит? Пока и не должно, просто раз уж занялись рукоделием, то сделайте и его --- пригодится.

Осталось разобраться какого лешего интервалы называются так странно? Нет никакой видимой связи между названиями интервала и количеством полутонов, его составляющих! Справочник\footnote{Для тех, кому зубрежка покажется проще понимания} по интервалам см. в таблице \ref{t:harmony:interval:names}. Собственно, ответ прост: по историческим причинам название интервала отражало не количество полутонов, а номер ступени мажорного музыкального лада (о ладах см. раздел \ref{ch:harmony:lad}). Так как каждая ступенька мажорного лада состояла из одного или двух полутонов, то определить количество полутонов по названию интервала без достаточного опыта затруднительно, если не представить в уме рисунок \ref{fig:harmony:interval:names}.

\begin{table}[!ht]
    \caption{Интервалы}
    \label{t:harmony:interval:names}
    \centering
    \begin{tabular}{l|l|l|c|l}
        \hline\hline
        Название интервала & Перевод            &               & Количество  & Кратко  \\
                           & на русский         &               & полутонов   &         \\
        \hline\hline
        Прима(prima)       & Первая (ступень)   & Чистая        & 0                 & ч.1 \\
        Секунда(secunda)   & Вторая             & Малая         & 1                 & м.2 \\
                           &                    & Большая       & 2                 & б.2 \\
        Терция(tertia)     & Третья             & Малая         & 3                 & м.3 \\
                           &                    & Большая       & 4                 & б.3 \\
        Кварта(quarta)     & Четвертая          & Чистая        & 5                 & ч.4 \\
                           &                    & Увеличенная   & 6                 & ув.4\\
        Квинта(quinta)     & Пятая              & Уменьшенная   & 6                 & ум.5\\
                           &                    & Чистая        & 7                 & ч.5 \\
        Секста(sexta)      & Шестая             & Малая         & 8                 & м.6 \\
                           &                    & Большая       & 9                 & б.6 \\
        Септима(septima)   & Седьмая            & Малая         & 10                & м.7 \\
                           &                    & Большая       & 11                & б.7 \\
        Октава(octava)     & Восьмая            & Чистая        & 12                & ч.8 \\
        \hline\hline
        Нона(nona)         & Девятая            & Малая         & 13                & м.9  \\
                           &                    & Большая       & 14                & б.9  \\
        Децима(decima)     & Десятая            & Малая         & 15                & м.10 \\
                           &                    & Большая       & 16                & б.10 \\
        Ундецима           & Одиннадцатая       & Чистая        & 17                & ч.11 \\
                           &                    & Увеличенная   & 18                & ув.11\\
        Дуодецима          & Двенадцатая        & Уменьшенная   & 18                & ум.12\\
                           &                    & Чистая        & 19                & ч.12 \\
        Терцдецима         & Тринадцатая        & Малая         & 20                & м.13 \\
                           &                    & Большая       & 21                & б.13 \\
        Квартдецима        & Четырнадцатая      & Малая         & 22                & м.14 \\
                           &                    & Большая       & 23                & б.14 \\
        Квинтдецима        & Пятнадцатая        & Чистая        & 24                & ч.15 \\
        \hline\hline
    \end{tabular}
\end{table}

\begin{figure}[!ht]
    \centering
    \includegraphics[width=\textwidth]{fig/intervals/interval-names} 
    \caption{Исторически имена интервалов --- это имена ступеней мажорного лада}\label{fig:harmony:interval:names}
\end{figure} 

Тогда все становится относительно просто. Например, терция, это расстояние от ноты ДО (первая ступень мажорного лада) до ноты МИ (третья ступень). Считаем ДО-РЕ --- 2 полутона, РЕ-МИ --- 2-а полутона. Получилось 4-е полутона. Только вот вспоминается, что терция бывает <<большая>> и <<малая>>. При таком подходе мы всегда будем получать значение для <<большого>> и <<чистого>> интервалов. Для <<малого>> или <<уменьшенного>> интервалов нужно уменьшить получившееся число на 1, а для <<увеличенного>> --- увеличить на 1. Значит: большая терция --- 4 полутона, малая --- 3.

Закрепим. Например, квинта: расстояние ДО-СОЛЬ --- 7 полутонов. Уменьшенная квинта --- 6 полутонов, чистая --- 7.

%\section{Лады? Лады}
\label{ch:harmony:lad}

\begin{Definition}[Лад]
    \emph{Лад}\footnote{На английском \emph{лад} --- \emph{mode}. Режим работы, способ, вид, метод} --- это интервальный шаблон, позволяющий из 12-и последовательных музыкальных звуков октавы выбрать \emph{условно} <<правильные>>. 
\end{Definition}

Если это определение показалось вам тяжеловатым, почитайте учебники или Википедию. От некоторых определений веет такой суровой философией, что хочется курить в глубокий затяг. 

Задача лада: из 12 музыкальных звуков, составляющих октаву, выбрать лишь несколько таких, которые можно играть в любом порядке и все равно будет МУЗЫКА! Задача не из тривиальных и кажется весьма субъективной, ведь всегда найдется кто-то, кто скажет: <<А мне не нравится!>>. 

Однако эта задача была решена\footnote{Не исключено, что кем-то она решается и в данный момент} предками неоднократно, и в культурном наследнии мы имеем немало ладов, самыми известными из которых являются \emph{мажорный} и \emph{минорный}.

Мажорный и минорный лады --- лады \emph{семиступенные}. То есть такой лад выбирает из 12 звуков октавы только 7.

\paragraph{Мажорный лад.} Начнем с мажорного лада, интервальная структура которого приведена на рисунке \ref{fig:harmony:lad:mode:maj}. Тёмными кружками обозначены <<выбранные>> ладом звуки --- \emph{ступени} лада. Например, вторая ступень мажорного лада находится на расстоянии 2-х полутонов от первой. Интервалы (в полутонах) между ступенями \emph{мажорного} лада расположены так:

\[
    \texttt{2-2-1-2-2-2-1}
\]

Всем с детства знакомое ДО, РЕ, МИ, ФА, СОЛЬ, ЛЯ, СИ есть не что иное, как 7 идеальных ноток, отобранных мажорным ладом, начиная от ноты ДО. Проверьте: (ДО-РЕ)=2 полутона, (РЕ-МИ)=2, (МИ-ФА)=1, (ФА-СОЛЬ)=2 и т.д. 

Интересно то, что уникальные имена получили только 7 нот, а остальные 5 нот октавы имеют производные имена (с суффиксом <<бемоль>> или <<диез>>) --- есть следствие использования ладов.

\begin{figure}[!ht]
    \centering
    \includegraphics{fig/intervals/mode-maj} 
    \caption{Интервальная структура мажорного лада}\label{fig:harmony:lad:mode:maj}
\end{figure} 

Таким образом, шаблон лада может накладываться на любую ноту, любой музыкальный звук. Сплошная теория относительности!

Когда первая ступень лада накладывается на определенную ноту, то набор нот, попавших на ступени лада, образует \emph{тональность}\footnote{На английском \emph{тональность} --- tonality}. Допустим, мы совместили первую ступень мажорного лада с нотой ДО, тогда мы получим тональность <<ДО-мажор>>: 
\[
    \text{ДО}\xrightarrow{2}
    \text{РЕ}\xrightarrow{2}
    \text{МИ}\xrightarrow{1}
    \text{ФА}\xrightarrow{2}
    \text{СОЛЬ}\xrightarrow{2}
    \text{ЛЯ}\xrightarrow{2}
    \text{СИ}\xrightarrow{1}
\]

Базовая нота, т.е. нота, на которую наложили перую ступень лада, называется \emph{тоникой}\footnote{Ступени мажорного и минорного ладов так часто используются в теории музыки, что получили собственные названия. Нам, чтобы разобраться, достаточно запомнить, что нота, попавшая в первую ступень называется \emph{тоника}. А для общего развития: 5-я ступень --- доминанта, 4-я --- субдоминанта, 3-я --- медианта. Повторюсь: это названия ступеней как мажорного, так и минорного ладов}.

Название тональности складывается из названия ноты, попавшей на первую ступень (\emph{тоники}) и названия лада. Обычно мелодия составляется только из семи нот, входящих в тональность. Так и говорят, например, мелодия в тональности <<ЛЯ-минор>>.

\begin{Example}[Тональность <<РЕ-мажор>>]
    \label{ex:harmony:lad:d:maj}
    
    Чтобы получить ноты в тональности РЕ-мажор, нам нужно совместить ноту РЕ и первую ступень мажорного лада. Отступаем два полутона, и на вторую ступень попадет нота МИ. На третью --- ФА-диез.
    
    Целиком:
    \[
        \text{РЕ}\xrightarrow{2} 
        \text{МИ}\xrightarrow{2} 
        \text{ФА-диез}\xrightarrow{1} 
        \text{СОЛЬ}\xrightarrow{2} 
        \text{ЛЯ}\xrightarrow{2} 
        \text{СИ}\xrightarrow{2} 
        \text{ДО-диез}\xrightarrow{1}
    \]
    
    В эту тональность попали нотки, имеющие производные названия: ФА-диез, ДО-диез.
\end{Example}

Задача определить ноты, входящие в ту или иную тональность, а также количество диезов и бемолей, является любимой пыткой среди музыкальных инквизиторов. Сдвинуть шаблончик --- дело плёвое. А вот ноты после этого назвать --- уже подвиг! Совершенно искусственная проблема, растущая только от принятого способа обозначать ноты.

Например, для певца, поющего по нотам\footnote{Да, есть люди которые могут делать такие штуки со своим голосом: тянуть гласные с нужной частотой основного тона} чтобы перейти из тональности <<ДО-мажор>> в <<РЕ-мажор>> достаточно каждую исходную нотку спеть двумя полутонами выше (сдвинуть шаблон) и не думать о том, какая нота получается в итоге (закодировать название ноты).

Короче, с практической вещи все проще, чем с теоретической!

Чтобы сыграть \emph{гамму}\footnote{Слово \emph{гамма} в русском очень похоже на \emph{Game} (игра) в английском. И вроде бы логично: гамма --- это то, что \emph{играется}! Но \emph{гамма} на английском --- \emph{scale}. Шкала, звукоряд} в заданной тональности нужно:
\begin{itemize}
    \item начать с ноты первой ступени;
    \item продолжить играть ноты тональности в порядке возрастания (или убывания) высоты;
    \item сыграв таким образом одну или несколько октав, закончить на ноте первой ступени (естественно уже в другой октаве); 
    \item (необязательно) проиграть только что сыгранную последовательность в обратном порядке.
\end{itemize}

Например, гамма в тональности <<ДО-мажор>> или просто <<гамма ДО-мажор>> это известное: 
\begin{center}
    ДО, РЕ, МИ, ФА, СОЛЬ, ЛЯ, СИ, ДО, СИ, ЛЯ, СОЛЬ, ФА, МИ, РЕ, ДО.
\end{center}


\paragraph{Минорный лад.} Интервальная структура \emph{минорного} лада приведена на рисунке \ref{fig:harmony:lad:mode:min}. Интервалы (в полутонах) между ступенями \emph{минорного} лада расположены так:
\[
    \texttt{2-1-2-2-1-2-2}
\]

\begin{figure}[!ht]
    \centering
    \includegraphics{fig/intervals/mode-min} 
    \caption{Интервальная структура минорного лада}\label{fig:harmony:lad:mode:min}
\end{figure} 

Заметьте, что если замкнуть минорную интервальную структуру в кольцо и немного повращать (а это можно сделать, так как в следующей октаве все повторится), то получится мажорный лад. Совместите первую ступень мажорного лада и третью минорного и убедитесь, что в принципе структура этих ладов одна и та же. 

Например, давайте положим в первую ступень минора ноту ЛЯ. Получим тональность, состоящую из нот:
\begin{center}
    ЛЯ, СИ, ДО, РЕ, МИ, ФА, СОЛЬ.
\end{center}

Ноты в тональности <<ЛЯ-минор>> те же, что и в <<ДО-мажор>> (как видно ни одной нотки с бемолем или диезом). Поэтому тональности <<ДО-мажор>> и <<ЛЯ-минор>> называются \emph{параллельными}. Как нетрудно догадаться, параллельных тональностей столько же, сколько фактических нот в октаве: 12. А вот гамму <<ЛЯ-минор>>:
\begin{center}
    ЛЯ, СИ, ДО, РЕ, МИ, ФА, СОЛЬ, ЛЯ, СОЛЬ, ФА, МИ, РЕ, ДО, СИ, ЛЯ
\end{center}
с гаммой <<ДО-мажор>> точно на слух не спутаешь!

\paragraph{Современные 7-ступенные лады.} Эти лады имеют сходную интервальную структуру и называются диатоническими\footnote{Диатонические лады или просто <<диатоника>> --- это система семиступенных ладов, постоенных из пяти интервалов величиной в два полутона, и двух полутоновых интервалов. $5\cdot2 + 2\cdot 1 = 12$ --- октава}. Мажор и минор --- также диатонические лады. На рисунке \ref{fig:harmony:lad:modes} иображена октава, разделенная на 12 полутонов. Лады отличаются друг от друга только тем, откуда начинается первая ступень. Каждый из семи возможных вариантов имеет собственное название. Например, первая ступень <<Лидийского>> лада начинается с 4-й отметки на октаве, и, обойдя от 4-й отметки всю окраву, легко получить его интервальную структуру:
\[
    \texttt{2-2-2-1-2-2-1}
\]

\begin{figure}[!ht]
    \centering
    \includegraphics[width=\textwidth]{fig/intervals/modes} 
    \caption{Интервальная структура диатонических ладов}\label{fig:harmony:lad:modes}
\end{figure} 

<<Лидийский>> лад активно используется в джазовой музыке. Если вы хотите сыграть <<ФА-лидийскую>> гамму, то достаточно проиграть октаву:
\begin{center}
    ФА, СОЛЬ, ЛЯ, СИ, ДО, РЕ, МИ, ФА, МИ, РЕ, ДО, СИ, ЛЯ, СОЛЬ, ФА
\end{center}
 
построенная от ноты ФА(F), <<ФА-лидийская>> тональность содержит (как видно из рисунка \ref{fig:harmony:lad:modes}) ноты без диезов и бемолей. Кстати, этот лад для мелодий, дарящих ощущение счастья.


\paragraph{Пентатоника.} Пентатоника --- это тоже лад, но имеющий только 5-ступеней. То есть пентатоника из 12 нот октавы выделяет только 5 <<правильных>>, из которых можно составлять мелодию. Аналогично 7-ступенным ладам, получают 5 вариантов ладов с различными названиями, см. рисунок \ref{fig:harmony:lad:pentatonic}.

\begin{figure}[!ht]
    \centering
    \includegraphics[width=\textwidth]{fig/intervals/pentatonic} 
    \caption{Интервальная структура пентатоники}\label{fig:harmony:lad:pentatonic}
\end{figure} 

Соответственно, например, интервальная структура мажорной пентатоники в полутонах:
\[
    \texttt{2-2-3-2-3}
\]

Обратите внимание, что пять ступеней пентатоники полностью содержатся в семиступенной структуре. Сравните рисунки \ref{fig:harmony:lad:pentatonic} и \ref{fig:harmony:lad:modes} и убедитесь, что пентатонику из диатоники можно получить, <<выкинув>> 4-ю и 7-ю отметки на октаве диатонических ладов.


%TODO: гамма лада в основных нотах

%\section{Гаммы? Импровизация в жестких рамках}
\label{ch:harmony:scales}

Мы уже разобрались с тем, что такое лад, тональность и гамма в разделе \ref{ch:harmony:lad}. Вкратце, гамма\index{гамма} --- это последовательно сыгранные ступени лада, отложенного от <<базовой>> ноты (тоники). Или, например, гамма --- это последовательно сыгранные ступени тональности.

Зачем играть гаммы?
\begin{itemize}
    \item Гамма приятно звучит, никакого дискомфорта, так почему бы и не сыграть?
    \item Начинающим полезно играть гаммы, проговаривая входящие ноты вслух, чтобы лучше запомнить их положение на грифе гитары.
    \item Гаммы --- это неплохая тренировка для пальцев. Даже профи, знающие множество простеньких пьес, чтобы разыграться перед выступлением, не гнушаются гаммами.
    \item Так как гамма --- это конкретный вариант \emph{лада}, то в каком порядке не играй ноты гаммы --- будет МУЗЫКА. Расслабьтесь, комбинируйте, играйте с длительностью, акцентами, импровизируйте. В конце концов музыка должна приносить удовольствие даже на этапе обучения!
\end{itemize}

Итак, ноты нотами, а играть-то нужно ручками. Поэтому на первых этапах придется потратить время на то, чтобы соотнести ноты с постановкой и движениями рук. Для гитары вопрос нот сводится к постановке пальцев левой\footnote{Если левши внимательно читали примечания, то они знают, что автор надеется, что они знают, что делать} руки на грифе. 

\begin{Definition}[Аппликатура]
    \emph{Аппликатура}\index{аппликатура}\footnote{От латинского applico --- прикладываю, прижимаю} --- порядок расположения и чередования пальцев при игре на музыкальном инструменте.
\end{Definition}

Чтобы задать аппликатуру, гитаристы обычно пользуются изображением участка грифа, на котором точками отмечены места прижатия струн, а при необходимости возле точки указан и номер прижимающего пальца левой руки. Такой рисунок называется \emph{аппликатурным боксом}\index{бокс аппликатурный}. Пальцы левой руки принято нумеровать слеюдующим образом:
\begin{itemize}
    \item указательный --- 1;
    \item средний --- 2;
    \item безымянный --- 3;
    \item мизинец --- 4.
\end{itemize}

Большой палец левой руки не нумеруется, ибо находится с тыльной стороны грифа и оказывает только моральную поддержку остальным пальчикам, имеющим дело со струнами.

Итак, нам уже известно, что только диатонических ладов (включающих мажор и минор) семь штук, пентатоник --- пять. Так что имеется большой выбор в каком ладу поиграть. А о тональностях и говорить нечего --- смело умножайте количество ладов на 12!

Давайте ограничимся только гаммой ДО-мажор\index{гамма!ДО-мажор}. При желании с остальными вы разоберетесь по аналогии. Не надо думать, что нужно уметь играть гаммы во всех ладах и тональностях. Вы же не робот! А если робот, то вот неплохое чтиво\footnote{Признанным специалистом в области гамм является Андреас Сеговия, чьи гаммы играет не первое поколение классических гитаристов. Открыв его книжку \cite{bib:segovia:Scales} можно увидеть 9 страниц нотной записи гамм (с аппликатурными пометками) во всех тональностях мажорного и минорного ладов. Эта книжка --- справочник, методичка, её с ужасом открывает на нужной странице бедный ученик музыкальной школы, терзает заданную учителем гамму, и с ужасом закрывает обратно, чтобы навсегда о ней забыть. Через несколько десятков лет, ставший профессионалом ученик, возможно и откроет книжку Сеговии, чтобы подглядеть как ставил на струны свои гениальные пальцы маэстро. Из любопытства. И только потому, что сам ужасно много знает}: \cite{bib:segovia:Scales} справочник от гениального гитариста и педагога Андреаса Сеговии.

\begin{Example}[Гамма ДО-мажор на одной струне]
    \label{ex:harmony:scales:d:maj}
    
    Нота ДО находится на первом ладу второй струны\footnote{Стандартный, МИ-СИ-СОЛЬ-РЕ-ЛЯ-МИ, строй}. Последовательно зажимайте на второй струне указательным пальцем левой руки лады 1-й, 3, 5, 6, 8, 10, 12, 13 и одновременно с этим защипывайте правой рукой вторую струну. Прозвучит гамма ДО-мажор. От ноты ДО первой октавы до ноты ДО второй октавы. 
    
    Делаем выводы.
    \begin{itemize}
        \item Неудобно: сдвиг руки вдоль по грифу --- слижком грубое движение, о быстрой игре можно забыть.
        \item Неэкономно: музыка --- не спорт, а если так махать руками, то понадобится допинг.
        \item Непрактично: возможности гитары не используются в полной мере, ведь нужные ноты можно найти поблизости --- на соседних струнах.
    \end{itemize}
\end{Example}

Два экономных варианта исполнения гаммы ДО-мажор приведены на рисунке \ref{fig:harmony:scales:c:dur1}. Левая рука двигается только поперек грифа и каждый палец отвечает только за свой лад. Аппликатура слева --- гамма в одну октаву, а спрва --- двухоктавная гамма. В серых кружочках --- местах прижатия струны, написано латинское обозначение ноты, а не номер пальца (так как пальцы вдоль грифа не сдвигаются, то в этом нет необходимости). Стрелочками показан порядок постановки пальцев на струны при игре <<по восходящей>> --- дойдя до конца, играйте в обратной последовательности.

\begin{figure}[!ht]
    \centering
    \includegraphics[width=\textwidth]{fig/intervals/c-dur-csale-1} 
    \caption{Два варианта аппликатур гаммы ДО-мажор}\label{fig:harmony:scales:c:dur1}
\end{figure} 

Стоит отметить, что во время испольнения не следует снимать (поднимать над грифом) раньше времени пальцы, которые можно оставить\footnote{Это принцип экономии: не делай лишних движений, расслабь и оставь на месте палец, который не нужно двигать. Новичку, пока нет растяжки и независимости в пальцах левой руки, это будет трудно: поднимаешь один палец, а за ним сам собой поднимается второй. Так что на начальных этапах этим правилом (а куда деваться?) придется пренебрегать. Пройдет время, организм поймет куда нужно эволюционировать и вы почувствуете, как экономность движений доставляет удовольствие}

Если внимательно поискать на грифе места (рисунок \ref{fig:guitar:notes-on-griph-lat} в помощь), где еще можно сыграть гамму ДО-мажор, не растопыривая пальцы слишком широко и не смещаясь вдоль грифа, то можно найти вариант, представленный на рисунке \ref{fig:harmony:scales:c:dur2}.

\begin{figure}[!ht]
    \centering
    \includegraphics[width=\textwidth]{fig/intervals/c-dur-csale-2} 
    \caption{Аппликатура гаммы ДО-мажор длиной в одну октаву}\label{fig:harmony:scales:c:dur2}
\end{figure} 

Но смещаться по грифу на практике всё-таки придется. Вариант двухоктавной гаммы ДО-мажор от Андреаса Сеговии приведен на рисунке \ref{fig:harmony:scales:c:dur:segovia}. В кружках, как и положено, написаны номера прижимающих пальцев левой руки. Фрагменты этого аппликатурного шаблона вы можете увидеть по-отдельности на рисунках \ref{fig:harmony:scales:c:dur1} и \ref{fig:harmony:scales:c:dur2}. На третьей струне придется испытать неудобства: после того, как 3-м пальцем на 4-м ладу сыграна нота СИ(B), нужно быстро и точно переставить (или <<съехать>>, если струна гладкая) кисть так, чтобы указательный палец встал на 5-й лад (ноту ДО(C)) --- дальше продолжаем в экономном режиме, как на рисунке \ref{fig:harmony:scales:c:dur2}.

\begin{figure}[!ht]
    \centering
    \includegraphics[width=\textwidth]{fig/intervals/c-dur-csale-segovia} 
    \caption{Аппликатура двухоктавной гаммы ДО-мажор}\label{fig:harmony:scales:c:dur:segovia}
\end{figure} 

А теперь небольшой бонус: вы научились играть не только варианты гаммы ДО-мажор. Сдвиньте, например, кисть на два лада вниз по грифу и сыграйте то, что помнят ваши руки. Вуаля: вы сыграли гамму РЕ-мажор!

Когда вы научитесь играть гамму ДО-мажор хотя бы в двух октавах (см. раздел \ref{ch:harmony:scales}) и запомните, где находятся на грифе 7-нот ДО-мажорной гаммы, то вы легко сыграете характерные гаммы для любого из перечисленных ладов в <<основных нотах>>, опираясь на известные аппликатуры ДО-мажорной гаммы. Достаточно лишь начать с нужной ноты и, в случае пентатоники, пропускать ненужные. Сверьтесь с рисунками \ref{fig:harmony:lad:modes} и \ref{fig:harmony:lad:pentatonic} и убедитесь:
\begin{itemize}
    \item Диатоника:
    \begin{itemize}
        \item Ионийский (мажорный) лад. Гамма ДО-мажор: <<C,D,E,F,G,A,B,C>>.
        \item Дорийский лад. РЕ-Дорийская гамма: <<D,E,F,G,A,B,C,D>>.
        \item Фригийский. МИ-Фригийская гамма: <<E,F,G,A,B,C,D,E>>.
        \item Лидийский. ФА-Лидийская гамма: <<F,G,A,B,C,D,E,F>>.
        \item Миксолидийский. СОЛЬ-Миксолидийская гамма: <<G,A,B,C,D,E,F,G>>.
        \item Эолийский (минорный). Гамма ЛЯ-минор: <<A,B,C,D,E,F,G,A>>.
        \item Локрийский. СИ-Локрийская гамма: <<B,C,D,E,F,G,A,B>>.
    \end{itemize}

    \item Пентатоника:
    \begin{itemize}
        \item Мажорная пентатоника. ДО-мажорная гамма: <<C,D,E,G,A,С>>.
        \item Египетская пентатоника. РЕ-Египетская гамма: <<D,E,G,A,C,D>>.
        \item Блюз-минор. МИ-блюз-минорная гамма: <<E,G,A,C,D,E>>.
        \item Блюз-мажор. СОЛЬ-блюз-мажорная гамма: <<G,A,C,D,E,G>>.
        \item Минорная. ЛЯ-минорная гамма: <<A,C,D,E,G,A>>.
    \end{itemize}
\end{itemize}

Ого! И все это на основе одной гаммы! А теперь представьте, что получившийся аппликатурный шаблончик можно сдвинуть вдоль грифа! Любой лад почти в любой тональности! Фантастика. Еще раз подтверждаем тезис, что музыкальная теория сложнее практики.

Справедливости ради нужно отметить, что если вы стали фанатом какого-либо лада, то можно посидеть над грифом гитары или над рисунком \ref{fig:guitar:notes-on-griph-ru} (или \ref{fig:guitar:notes-on-griph-lat}) и разработать наиболее эффективную аппликатуру для игры в любимой тональности.

%\section{Чем больше звуков, тем лучше? Аккорды}
\label{ch:harmony:chords}

Из раздела \ref{ch:harmony:interval} вы узнали, что \emph{интервал} для физиков --- это \emph{мера расстояния} между двумя звуками, а для лириков --- это \emph{два звука}, сыгранных одновременно или друг за другом.

Аккорд же --- понятие целиком лирическое.

\begin{Definition}[Аккорд]
    \emph{Аккорд} --- это три или более музыкальных звука, извлеченных (звучащих) одновременно.
\end{Definition}

Видели ведь, наверное, как играют <<боем>>? Правая рука ходит вверх-вниз, и ритмично бьет по всем струнам одним или несколькими пальцами. Говорить о том, что звуки извлекаются одновременно не приходится --- струны задеваются одна за другой, но так как это происходит достаточно быстро, то \emph{звучат} они некоторое время все вместе и мы слышим \emph{аккорд}.

Какие же звуки должны входить в аккорд, чтобы он звучал гармонично? Точно не любые. Есть ли правила построения аккордов?

Да, система аккордов сложилась и давно устоялась. В основе этой системы лежат мажорный и минорный лады, и она определяет интервальную структуру аккордов и правила их именования. Кстати, поэтому аккорды делятся на две большие группы: мажорные (веселые) и минорные (грустные).

TODO

% интервальная структура аккорда и названия
% как декодировать этот шифр --- обозначения аккордов?

% TODO: струн шесть - звука три
% пример аппликатурного бокса

% ходовые аккорды в первой позиции

% аппликатуры простейших аккордов, которые можно унести с баррэ в любое место грифа.



\section{Так квинтовый или квартовый? Квинто-квартовый круг}
\label{ch:harmony:kvinto-kvarto-round}

На самом деле полное название этого полезного помошника, которго легко сделать из бумаги: <<квинто-квартовый круг мажорных и минорных тональностей>>. Выглядит он странно: см. рисунок \ref{TODO}. И хотелось бы не только научиться им пользоваться, но и понять почему он именно такой.

TODO рисунок

Круг может помочь, если вы хотите:
\begin{itemize}
    \item Подобрать <<сочетающиеся>> аккорды для аккомпанемента песен.
    
    \item Определить, какие ноты входят в ту или иную мажорную или минорную тональность.
    
    \item Сменить <<тональность>> аккомпанемента песни. Человеческим языком: \emph{каждую} ноту исходной тональности нужно повысить или понизить на заданное количество полутонов. Зачем? Чтобы было удобнее петь. При этом интервальная структура (т.е. характер, например, веселый или грустный) музыки не изменится, а произведение <<в целом>> будет звучать ниже или выше, <<подстраиваясь>> таким образом под голос исполнителя.    

    \item Решить <<академические>> задачки, например, определить тональность произведения по нотам или определить параллельную тональность.
    
    \item И много чего ещё\ldots
\end{itemize}

Мы не будем сейчас учиться как решать эти задачки, мы начнем разбираться с устройством круга. А в процессе станет понятно не только, как решать эти задачки, но и многое другое.

Глядя на рисунок \ref{fig:harmony:interval:octave-kon-dis} можно увидеть, что для отдельно взятого звука в пределах октавы существует лишь \emph{два} звука, образующий с исходным звуком совершенный консонанс. Относительно исходного эти звуки находятся на расстоянии в 5 (кварта) и 7 (квинта) полутонов. Оказывается, что можно переупорядочить ноты так, чтобы совершенные консонансы стали соседями исходного звука: один справа, другой слева!

Начнём, например с ноты ДО, и отступая по семь полутонов по часовой стрелке:
\[
    C\rightarrow 
    G\rightarrow 
    D\rightarrow 
    A\rightarrow 
    E\rightarrow 
    B\rightarrow 
    {F\sharp}\rightarrow
    {C\sharp}\rightarrow
    {G\sharp}\rightarrow
    {D\sharp}\rightarrow
    {A\sharp}\rightarrow
    F\rightarrow 
    C
\]
замкнем круг и получим результат, представленный на рисунке \ref{fig:harmony:kvinto-kvarto:kons-rearrange}.

\begin{figure}[!ht]
    \centering
    \includegraphics[scale=0.7]{fig/kvinto-kvarto/kons-rearrange} 
    \caption{Консонансы по соседству}\label{fig:harmony:kvinto-kvarto:kons-rearrange}
\end{figure} 

Теперь достаточно ткнуть в любую ноту на получившемся круге и узнать её совершенные консонансы: соседом против часовой стрелки будет совершенный консонанс на расстоянии 5 полутонов (чистая кварта), а по часовой --- консонанс на расстоянии 7 полутонов (чистая квинта). Например, возмьем ноту ЛЯ(A) и сразу определяем консонансы: РЕ(D) --- кварта от ЛЯ и МИ(E) --- квинта.

Обратите внимание на то, что среди переупорядоченных нот явно выделились две цепочки: цепочка нот без диезов и бемолей и цепочка нот со знаками альтерации. Конечно, это не случайность: ноты без диезов и бемолей --- это названия ступеней мажорного лада (а точнее --- тональность ДО-мажор). И само-собой, мажорный лад в своё время был сформирован с учетом расположения совершенных консонансов и вполне возможно, автор при этом глядел на полученный нами круг.

Итак, ноты тональности ДО-мажор выстроились в одну цепочку. При этом тоника --- ДО, идет в этой последовательности второй, если считать по часовой стрелке.

Таким образом упростилась задача определения нот для любой мажорной тональности: 
\begin{enumerate}
    \item нужно отметить тонику на круге и отступить от нее на один сектор против часовой стрелки;
    \item включая полученную ноту, двигаясь по часовой стрелке, отсчитать семь нот тональности.
\end{enumerate}

Например, требуется определить ноты тональности РЕ-мажор (см. рисунок \ref{fig:harmony:kvinto-kvarto:d-maj}). Отступаем от $D$ против часовой стрелки, а затем по часовой собираем 7 нот:
\[
    G\rightarrow 
    D\rightarrow 
    A\rightarrow 
    E\rightarrow 
    B\rightarrow 
    {F\sharp}\rightarrow
    {C\sharp}
\]

\begin{figure}[!ht]
    \centering
    \includegraphics[scale=0.5]{fig/kvinto-kvarto/kvinto-kvarto-d-maj} 
    \caption{Ноты тональности D-maj}\label{fig:harmony:kvinto-kvarto:d-maj}
\end{figure} 

Естественно, что ноты по высоте пока не упорядочены, но это совсем несложно сделать, зная, что тоника внизу:
\[
    D\rightarrow 
    E\rightarrow 
    {F\sharp}\rightarrow
    G\rightarrow 
    A\rightarrow 
    B\rightarrow 
    {C\sharp}
\]

